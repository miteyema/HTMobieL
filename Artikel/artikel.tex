%%%% ijcai11.tex

\typeout{IJCAI-11 Instructions for Authors}

% These are the instructions for authors for IJCAI-11.
% They are the same as the ones for IJCAI-07 with superficical wording
%   changes only.

\documentclass[a4paper]{artikel3}
% The file ijcai11.sty is the style file for IJCAI-11 (same as ijcai07.sty).
\usepackage{ijcai11}

% Use the postscript times font!
\usepackage{times}


% the following package is optional:
%\usepackage{latexsym} 

%%%%%%%%%%%%%% CUSTOM BEGIN %%%%%%%%%%%%%%

% remove if not a draft
%\usepackage{draftwatermark}

\usepackage[english]{babel}
\usepackage[utf8]{inputenc}
\usepackage[colorlinks]{hyperref}
\usepackage{multirow}
\usepackage{pgfplotstable} %nodig voor CSV to Latex
\usepackage{booktabs} % voor layout comparisontable

% engelse term die we niet vertalen naar het nederlands
\newcommand{\term}[1]{\emph{#1}}

% code commande
\newcommand{\code}[1]{\texttt{#1}}

% zodat we niet http:// staan hebben in onze tekst, maar de link wel werkt
\renewcommand{\url}[1]{\href{http://#1}{#1}}

% paragraphs anders te feel white space
\usepackage{setspace}
\renewcommand{\paragraph}[1]{\vspace{2mm} \noindent {\bf #1}  }
\newcommand{\framework}[2]{ \emph{#1 (\textbf{#2}): }} %enkel in usage stukje om framework in te leiden

\newcommand{\challenge}[1]{\paragraph{#1}}

%%%%%%%%%%%%%% CUSTOM END %%%%%%%%%%%%%%

% Following comment is from ijcai97-submit.tex:
% The preparation of these files was supported by Schlumberger Palo Alto
% Research, AT\&T Bell Laboratories, and Morgan Kaufmann Publishers.
% Shirley Jowell, of Morgan Kaufmann Publishers, and Peter F.
% Patel-Schneider, of AT\&T Bell Laboratories collaborated on their
% preparation.

% These instructions can be modified and used in other conferences as long
% as credit to the authors and supporting agencies is retained, this notice
% is not changed, and further modification or reuse is not restricted.
% Neither Shirley Jowell nor Peter F. Patel-Schneider can be listed as
% contacts for providing assistance without their prior permission.

% To use for other conferences, change references to files and the
% conference appropriate and use other authors, contacts, publishers, and
% organizations.
% Also change the deadline and address for returning papers and the length and
% page charge instructions.
% Put where the files are available in the appropriate places.

\title{Comparative study of frameworks for \\ the development of mobile HTML5 applications}
\author{Tim Ameye \\ tim.ameye@student.kuleuven.be \And Sander Van Loock \\ sander.vanloock1@student.kuleuven.be}
\begin{document}

\maketitle

%\begin{abstract}
% TODO
%\end{abstract}

\section{Introduction} % (1 blz)
\label{sec:introduction}
The mobile landscape is very heterogeneous.
We start by exploring the variety of mobile devices.
Under the hood of these devices, we briefly look at the mobile operating systems that run on them. 
Next, we explore the ways to make mobile applications.
Because of the emphasis on mobile HTML5 applications, we finally look at the mobile web browsers and the three building blocks of such applications: HTML5, CSS3 and JavaScript.

\subsection{Mobile devices}
Since the release of the Apple iPhone in 2007~\cite{David2011}, the demand for smartphones is ever since increasing. 
Today, over one billion smartphones are in use globally~\cite{Yang2012}.
This will double by 2015~\cite{Gillett2012}.
Not only smartphones, but also tablets conquer market share as mobile devices.
\cite{Gillett2012} states that 760 million tablets will be in use globally by 2016.
Not only Apple is producing mobile devices.
Other companies like Google and Microsoft are on the track too.
It may not come as a shock that all those mobile devices come in different shapes and sizes.
\subsection{Mobile operating systems}
In this section two mobile operating systems (OS) with the greatest market share are discussed: iOS and Android~\cite{David2011, Hales2012}.

\paragraph{iOS}
This mobile OS from Apple first appeared in 2007.
Since then, a new version was released every year with the most recent version being iOS~6 which was released in September~2012~\cite{Deitel2012, PhilDutson2012}.
 
\paragraph{Android}
In 2005, Google bought Android Inc. and released its first stable mobile OS in 2008~\cite{Satyesh2012}.
Like with iOS, new versions were released on a yearly basis.
Android~4.2 is the latest version and was released in October 2012~\cite{Sawers2012}.

\subsection{Mobile applications}
There are three possible ways to make a mobile application~\cite{Accenture2012,Hales2012}.
The first one is a web (HTML5) application which is run entirely in the web browser.
The second type of application is a native application which is installed on the device.
Lastly, a mix of both can be made which is called a hybrid application.

Advantages~\cite{Accenture2012} of web applications are the presence of a web browser on mobile devices, so everyone can use it directly regardless of there mobile OS and with installing the app first.
Secondly, the code need only to be written once in contrast to native, were a codebase has to be maintained for every mobile OS.
Lastly, web applications do not need to be verified by a store (like the App Store for iOS applications) prior to release.

In contrast, native applications provide~\cite{Accenture2012} better performance.
Secondly, it is easier to use features of the mobile device (like GPS and camera) on a native application.
Third is a security issue were web applications lack behind.
Lastly, publishing applications through a store increases publicity.

Hybrid applications seem to patch the issues of web application with the benefits of native application.
However, as of December 2012, HTML5 went into "Candidate Recommendation"~\cite{Jacobs2012}.

\subsection{Mobile web browsers}
As of 2008, we speak of the mobile web~\cite{Hales2012}.
We access the Internet more and more form our browser on our mobile device.
These browsers are becoming platforms on their own because to give access to the features of the device like GPS and camera. 

\paragraph{Mobile Safari}
This is the mobile browser from Apple shipped with iOS.
Apple did make sure to catch up with the latest HTML5 specifications in its browser~\cite{Hales2012}.

\paragraph{Android browser}
The implementation of HTML5 specifications has dragged, but as of Android 4.0, this is a lot better~\cite{Hales2012}.
It is now also possible to install the mobile version of the Chrome desktop browser.

\subsection{HTML5, CSS3 and JavaScript}

\section{Comparison criteria and related work}
\label{sec:comparisoncriteria}

%TODO POC inleiden.

Many in-depth comparisons of HTML5 mobile frameworks already exists today.  
However, none of them are scientifically published or use a large proof of concept (POC) to validate their comparison.  
The idea of using a small-scale POC is not new.  
\cite{Oeflman2011} and~\cite{Kosmaczewski2012},  for example,  uses a small geosocial game and todo list application respectivly to compare mobile HTML5 frameworks.
Blog posts like~\cite{Sarrafi2012a,Ayuso2012,Rozynski2011} all have their own criteria and methodology to assess different mobile frameworks.  
The overall applications of the criteria changes from use case to use case.  
\cite{Rozynski2011} presents the chosen criteria and discusses each for each framework.  
\cite{Ayuso2012} presents a whole bunch of criteria but all of them are discusses at once per framework.  
Thereafter,  advantages and disadvantages are subtracted and proposed to the reader.  
\cite{Sarrafi2012a} finally,  presents their chosen criteria together with a scorecard and explanation of scores per criteria.  
Each framework gets evaluated based on the scores for each criteria.

All these blog posts compare more than two frameworks where some of them are mobile and some are hybrid.  
Other websites like~\cite{Bristowe2012,Burris} only focus on two mobile HTML5 frameworks while on the other side of the spectrum~\cite{Falk2011} tries to compare as much as frameworks as possible in a large tabular form.

% \begin{table}[h!]
% \centering
% \begin{tabular}{lcc}
%   & \textbf{Hybrid} & \textbf{Mobile}\\
%   \hline \hline
% \cite{Sarrafi2012a} & 0 & 4\\
% \cite{Rozynski2011} & 2 & 3\\
% \cite{Ayuso2012}* & 1** & 6\\
% \hline
% \multicolumn{3}{l}{*is still under development} \\
% \multicolumn{3}{l}{**a combination of Bootstrap of jQuery and Angular JS} \\
% \end{tabular}
% \caption{How many frameworks of which type are compared}
% \label{table:references_frameworks}
% \end{table}

As mentioned earlier,  we will compare four HTML5 mobile frameworks based upon a large-scale POC.  
Our strategy tries to combine some elements that can be found in the literature.  
We derived five important criteria:  community,  productivity,  usage,  support and performance.  
A methodology to measure these criteria will be described below.  
A score will be assigned to each criteria and will be plotted in a spider graph~\cite{Few2005}.   
The POC will drive the methodologies to test the criteria if possible.  
To be complete,  a full detailed table will be provided where the significant difference of the frameworks will be presented. 	

\paragraph{Community}
Social networks provide a good indication to determine the community that goes with the framework.  
\cite{Sarrafi2012a,Ayuso2012} use Twitter followers,  Stack Overflow questions,  GitHub stars and GitHub forkers as community indication.  
We would include Facebook likes also.  The summation of these five numbers give the score for the framework community.  

\paragraph{Productivity}
We both implement the full POC in two different frameworks and carefully record the time to implement it.  
As we both start similar background,  time differences can explain different learning curves.  
To double check this criteria,  we also partially implement the POC in the other two frameworks.  
Just like~\cite{Burris},  the partially implementation can be the POCs login screen.  
The summation times to implement the POC and login screen give the score for the frameworks productivity.  

\paragraph{Usage}
The POC will be subdivided in a number of implementation challenges.  
To score of this criteria will depend on the completion of the challenges by the framework.  
The assessment of the challenges can be found in table~\ref{table:challenges-scores}.  
The summation of all challenges score give the score for the frameworks usage.

\begin{table}[h]
\centering
\begin{tabular}{l|l}
\textbf{Score} & \textbf{Assessment criteria}\\
  \hline \hline
2 & Supported by the framework\\
1 & A plug in is needed\\
0 & Custom implementation or hack\\
\end{tabular}
\caption{Assessment criteria for implementation challenges}
\label{table:challenges-scores}
\end{table}

\paragraph{Support}
This criteria indicates the correct functionality of the POC implementation on different devices and browsers,  similar as the cross platform capabilities of~\cite{Sarrafi2012a}.  
In each context,  we will walk trough the scenario and rate the execution.  
If some parts of the scenario are infeasable,  points will be subtracted.  
The summation of scores for every iteration of the scenario in each context will give the score for the frameworks support.

\paragraph{Performance}
Different aspect of the framework influence the performance.  
We will test the time to load and render the full POCs and isolated screens with Goolge Page Speed~\cite{Google2012},  as suggested by~\cite{Morgan2011}.   
Also,  isolated tests will record the loading time of dummy pages with 1000 buttons.  
The summation of render times will give the score for the frameworks performance.

\section{Frameworks} % (1 blz per framework + tabel)
\label{sec:frameworks}

\subsection{jQuery Mobile} % (1 blz Tim)
\label{sec:jqm}

jQuery Mobile (jQM) is a mobile HTML5 user interface~(UI) framework that was announced in 2010~\cite{Resig2010}. 
In November 2011 version~1.0 was released~\cite{Parker2011} and one year later in October, version~1.2 was released~\cite{Parker2012}. 
As at the time of writing, jQM released version~1.3 \cite{Parker2013a}.
The framework is controlled by the jQuery Project that also manages jQuery Core. 
The latter is a JavaScript library where jQM is dependent on~\cite{JQuery2012}. 
jQM is among other things sponsored by Adobe, Nokia, BlackBerry and Mozilla~\cite{JQuery2012a}.

\paragraph{Licence}
As of September 2012 it is only possible to use jQM under the Massachusetts Institute of Technology~(MIT) licence~\cite{Dmethvin2012}. 
This means that the code is released as open source and can also be used in proprietary projects~\cite{PhilDutson2012}.

\paragraph{Documentation}
One can find the documentation of jQM on \url{www.jquerymobile.com/demos/1.2.0}. On the one hand it contains an overview of all possible UI components. 
By checking the source code, you can find out what code to write to get the same result. 
On the other hand it explains the API on how to configure defaults, use events, methods, utilities, data attributes and theme the framework \cite{JQuery2012b}.

\paragraph{Code and development}
jQM is a UI framework and thus provides mainly UI components. 
jQM provides 6 categories of components: pages and dialogs, toolbars, buttons, content formatting, form elements and litsviews~\cite{JQuery2012b}. 
One can obtain these components by writing HTML5 with jQM specific \code{data-}* attributes. 
When running the application, jQM will add the extra necessary code to correctly show these components by doing progressive enhancement.

There are three ways of writing a web application in jQM~\cite{Broulik2012}. 
The first one is to write the full application that is composed of many screens, on one single web page.
The advantage is that there are initially less requests to the server.
The second option is to write a web page for each screen. 
The advantage here is that the first viewed screen is downloaded more quickly. 
However, with each transition, the next screen has to be fetched which can delay navigation.
This is done with AJAX by default in jQM.
Lastly, you can mix the two above to find an optimum by putting the most likely viewed screens on one web page and the less likely viewed on separated pages.  

\paragraph{Browser support}
\label{sec:jqm-browser-support}
jQM divides browsers into three grades: A, B and C. 
An A graded browsers supports everything of the jQM framework, where a C graded browsers only provides basic HTML experience (so for example no CSS3 transitions)~\cite{JQuery2012d}.

\subsection{Sencha Touch} % (1 blz Sander)
\label{sec:sencha_touch}

Sencha Touch (ST) is a framework developed by Sencha,  a company founded in 2010 as a composition of Ext JS, jQuery Touch and Raphaël.  
Ext JS is a JavaScript framework for the development of web applications.  
jQuery Touch is a jQuery plug in for mobile development that adds touch events to jQuery and depends on the WebKit engine.  
Finally,  Raphaël, is a JavaScript library for vector drawings.  
Pieces of the first two technologies can be found in the implementation of ST framework.    

As at the time of writing,  ST is at version 2.1.1~\cite{Inc.}.

\paragraph{Documentation}
All documentation for ST 2.1.1 can be found at \url{docs.sencha.com/touch/2-0}.  
The most important features,  are provided with code examples and an example of the code after rendered by the browser.  
The key concepts of ST are explained in extensive tutorials:  some texts, some videos.  

Another handy tool to discover the ST features is the 'Kitchen Sink'~\cite{Inc.2013}.  
This is a web application,  written in ST,  that lines up all possibilities of the framework combined with the corresponding code.

\paragraph{Licenses}
ST is free within a commercial context in which the developer does not share the code with its users.  
There is also the option to use an open source version.  
This comes with a GNU GPL v3 license which implies a free code redistribution as most important property.
More detailed licenses can be found at~\cite{SenchaInc.}.
  
\paragraph{Code and development}
ST is written on top of Ext JS,  and can also be considered as JavaScript framework.  
All code needs to be written in JavaScript and loaded by one HTML container.  
An other important aspect of ST is that is supports the Model-View-Controller (MVC) pattern.  
Models group fields to data-objects,  views define how the content is presented to the user and controllers connect these based on events.

ST contains all UI elements as JavaScript objects.  
Just like object-oriented programming,  those objects are part of a class system.  
Classes can both be defined (\code{Ext.define}) or created (\code{Ext.create}).  
Single-inheritance and overriding is also possible.    

To enhance performance,  it is the programmers task to create components before they are used.  
In this way,  the programmers partially determines the performance of the applications.

\paragraph{Browser support}
Just like jQuery Touch,  ST is based upon the WebKit browser engine.  
This forms the major requirement for browser support.  
Although most mobile browsers contain this engine,  some like FireFox Mobile and Opera Mobile lack behind~\cite{JohnEClark2012}.  
Following~\cite{Wokke2013}, the next release of the Opera browser will contain this engine,  a trend that most browser vendors will (have to) follow.

The framework offers the programmer methods to ask for the current context where the end-user is working in. 
Properties like \code{Ext.env.Browser} and \code{Ext.env.OS} or methods like \code{Ext.Viewport.getOrientation} and \code{Ext.feature.has} can determine this context~\cite{JohnEClark2012}.  
The latter has functionalities,  just like Modernizer~\cite{Modernizr2012}.  

\section{Evaluation}

\subsection{Community} % (0,5 blz)
\label{sec:community}

The activity around frameworks in social media has to be considered with a lot of care.  
If,  for example,  a framework has a lot of questions on Stack Overflow,  one could conclude the framework has a lot ambiguities.  
Someone else could say,  the framework has a large active internetcommunity build around it.  
Also,  the numbers of Twitter followers and Facebook likes only give an indication of the popularity of the framework,  there is no exact science involved.

As depicted in the comparison table in appendix \ref{app:comparison_table},  jQM has a much larger social community.  
An explanation could be explained by the large success of jQuery,  a popular JavaScript framework itself~\cite{Hales2012}.

The fact that ST had the first stable 1.0 release in mobile UI frameworks~\cite{Oeflman2011},  is not reflected in the community numbers.  
Alse,  ~\cite{Inc.} states that Sencha is the largest provider of open-source web application frameworks with more than two million developers.  
ST itself embraces about one quarter of this number.  Only a tiny fraction of the developers 

ST is not available on GitHub,  that is why no data is given.

\subsection{Usage}
\label{sec:poc}
In this section we evaluate the usage by stating challenges.
Each challenge consists of subchallenges that are explained first.
Afterwards the framework is graded on each of these challenges and commented why that score was given.

\challenge{Forms (C1,C2,C3,C4,C5)}
Create forms with placeholders, but without labels (C1).
Use the text, number and email as form types (C2).
Custom date picker with data range (C3).
Custom date picker only with month and year (C4).
Clearing the form (C5).

\framework{jQM}{2,2,1,1,0}
Placeholders can be achieved with the HTML5 \code{placeholder} attribute.
Labels are mandatory in jQM for assistive technologies, but can be hidden via the \code{ui-hide-label} class~\cite{JQuery2013}. 
The types of form elements used were \code{text}, \code{number} and \code{email}. %TODO redundant?
The \code{date} type could not be used, because of the lack of support in mobile browsers~\cite{Deveria2013b}.
Not only the lack of support, but also the need to specify a range the user can choose from, justifies the use of the Date \& Time Picker of Mobiscroll~\cite{Mobiscroll2013}.
The need for a custom date picker with only a month and year needed to be hardcoded.
Clearing the form after it was send, has to be programmed manually.

\framework{ST}{2,2,0,0,0}
Placeholders,  text fiels, email fields, number field are supported by the framework and can be easily created.  
Labels can be avoided by not defining them.  
Creating custom datepickers as needed in the POC is not supported.  
It is impossible to ignore the days field and only years can be delimited.  
Clearing the form after it was send, has to be programmed manually.

\challenge{Form filling (C1,C2,C3)}
Filling in the form elements with data (C1).
Make the form elements read-only (C2).
%Back button to go back to the list (C3). %TODO niet van toepassing

\framework{jQM}{0,0}
There is no automatic mapping between the data and the form.
Setting the form elements read-only can be done easily for the types \code{input} and \code{textarea} with the \code{readonly} attribute.
With \code{fieldset}, the \code{disabled} attribute was the solution.
Lastly with \code{select}, the other options were deleted from the list.
%Adding a back button was no problem.

\framework{ST}{2,2}
%To show the list of clickable expenses,  ST provides a navigation view.  
%This mechanism allows to create the list of expenses together with a handler that gets triggers after an item selection.  
The controller can fill an empty form with a model instance.
Field names that correspond to the objects properties are linked and filled in.
Setting the value of the correct radiobutton does not work,  a new fieldset has to be created on which the \code{setGroupValue}-method needs to be applied.  
Making form elements read-only can be done by setting the \code{readOnly} or \code{disabled} property to true.
%A back button is automatically created to return to the overview list.

\challenge{Form validation (C1,C2,C3,C4)}
Validation rules for a required, number and email field (C1).
Validation depends on custom conditions (C2).
Error dialog with invalid fields and explanation (C3).
Red borders around invalid fields (C4).

\framework{jQM}{1,1,0,0}
Because of the lack of support on mobile browsers for the \code{required} attribute~\cite{Deveria2013}, the plugin of Zaefferer~\cite{Zaefferer2013} was used to check for required fields.
This plugin also provides built-in validation methods that were needed: \code{number}, \code{email} and \code{date} and custom validation conditions.
The plugin shows errors below the corresponding field, so customisation of the plugin was needed to show them in a dialog.
To give the invalid fields a red border, CSS was used directly for \code{input} and \code{textarea}.
Because of the progressive enhancement for \code{select} and \code{fieldset}, a bit of DOM traversal was needed.

\framework{ST}{2,1,2,0}
Required fields and email validation can be assigned to a model.  
To add custom validation rules or messages,  the \code{validate}-methode of \code{Ext.data.Model} needs to be overriden.  
This method accepts a model instance and returns the possible validation errors.  
The errors must be iterated to concatenate the validation messages and color the red borders.
The borders are added by adding a custom CSS class to the form element.

\challenge{Signature (C1)}
Draw a signature with finger or pen (C1).

\framework{jQM}{1}
At first, Signature Pad~\cite{Bradley2013} was used as plugin.
Because of the time spend on changing the default layout, jSignature~\cite{Systems2013} was used instead.
An advantage over the former was the automatic scaling to 100\% width and the layout without the bells and whistles.
However, the plugin does not work on Android~2.3 and lower~\cite{Systems2013}.

\framework{ST}{1}
Drawing a signature with a pen or a finger is handled by a plugin~\cite{SimFla2011}.  
Plugins can easily be added to the framework by placing the plugin file in the \code{ux} folder and loading it in the main JavaScript file.  
This plugins could be used as-is by using the newly \code{signaturefield}.  

\challenge{Show PDF (C1,C2)}
Requesting a PDF with POST parameters (C1).
Show the PDF (C2).

\framework{jQM}{0,0}
The download the PDF, a hidden form is used because AJAX is not the preferred way for fetching raw data.
When the users taps an expense form in the list, this hidden form is submitted with the correct parameters to download the PDF.
The mobile device opens the corresponding native PDF viewer to show the PDF.

\framework{ST}{1,1}
A plugin for a PDF-viewer can be found at~\cite{Fiedler2012}.  
Some modifications were necessary to made it compliant with the POC.  
The PDF must be fetched from the backend with a parameterized POST request instead of a simple GET request.  
The plugin automatically creats views for every PDF page.  

\challenge{Attach image (C1,C2,C3)}
Choose image (C1).
Convert image to base64 (C2).
Preview image (C3).

\framework{jQM}{0,0,0}
A form element of type \code{file} is used to let the user attach an image.
Depending on the mobile device, the options are to attach a local image or to take one with the camera.
The attached image is read by the FileReaderAPI, converted to an image and imported on the canvas.
The base64 encoding can be retrieved by calling the \code{.toDataURL()} on the latter and is stored in session storage.
However the FileReaderAPI is not supported on Android~2.3 and lower or iOS less than 6.0~\cite{Deveria2013a}.
Another limitation is the file size that can be imported on a canvas on iOS devices which is depend on the RAM size~\cite{Apple2012}.
The final limitation is the size of the encoded evidence in base64, which can exceeded the limit for session storage, that is depend on the mobile browser~\cite{Gonzalez2012}.

\framework{ST}{1,1,1}
A plugin to upload files can be found~\cite{Smirnov2012} to create buttons with the \code{fileupload} xtype.  
This button enables users to select an image and passes it to a PHP file.  
This file uploads the image and converts it to base64.
A requirement is that your server is able to run that PHP file.    
A ST image can be created based on this base64 string.

\challenge{Autocomplete (C1,C2)}
Fetch suggestions based on input (C1).
Show suggestions in clickable drop down (C2).

\framework{jQM}{1,1}
As of version~1.3, jQM has built-in autocomplete support~\cite{JQuery2013c}.
Because the POC used 1.2, the plugin of Matthews~\cite{Matthews2013} was used.
To show only five suggestion, a little customisation by the \code{slice} function was needed because the plugin did not provided this.

\framework{ST}{1,0}
A plugin can be found at~\cite{Mysamplecode2012} to create an autoplugin field.  
A request to the backend returns a JSON array named \code{data} for each keyword.  
ST can parse this array is two ways:  with a \code{JsonReader} or an \code{ArrayReader}.  
The first requires that a JSON key precedes each item,  the latter assumes each item in the array maps to a field of a model.  
Both strategies can not be used to parse the array and create seperate model instances for each array item.  
This implies that no clickable drop down could be implemented.


\challenge{AJAX: text, JSON \& XML}
Fetch plain text (C1).
Fetch and parse JSON (C2).
Send JSON payload (C3).
Fetch and parse XML (C4).

\framework{jQM}{2,2,2,2}
AJAX requests are made via the jQuery library where jQM is dependent on.
Handling plain data and JSON as response is straightforward and the data can be used directly.
When using XML, the data has to be traversed like with HTML, this means by using selectors.
Sending JSON can be achieved by converting the instance into a string object using the vanilla JavaScript function \code{JSON.stringify} and add it as data to the jQuery AJAX request.

\framework{ST}{2,2,2,2}
Ajax request can be done either explicitly via a direct \code{Ext.Ajax.request}-call or implicitly via stores.  
The plain text of the AJAX response can be used in the callback.  
Stores abstract AJAX calls.  
They can be configured with a model to define the structure of the recorded objects.  
A proxy configures readers and writers that define where the data can be read or written.  
This can be locally at the client side or via a remote server.  
Readers and writers contain the format of the data - JSON or XML - and automatically parse this data to fields of the corresponding model. 
Sending JSON payload must be done via an AJAX request where the \code{jsonData} is encoded via \code{Ext.encode}.

\challenge{Device-specific layout (C1,C2,C3)}
Recognising smartphone or tablet (C1).
Show left sided menu on tablet (C2).
Enabling smartphone menu (C3).

\framework{jQM}{0,0,0}
No functionality is provided by the framework, so a search for plugins was started with~\cite{Deering2012} which led to: Splitview~\cite{Rahman2013}, SimpleSplitView~\cite{Yared2013} and Multiview~\cite{Franck2012}. 
All these had shortcomings. The first changed the jQM library code, the second uses jQM~1.0.1 and the last had problems with adapting to different browser sizes.
\cite{Hadlock2012} showed the use of CSS3 media queries to accomplish the split screen without a plugin.
The documentation of jQM~1.2 also used a similar approach~~\cite{JQuery2012b}.
This approach is also encouraged in the documentation of jQM~1.3~\cite{JQuery2013e}.
The smartphone menu can be accessed on both tablet and smartphone devices when clicking on the sub header.

\framework{ST}{2,2,0}
Detection of a smartphone context is done via the \code{Ext.os.is.Phone}-method.  
If this method returns false,  we assume to be in tablet mode.    
The main screen of the POC requires a splitted view in tablet mode.  
A \code{vbox} layout splits the viewport with a vertical axis.  
The \code{flex} property defines the ratio of the sizes of both resulting components.  
%TODO niet meer relevant The right component is the page that changes by clicking the menu items in the left component.  
%This can be realized via a \code{card} layout where changing the page implies setting the active component of the \code{card} layout.
In smartphone mode,  the left screen is made invisible and an extra menubutton in the header is created.  
Making the sub header clickable is not possible  


\section{Future work} %(0,5 blz)
\label{sec:future_work}

We left out three other criteria (productivity, support and performance) that we did not yet evaluated.
Also we left out some other challenges in the evaluation of the usage, namely:

\challenge{Offline (C1,C2,C3)}
Saving credentials (C1).
Saving unsent expenses (C2).
Making app offline (C3).

\challenge{Load screen and dialog (C1,C2)}
Show load screen (C1).
Show dialog (C2).

\challenge{Data conversion (C1,C2)}
Currency conversion (C1).
Id to string (C2).

\challenge{List (C1,C2,C3)}
Load data into list and styling list items (C1).
Clickable items with custom action or linking to record of item (C2).
Sorting data (C3).

\challenge{Anatomy of page (C1,C2,C3)}
 Show header, footer and sub header with titles and buttons (C1).
 Tabbar (C2).
 Button color customisation (C3).

\section{Conclusion} %(0,5 blz)
\label{sec:conclusion}

%% The file named.bst is a bibliography style file for BibTeX 0.99c
%\bibliographystyle{named}
\bibliographystyle{abbrv}
\bibliography{literatuurstudie-tex,vergelijking-tex,evaluatie-tex}

\appendix
\section{Comparison table}
\label{app:comparison_table}
%CSV to latex tutorial in blogpost: http://texblog.org/2012/05/30/generate-latex-tables-from-csv-files-excel/
%Manual: http://pgfplots.sourceforge.net/pgfplotstable.pdf

\begin{center}

\pgfplotstableset{
  begin table/.add={}{[t]
  %\label{table:comparisioncriteria}
  %\caption{The caption}%
  },
}

\pgfplotstabletypeset[
  col sep=semicolon,
  string type,
  header=true,
  columns={Criteria,jQuery Mobile,Sencha Touch},
  columns/Criteria/.style={column name=Criteria, column type={l}},
  columns/jQuery Mobile/.style={column name=jQuery Mobile, column type={c}},
  columns/Sencha Touch/.style={column name=Sencha Touch, column type={c}},
  every head row/.style={
    before row=\toprule,
    after row=\hline\hline},
  every last row/.style={
    after row=\bottomrule}
]{comparisontable.csv}


\end{center}
\end{document}