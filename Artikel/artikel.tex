%%%% ijcai11.tex

\typeout{IJCAI-11 Instructions for Authors}

% These are the instructions for authors for IJCAI-11.
% They are the same as the ones for IJCAI-07 with superficical wording
%   changes only.

\documentclass[a4paper]{article}
% The file ijcai11.sty is the style file for IJCAI-11 (same as ijcai07.sty).
\usepackage{ijcai11}

% Use the postscript times font!
\usepackage{times}


% the following package is optional:
%\usepackage{latexsym} 

%%%%%%%%%%%%%% CUSTOM BEGIN %%%%%%%%%%%%%%

% remove if not a draft
%\usepackage{draftwatermark}

\usepackage[english]{babel}
\usepackage[utf8]{inputenc}
\usepackage[colorlinks]{hyperref}
\usepackage{parskip}
\usepackage{multirow}

% engelse term die we niet vertalen naar het nederlands
\newcommand{\term}[1]{\emph{#1}}

% code commande
\newcommand{\code}[1]{\texttt{#1}}

% zodat we niet http:// staan hebben in onze tekst, maar de link wel werkt
\renewcommand{\url}[1]{\href{http://#1}{#1}}

% paragraphs anders te feel white space
\usepackage{setspace}
\renewcommand{\paragraph}[1]{{\bf #1} }

%%%%%%%%%%%%%% CUSTOM END %%%%%%%%%%%%%%

% Following comment is from ijcai97-submit.tex:
% The preparation of these files was supported by Schlumberger Palo Alto
% Research, AT\&T Bell Laboratories, and Morgan Kaufmann Publishers.
% Shirley Jowell, of Morgan Kaufmann Publishers, and Peter F.
% Patel-Schneider, of AT\&T Bell Laboratories collaborated on their
% preparation.

% These instructions can be modified and used in other conferences as long
% as credit to the authors and supporting agencies is retained, this notice
% is not changed, and further modification or reuse is not restricted.
% Neither Shirley Jowell nor Peter F. Patel-Schneider can be listed as
% contacts for providing assistance without their prior permission.

% To use for other conferences, change references to files and the
% conference appropriate and use other authors, contacts, publishers, and
% organizations.
% Also change the deadline and address for returning papers and the length and
% page charge instructions.
% Put where the files are available in the appropriate places.

\title{Comparative study of frameworks for \\ the development of mobile HTML5 applications}
\author{Tim Ameye \\ tim.ameye@student.kuleuven.be \And Sander Van Loock \\ sander.vanloock1@student.kuleuven.be}
\begin{document}

\maketitle

\begin{abstract}

\end{abstract}

\section{Introduction} % (1 blz)
\label{sec:introduction}
%iOS en Android + afbeeldingen van martkaandeel
%mobiele applicatie native/hybrid/mobile
%browsers

\section{Comparison criteria and related work}
\label{sec:comparisoncriteria}

Many in-depth comparisons of HTML5 mobile frameworks already exists today.  However, none of them are scientifically published or use a large proof of concept (POC) to validate their comparison.  Blog posts like~\cite{Sarrafi2012a,Ayuso2012,Rozynski2011} all have their own criteria and methodology to assess different mobile frameworks.  The overall applications of the criteria changes from use case to use case.  \cite{Rozynski2011} presents the chosen criteria and discusses each for each framework.  \cite{Ayuso2012} presents a whole bunch of criteria but all of them are discusses at once per framework.  Thereafter,  advantages and disadvantages are subtracted and proposed to the reader.  \cite{Sarrafi2012a} finally,  presents their chosen criteria together with a scorecard and explanation of scores per criteria.  Each framework gets evaluated based on the scores for each criteria.

All these blog posts compare more than two frameworks where some of them mobile and some are hybrid (see table~\ref{table:references_frameworks}).  Other websites like~\cite{Bristowe2012,Burris} only focus on 2 mobile HTML5 frameworks while on the other side of the spectrum~\cite{Falk2011} tries to compare as much as frameworks as possible in a large tabular form.

%TODO sommige posts ook samenvattende tabel?  Iets wat wij ook zelf ambieëren

\begin{center}
\begin{tabular}{lll}
  & Hybrid & Mobile\\
\cite{Sarrafi2012a} & 0 & 4\\
\cite{Rozynski2011} & 2 & 3\\
\cite{Ayuso2012}* & 1** & 6 \\
\multicolumn{3}{l}{*is still in development} \\
\multicolumn{3}{l}{**a combination of Bootstrap, jQuery and Angular JS}
\label{table:references_frameworks}
\end{tabular}
\end{center}

As mentioned earlier,  we will compare 4 HTML5 mobile frameworks based upon a large-scale proof of concept.  

%TODO mini poc?:  monocaffe, codefessions

%TODO ISO vermelden

\paragraph{Community}% Gemeenschap
% 
% De gemeenschap en populariteit van een raamwerk is makkelijk in cijfers uit te drukken. We voorzien een tabel waar we per raamwerk het aantal volgers op Twitter, watchers/forkers van GitHub en aantal likes van Facebook zullen onderbrengen. Een grafiek van Google Trends, die het zoekvolume op Google uitzet per tijd, zal voor elk raamwerk na deze tabel worden toegevoegd.
% De som van volgers, watchers en likes vormt de score voor het gemeenschapscriteria.
% 
\paragraph{Productivity}% Productiviteit
% 
% De productiviteit moet een indicatie geven hoe lang het duurt om met het raamwerk vertrouwd te raken. Hiervoor gaan we het feit uitbuiten dat we deze thesis met twee maken. Elk zullen we de proof-of-concept (POC) in twee verschillende raamwerken maken wat het totaal op 4 raamwerken brengt. De tijd die nodig is om de volledige POC te implementeren is een indicatie voor de productiviteit. Ook zal elk van ons een loginscherm maken in de andere twee raamwerken. Als Tim de POC implementeert in raamwerken A en B en Sander in raamwerken C en D, zal dus Tim ook een loginscherm maken in raamwerken C en D en vice versa.  Dit scherm bevat UI-elementen, validaties en backend integratie. We kunnen dit dus als voldoende steekproef beschouwen om ervaring met een raamwerk te testen. De tijd die ieder nodig heeft om dit scherm te bouwen, geeft ook een indicatie van de nodige leertijd.
% De som van de uren voor het implementeren van de POC en het loginscherm vormt de score voor de productiviteit.
% 
\paragraph{Usage}% Gebruik
% 
% Om het verschil in gebruik van de raamwerken te onderzoeken, zullen we steunen op de logboeken die we hebben bijgehouden tijdens het implementeren van de POC. We gaan er vanuit dat de POC relevante kenmerken bevat om de raamwerken zoveel mogelijk uit te buiten. We zullen bij het vergelijken de moeilijkheden groeperen zodat het snel duidelijk wordt welke functionaliteit in een bepaald raamwerk moeilijker/makkelijker te implementeren is. Een hogere score wordt toegekend wanneer bepaalde functionaliteit reeds aangeboden wordt door het raamwerk. Een lagere score betekent dat een plugin werd gezocht. Wanneer een hack noodzakelijk was om de functionaliteit te beogen, zal de laagste score worden toegekend.
% 
\paragraph{Support}% Ondersteuning
% 
% We zullen een scenario uitwerken waarbij we het gebruik van de POC doorlopen. Dit scenario zullen we op een reeks van verschillende mobiele apparaten herhalen. Bij het mislukken van een stap in het scenario, zal een punt worden afgetrokken. Hierdoor kunnen we een score bekomen hoe goed de applicatie, geïmplementeerd in een bepaald raamwerk, scoort op een bepaald apparaat. De gebruikte apparaten en versies van besturingssystemen en browsers zullen worden bepaald aan de hand van de marktaandeel en de beschikbaarheid van deze apparaten aan het departement HCI.
% Een kanttekening hierbij is of het raamwerk ondersteunt de webapplicatie om te vormen tot een native applicatie. Ook bekijken we of de native look-and-feel per besturingssysteem door het raamwerk kan worden benaderd.
% 
\paragraph{Performance}% Performantie
% 
% Een laatste criterium is de performantie. Dit zal worden uitgedrukt in de tijd nodig om te applicatie te renderen. Hierbij zullen we verschillende testen doen. Ten eerste zullen we gebruik maken van de volledige POC. We zullen kijken naar de rendertijd van de applicatie aan de hand van Google Page Speed. Daarnaast zullen we een geïsoleerde test doen met het verzenden van een AJAX request, omdat sommige raamwerken altijd eerst een OPTION request sturen. Verdergaand op het versturen van requests zullen we kijken hoe lang het duurt vooraleer een expense daadwerkelijk verzonden is naar de backend. Vervolgens zullen we ook geïsoleerde testen doen met UI-elementen. Hierbij zullen we een pagina voorzien van 1000 buttons en meten hoe lang het duurt tot deze pagina gerendert wordt. Als laatste zullen we een test op de aparte loginschermen uitvoeren. Hier kunnen we dan ook het aantal lijnen code bekijken en de bijhorende performantie. De redenen waarom we ook deze geïsoleerde applicatie gebruiken is omdat we ervan uitgaan dat niet de volledige POC in ieder raamwerk zal kunnen worden geïmplementeerd. Dit is in tegenstelling tot de geïsoleerde login applicatie, waarbij we exact kunnen vergelijken.

\newpage
\section{Frameworks} % (1 blz per framework + tabel)
\label{sec:frameworks}

\subsection{jQuery Mobile} % (1 blz Tim)
\label{sec:jqm}

jQuery Mobile (jQM) is a mobile HTML5 user interface~(UI) framework that was announced in 2010~\cite{Resig2010}. 
In November 2011 version~1.0 was released~\cite{Parker2011} and one year later in October, version~1.2 was released~\cite{Parker2012}. 
As at the time of writing, jQM will be releasing version~1.3 very soon~\cite{Parker2013}.
The framework is controlled by the jQuery Project that also manages jQuery Core. 
The latter is a JavaScript library where jQM is dependent on~\cite{JQuery2012}. 
jQM is among other things sponsored by Adobe, Nokia, BlackBerry and Mozilla~\cite{JQuery2012a}.

\paragraph{Licence}
As of September 2012 it is only possible to use jQM under the Massachusetts Institute of Technology~(MIT) licence~\cite{Dmethvin2012}. 
This means that the code is released as open source and can also be used in proprietary projects~\cite{PhilDutson2012}.

\paragraph{Documentation}
One can find the documentation of jQM on \url{www.jquerymobile.com/demos/1.2.0}. On the one hand it contains an overview of all possible UI components. 
By checking the source code, you can find out what code to write to get the same result. 
On the other hand it explains the API on how to configure defaults, use events, methods, utilities, data attributes and theme the framework \cite{JQuery2012b}.

\paragraph{Code and development}
jQM is a UI framework and thus provides mainly UI components. 
jQM provides 6 categories of components: pages and dialogs, toolbars, buttons, content formatting, form elements and litsviews~\cite{JQuery2012b}. 
One can obtain these components by writing HTML5 with jQM specific \code{data-}* attributes. 
When running the application, jQM will add the extra necessary code to correctly show these components by doing progressive enhancement.

There are three ways of writing a web application in jQM~\cite{Broulik2012}. 
The first one is to write the full application that is composed of many screens, on one single web page.
The advantage is that there are initially less requests to the server.
The second option is to write a web page for each screen. 
The advantage here is that the first viewed screen is downloaded more quickly. 
However, with each transition, the next screen has to be fetched which can delay navigation.
This is done with AJAX by default in jQM.
Lastly, you can mix the two above to find an optimum by putting the most likely viewed screens on one web page and the less likely viewed on separated pages.  

\paragraph{Browser support}
\label{sec:jqm-browser-support}
jQM divides browsers into three grades: A, B and C. 
An A graded browsers supports everything of the jQM framework, where a C graded browsers only provides basic HTML experience (so for example no CSS3 transitions)~\cite{JQuery2012d}.

\subsection{Sencha Touch} % (1 blz Sander)
\label{sec:sencha_touch}

Sencha Touch is a framework developed by Sencha,  a company founded in 2010 as a composition of Ext JS, jQuery Touch and Raphaël.  Ext JS is a JavaScript framework for the development of web applications.  jQuery Touch is a jQuery plug in for mobile development that adds touch events to jQuery and depends on the WebKit engine.  Finally,  Raphaël, is a JavaScript library for vector drawings.  Pieces of the first two technologies can be found in the implementation of Sencha Touch framework.    

Currently,  Sencha Touch is at version 2.1.1~\cite{Inc.}.

\paragraph{Documentation}
All documentation for Sencha Touch 2.1.1 can be found at \url{docs.sencha.com/touch/2-0}.  The most important features,  are provided with code examples and an example of the code after rendered by the browser.  The key concepts of Sencha Touch are explained in extensive tutorials:  some texts, some videos.  

Another handy tool to discover the Sencha Touch features is the 'Kitchen Sink'~\cite{Inc.2013}.  This is a web application,  written in Sencha Touch,  that lines up all possibilities of the framework combined with the corresponding code.

% \paragraph{Marktadoptatie}
% Volgens de Sencha website is 50\% van de Fortune 100 - een lijst van de grootste Amerikaanse bedrijven gerangschikt op jaaromzet - een Sencha klant~\cite{Inc.}.  Enkele van hun grootste klanten zijn CNN,  Samsung,  Cisco en  Visa.

\paragraph{Licenses}
Sencha Touch is free within a commercial context in which the developer does not share the code with its users.  There is also the option to use an open source version.  This comes with a GNU GPL v3 license which implies a free code redistribution as most important property.
More detailed licenses can be found at~\cite{SenchaInc.}.
  
% Voor de ontwikkeling van eigen raamwerken of SDKs betaal je een \term{original equipment manufacturer} (OEM) licentie.  Dit wil zeggen dat bedrijven hun producten gaan verkopen onder hun eigen merk en naam, maar gebruik maken van Sencha.  Omdat het gebruik hiervan per gebruiker verschilt,  worden OEM licenties op maat gemaakt~\cite{Inc.}.

\paragraph{Code and development}
Sencha Touch is written on top of Ext JS,  and can also be considered as JavaScript framework.  All code needs to be written in JavaScript and loaded by one HTML container.  An other important aspect of Sencha Touch is that is supports the Model-View-Controller (MVC) pattern.  Models group fields to data-objects,  views define how the content is presented to the user and controllers connect these based on events.
 
% In theorie zou het verschil tussen mobiele websites en applicaties enkel in de views terug te vinden zijn.  Echter,  dit wordt nog niet volledig ondersteund en raadt men dus aan om hiervoor aparte projecten te voorzien.

% Om het de ontwikkelaars makkelijker te maken biedt Sencha ook SDK tools aan.  Momenteel bevinden deze zich nog in bèta.  Concreet zijn deze tools commando's voor de terminal die onder andere nieuwe projecten kunnen aanmaken, JavaScript bestanden kunnen optimaliseren maar vooral de webapplicatie kunnen omzetten naar native applicaties voor iOS en Android.

% \paragraph{Debugging}
% Het debuggen van je code gebeurt voornamelijk in de browser zelf.  Tools als de Safari Web Inspector,  Chrome Developer Tools of Firebug moeten de fouten kunnen opsporen.  De broncode van Sencha Touch kan ook ingeladen worden met \code{sencha-touch-debug.js} als bibliotheek.  Deze versie is niet gecomprimeerd en bevat commentaar en documentatie om makkelijker te zoeken waar in de code de fout zich juist bevond.

%\subsubsection{Functionele kenmerken} %TODO hernoemen..
Sencha Touch contains all UI-elements as JavaScript objects.  Just like object-oriented programming,  those objects are part of a class system.  Classes can both be defined (\code{Ext.define}) or created (\code{Ext.create}).  Single-inheritance and overriding is also possible.    

To enhance performance,  it is the programmers task to create components before they are used.  In this manner,  programmers can mimic asynchronous loading of pages by creating them in advance.

% De basisklasse van alle objecten is \code{Ext.Component}.  Componenten kunnen gerenderd worden, zichzelf tonen of verbergen,  centreren op het scherm en zichzelf aan- of uitzetten.   Het aanmaken van componenten kan compacter door het gewenste component als \code{xtype} te definiëren.  
% 
% Een andere belangrijke component is \code{Ext.Container}.  Containers kunnen subcomponenten bevatten en een lay-out specifiëren.  Alle componenten krijgen een naam die verwijst naar een namespace.  Dit is handig om conflicten te vermijden tussen je eigen objecten en de standaard objecten van het raamwerk.  
% 
% Voor een opsomming van alle raamwerk compontenten verwijzen we naar de documentatie~\cite{Inc.2013a}.

% \paragraph{Model}
% Data kan intern worden voorgesteld met models.  Dit is iets wat hoort bij het MVC patroon.  Een model specifieert een lijst van velden die bij het model horen waarbij een veld een naam en een type heeft.  Optioneel kunnen validaties bij de velden worden toegevoegd om data consistent te houden.  
% 
% \paragraph{Store}
% \code{Ext.data.Store} is de klasse om instanties van een model op te slaan.  Een \term{store} wordt voorzien van een \term{proxy}.  Deze kan data aan de client of server zijde opslaan.  Een \term{proxy} voor opslag aan client zijde kan zowel in het RAM geheugen als in de \term{local storage} van de browser opslaan.  Een \term{proxy} voor server opslag kan data verzenden via AJAX (zelfde domein) of JSONP (verschillende domeinen).  Een \term{proxy} kan ook nog voorzien worden van een \term{reader} die aangeeft hoe de ontvangen data gelezen moet worden.
% 
% \paragraph{View}
% Een \term{view} is de benaming voor objecten die aan de gebruiker kunnen getoond worden.  Een voorbeeld hiervan zijn lijsten,  waar vaak de data van een \term{store} wordt in weergegeven.  Zo'n lijst kan makkelijk gefilterd of gesorteerd worden op basis van velden uit het model.  Hiervoor moeten we \term{filters} of \term{sorters} aan de \term{store} toevoegen.  De lay-out van één lijstitem bepalen kan via een \code{XTemplate}.  Het sjabloon bepaalt de HTML structuur van elk item.  Alle gedefinieerde velden van het model kunnen in de template worden opgeroepen of gemanipuleerd.

%TODO controller?

%\subsubsection{Niet-functionele kenmerken}
% \paragraph{Performantie}
% In vergelijking met versie 1.1 van Sencha Touch is de performantie gestegen om wille van verschillende factoren.  De introductie van het klasse systeem,  zoals besproken in de vorige sectie,  laat toe objecten dynamisch te laden.  Het grote verschil tussen \code{Ext.define} en \code{Ext.create} is dat objecten enkel in het geheugen worden geladen na creatie.  Het is dus de taak van de programmeur om objecten enkel te construeren wanneer ze nodig zijn.
% 
% Verder kwam versie 2.0 met een nieuwe lay-out \term{engine} die vooral het verwisselen van oriëntatie van het toestel versnelde.  Ook een verbetering in performantie op Android toestellen,  voornamelijk bij scrollen en animaties,  werd ingevoerd~\cite{Inc.}.
% 
% Een benchmark voor deze verbeteringen zijn de opstarttijden van de Kitchen Sink applicatie.  Het opstarten gebeurde met de verschillende Sencha Touch versies en op verschillende toestellen.  De resultaten zijn terug te vinden op figuur \ref{fig:sencha_performance}.  Op bijna elk toestel blijkt Sencha Touch 2.0 ongeveer één seconde sneller te werken~\cite{SenchaInc.2013}.

\paragraph{Browser support}
Just like jQuery Touch,  Sencha Touch is based upon the WebKit browser engine.  This forms the major requirement for browser support.  Although most mobile browsers contain this engine,  some like FireFox Mobile and Opera Mobile lack behind~\cite{JohnEClark2012}.  Following~\cite{Wokke2013}, the next release of the Opera browser will contain this engine,  a trend that most browser vendors will (have to) follow.

Sencha Touch offers the programmer methods to ask for the current context where the end-user is working in.  Properties like \code{Ext.env.Browser} and \code{Ext.env.OS} or methods like \code{Ext.Viewport.getOrientation} and \code{Ext.feature.has} can determine this context~\cite{JohnEClark2012}.  The latter has functionalities,  just like Modernizer~\cite{Modernizr2012}.  

\subsection{Table}
\label{sec:table} 

\section{Comparison} %vergelijkingscriteria (2 blz)
\label{sec:comparison}

\subsection{Explanation} %verantwoorden/inleiding criteria
\label{sec:explanation}

\subsection{Community} % (0,5 blz)
\label{sec:community}

% \paragraph{Community}
% Met 7.400 volgers op GitHub~\cite{GitHub2012} en 11.200 volgers op Twitter~\cite{Twitter2012} komt de grote kracht van jQuery Mobile van zijn community . Dit heeft grotendeels te maken met het feit dat jQuery Mobile geniet van het succes van jQuery, dat ook zeer populair is~\cite{Hales2012}.
%TODO Sander: dit komt in de vergelijkingscriteria

\subsection{Proof of concept} %puntjes logboek + POC uitleggen (1,5 blz)
\label{sec:poc}

\section{Future work} %(0,5 blz)
\label{sec:future_work}

\section{Conclusion} %(0,5 blz)
\label{sec:conclusion}

%% The file named.bst is a bibliography style file for BibTeX 0.99c
\bibliographystyle{named}
\bibliography{referenties}

\end{document}

