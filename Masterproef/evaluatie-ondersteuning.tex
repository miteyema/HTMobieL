\section{Ondersteuning}
\label{sec:evaluatie-ondersteuning}
De ondersteuning van de vier raamwerken wordt samengevat per uitdaging in tabel~\ref{tabel:evaluatie-ondersteuning}. 
Daarna zal iedere uitdaging per raamwerk uitvoerig worden besproken.

\begin{table}[H]
\centering
\pgfplotstabletypeset[
  begin table=\begin{tabular}{p{8cm} p{1cm} p{1cm} p{1cm} p{1cm}},
  end table=\end{tabular},
  skip coltypes=true,
  col sep=comma,
  string type,
  header=true,
  columns={Uitdaging,jQM,ST,Kendo,Lungo},
  columns/Uitdaging/.style={column name=\textbf{Uitdaging}, column type={l}},  
  columns/jQM/.style={column name=\textbf{\jqma}, column type={c}},
  columns/ST/.style={column name=\textbf{\sta}, column type={c}},
  columns/Kendo/.style={column name=\textbf{\kendoa}, column type={c}},
  columns/Lungo/.style={column name=\textbf{\lungoa}, column type={c}},
  every head row/.style={
    before row=\toprule,
    after row=\midrule},
  every last row/.style={
  	before row=\toprule,
 	after row=\bottomrule}
]{tabellen/ondersteuning-gonzalo.csv}
\caption{Samenvattende tabel voor ondersteuningscriterium}
\label{tabel:evaluatie-ondersteuning}
\end{table}

%%%%%%%%%%%%%
\subsection{U1: Formulieren}
\label{sec:evaluatie-ondersteuning-u1}

In tabel \ref{tabel:evaluatie-ondersteuning-u1} worden de resultaten getoond van de vijf deeluitdagingen van U1:~Formulieren.
Onder de tabel wordt per raamwerk verklaard waarom dat resultaat werd behaald.

\begin{table}[H]
\centering
\pgfplotstabletypeset[
  begin table=\begin{tabular}{p{8cm} p{1cm} p{1cm} p{1cm} p{1cm}},
  end table=\end{tabular},
  skip coltypes=true,
  col sep=comma,
  string type,
  header=true,
  skip coltypes=true,
  columns={Uitdaging,jQM,ST,Kendo,Lungo},
  columns/Uitdaging/.style={column name=\textbf{Uitdaging}, column type={l}},  
  columns/jQM/.style={column name=\textbf{\jqma}, column type={c}},
  columns/ST/.style={column name=\textbf{\sta}, column type={c}},
  columns/Lungo/.style={column name=\textbf{\lungoa}, column type={c}},
  columns/Kendo/.style={column name=\textbf{\kendoa}, column type={c}},
  every head row/.style={
    before row=\toprule,
    after row=\midrule},
  every last row/.style={
  	before row=\midrule,
    after row=\bottomrule}
]{tabellen/ondersteuning/u1.csv}
\caption{Ondersteuning voor U1: Formulieren}
\label{tabel:evaluatie-ondersteuning-u1}
\end{table}

\paragraph{\jqm}
\paragraph{\st}
\paragraph{\kendo}
\paragraph{\lungo}

%%%%%%%%%%%%%
\subsection{U3: Formuliervalidatie}
\label{sec:evaluatie-ondersteuning-u3}

\paragraph{\jqm}
\paragraph{\st}
\paragraph{\kendo}
\paragraph{\lungo}

%%%%%%%%%%%%%
\subsection{U4: Handtekening}
\label{sec:evaluatie-ondersteuning-u4}

\paragraph{\jqm}
\paragraph{\st}
\paragraph{\kendo}
\paragraph{\lungo}

%%%%%%%%%%%%%
\subsection{U5: Toon PDF}
\label{sec:evaluatie-ondersteuning-u5}

\paragraph{\jqm}
\paragraph{\st}
\paragraph{\kendo}
\paragraph{\lungo}

%%%%%%%%%%%%%
\subsection{U6: Toevoegen van afbeelding}
\label{sec:evaluatie-ondersteuning-u6}

\paragraph{\jqm}
\paragraph{\st}
\paragraph{\kendo}
\paragraph{\lungo}

%%%%%%%%%%%%%
\subsection{U7: Auto-aanvullen}
\label{sec:evaluatie-ondersteuning-u7}

\paragraph{\jqm}
\paragraph{\st}
\paragraph{\kendo}
\paragraph{\lungo}

%%%%%%%%%%%%%
\subsection{U9: Toestelspecifieke lay-out}
\label{sec:evaluatie-ondersteuning-u9}

\paragraph{\jqm}
\paragraph{\st}
\paragraph{\kendo}
\paragraph{\lungo}

%%%%%%%%%%%%%
\subsection{U10: Offline}
\label{sec:evaluatie-ondersteuning-u10}

In tabel \ref{tabel:evaluatie-ondersteuning-u10} worden de resultaten getoond van de twee deeluitdagingen van U10:~Offline.
Onder de tabel wordt per raamwerk verklaard waarom dat resultaat werd behaald.

\begin{table}[H]
\centering
\pgfplotstabletypeset[
  begin table=\begin{tabular}{p{8cm} p{1cm} p{1cm} p{1cm} p{1cm}},
  end table=\end{tabular},
  skip coltypes=true,
  col sep=comma,
  string type,
  header=true,
  skip coltypes=true,
  columns={Uitdaging,jQM,ST,Kendo,Lungo},
  columns/Uitdaging/.style={column name=\textbf{Uitdaging}, column type={l}},  
  columns/jQM/.style={column name=\textbf{\jqma}, column type={c}},
  columns/ST/.style={column name=\textbf{\sta}, column type={c}},
  columns/Lungo/.style={column name=\textbf{\lungoa}, column type={c}},
  columns/Kendo/.style={column name=\textbf{\kendoa}, column type={c}},
  every head row/.style={
    before row=\toprule,
    after row=\midrule},
  every last row/.style={
  	before row=\midrule,
    after row=\bottomrule}
]{tabellen/ondersteuning/u10.csv}
\caption{Ondersteuning voor U10: Offline}
\label{tabel:evaluatie-ondersteuning-u10}
\end{table}

\paragraph{\jqm}
\paragraph{\st}
\paragraph{\kendo}
\paragraph{\lungo}