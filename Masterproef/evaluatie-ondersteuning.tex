\section{Ondersteuning}
\label{sec:evaluatie-ondersteuning}

In deze sectie zal de ondersteuning van de raamwerken op mobiele apparaten worden onderzocht.
In de secties \ref{sec:evaluatie-ondersteuning-toestel} tot \ref{sec:evaluatie-ondersteuning-offline} zal iedere uitdaging per raamwerk uitvoerig worden besproken.
Voor de score van ondersteuning wordt naar formule \ref{eq:ondersteuning} verwezen waar de som van de ondersteuning op alle acht apparaten wordt gemaakt.
De scores van de vier raamwerken wordt samengevat per uitdaging in tabel~\ref{tabel:evaluatie-ondersteuning-u}.
Een ander zicht op diezelfde data wordt bekomen door de ondersteuning van de vier raamwerken samen te vatten per apparaat.
Deze weergave wordt getoond in tabel~\ref{tabel:evaluatie-ondersteuning-a}.
Voor een gedetailleerd overzicht van de ondersteuning per apparaat wordt verwezen naar appendix~\ref{app:ondersteuning}.

%\begin{landscape}
\begin{table}
\centering
\resizebox{14cm}{!} {
\pgfplotstabletypeset[
	column type=l,
	every head row/.style={
		before row={%
			\toprule
			\textbf{Uitdaging}
			& \multicolumn{2}{c}{\textbf{\sta}}
			& \multicolumn{2}{c}{\textbf{\kendoa}} 
			& \multicolumn{2}{c}{\textbf{\jqma}}
			& \multicolumn{2}{c}{\textbf{\lungoa}} \\
			\cmidrule(r){2-3}
			\cmidrule(r){4-5}
			\cmidrule(r){6-7}
			\cmidrule(r){8-9}
		},
		after row=\midrule,
		},
  	every last row/.style={
  		before row=\toprule,
 		after row=\bottomrule},
	columns={Uitdaging,ST(abs),ST(max),Kendo(abs),Kendo(max),jQM(abs),jQM(max),Lungo(abs),Lungo(max)},
	begin table=\begin{tabular}{p{7.5cm}cccccccc},
	end table=\end{tabular},
	header=true,
	skip coltypes=true,
	columns/Uitdaging/.style ={column name=},
	columns/ST(abs)/.style ={column name=Score},
	columns/ST(max)/.style={column name=Max},
	columns/Kendo(abs)/.style ={column name=Score},
	columns/Kendo(max)/.style={column name=Max},
	columns/jQM(abs)/.style ={column name=Score},
	columns/jQM(max)/.style={column name=Max},
	columns/Lungo(abs)/.style ={column name=Score},
	columns/Lungo(max)/.style={column name=Max},
	col sep=comma,
	string type,
]{tabellen/ondersteuning-u.csv}
}
\caption{Ondersteuning per uitdaging.}
\label{tabel:evaluatie-ondersteuning-u}
\end{table}
%\end{landscape}

\begin{table}
\centering
\resizebox{14cm}{!} {
\pgfplotstabletypeset[
	column type=l,
	every head row/.style={
		before row={%
			\toprule
			\textbf{Apparaat}
			& \multicolumn{2}{c}{\textbf{\sta}}
			& \multicolumn{2}{c}{\textbf{\kendoa}} 
			& \multicolumn{2}{c}{\textbf{\jqma}}
			& \multicolumn{2}{c}{\textbf{\lungoa}} \\
			\cmidrule(r){2-3}
			\cmidrule(r){4-5}
			\cmidrule(r){6-7}
			\cmidrule(r){8-9}
		},
		after row=\midrule,
		},
  	every last row/.style={
  		before row=\toprule,
 		after row=\bottomrule},
columns={Apparaat,ST(abs),ST(max),Kendo(abs),Kendo(max),jQM(abs),jQM(max),Lungo(abs),Lungo(max)},
	begin table=\begin{tabular}{p{7.5cm}cccccccc},
	end table=\end{tabular},
	header=true,
	skip coltypes=true,
	columns/Apparaat/.style ={column name=},
	columns/ST(abs)/.style ={column name=Score},
	columns/ST(max)/.style={column name=Max},
	columns/Kendo(abs)/.style ={column name=Score},
	columns/Kendo(max)/.style={column name=Max},
	columns/jQM(abs)/.style ={column name=Score},
	columns/jQM(max)/.style={column name=Max},
	columns/Lungo(abs)/.style ={column name=Score},
	columns/Lungo(max)/.style={column name=Max},
	col sep=comma,
	string type,
]{tabellen/ondersteuning-a.csv}
}
\caption{Ondersteuning per apparaat.}
\label{tabel:evaluatie-ondersteuning-a}
\end{table}

Zoals besproken in sectie \ref{sec:vergelijking-ondersteuning} wordt enkel het raamwerk en niet een eigen implementatie op ondersteuning getest.
Daarom verschilt de maximale score voor ondersteuning per raamwerk.
Welke uitdagingen door het raamwerk konden worden geïmplementeerd werd in het vorige criterium geduid.
De maximale scores voor \st{},  \kendo{},  \jqm{} en \lungo{} zijn respectievelijk $96$, $104$, $104$ en $80$.
De score voor ondersteuning is respectievelijk $83$, $92$, $95$ en $62$.
Wanneer de relatieve score per raamwerk bekeken wordt, heeft \st{} een score van $86\%$,  \kendo{} $88\%$,  \jqm{} $91\%$ en \lungo{} $78\%$.
Hoewel de verschillen klein zijn kan gezegd worden dat \jqm{} de beste ondersteuning biedt,  gevolgd door een \kendo{} en \st{}.
Er kan geconcludeerd worden dat de raamwerken buiten \lungo{} behoorlijk hoog scoren.
Dit wil zeggen dat de keuze voor één van de drie raamwerken onafhankelijk is van de ondersteuning.
De verschillende raamwerken ondersteunen dezelfde apparaten.
Dit komt doordat de raamwerken allemaal steunen op dezelfde onderliggende technologie, namelijk HTLM5.

Alle raamwerken buiten \lungo{} ondersteunen gemiddeld $88\%$ van de geïmplementeerde uitdagingen.
\lungo{} ondersteunt slechts $78\%$ van de geïmplementeerde uitdagingen.
De grootste problemen werden bij formulieren waargenomen.
Ook kon er niet genavigeerd worden doorheen de volledige POC op de \htc{} en \gtab{}. 
Dit komt doordat de \term{tap} gebeurtenis werd gebruikt om te navigeren.
Op de \htc{} en \gtab{} werden deze echter niet afgevuurd.
Op de andere toestellen had dit een dubbelklikeffect.
Hierdoor gebeurde de eerste klik op het huidig scherm en de tweede klik op het volgende.
Zo kan het bijvoorbeeld gebeuren dat een menu op het volgende scherm al openklapt, zonder dat de gebruiker dit verwachtte.

Android 2.3 toestellen - \htc{} en \gtab{} - hebben het minste ondersteuning.
Android 4 en iOS toestellen scoren gemiddeld $95\%$.
De gemiddelde score bij alle Android-toestellen is $79\%$ in tegenstelling tot $95\%$ ondersteuning op iOS-toestellen.
Er moet opgemerkt worden dat enkel versie $5$ en $6$ van iOS werd getest (zie tabel \ref{tabel:toestellen-hci}).
Van alle acht apparaten is de \iphoneiii{} het oudst (officiële voorstelling in juni 2009) gevolgd door de \gs{} (officiële voorstelling in maart 2010)~\cite{Staff2009,Gideon2010}.
Hoewel beide apparaten opgewaardeerd zijn naar een recenter besturingssysteem, is de trend dat Android-toestellen dit minder vaak aanbieden ten opzichte van iOS-toestellen.
De reden hiervoor is de grotere heterogeniteit tussen Android-toestellen.


Een opmerkelijk resultaat is de maximale ondersteuning van formuliervalidatie en handtekening voor alle raamwerken behalve \lungo{}.
Als laatste ondersteunen alle raamwerken offline opslag op alle beschikbare apparaten.


%%%%%%%%%%%%%
\subsection{\uit{toestel}}
\label{sec:evaluatie-ondersteuning-toestel}

\st{} en \kendo{} voorzien methoden om de context waarin de applicatie wordt uitgevoerd,  op te vragen.
De implementatie van \jqm{} en \lungo{} steunt op CSS3.
Het soort apparaat (smartphone of tablet) werd op alle toestellen behalve de \gtab{} correct herkend.
Door de lage resolutie werd deze als smartphone gecategoriseerd op \st{},  \jqm{} en \lungo{}.
Bij \jqm{} en \lungo{} geldt dit niet wanneer de applicatie liggend wordt opgestart.
Daarentegen herkende \kendo{} de \gtab{} als tablet zowel in staande als liggende modus.


% \paragraph{\st} \score{7}{8}
% Alle toestellen buiten de \gtab{} werden door het \st{} raamwerk herkend.
% Het genereren van de toestelspecifieke lay-out werd op elk toestel wel ondersteund.
% 
% \paragraph{\kendo} \score{8}{8}
% Net zoals bij \jqm{} werd de \gtab{} enkel als tablet herkend als de applicatie liggend werd geladen.
% De toestelspecifieke lay-out werd correct weergegeven.
% Ook werd het besturingssysteem herkend en werd de \term{native look-and-feel} nagebootst.
% 
% \paragraph{\jqm} \score{7}{8}
% Zeven van de acht toestellen werden met succes herkend als smartphone of tablet.
% Echter door de lage resolutie van de \gtab{}, werd staand de tablet als smartphone herkend.
% Indien het gekozen breekpunt van de CSS3 Media Query wordt geoptimaliseerd voor dit toestel, kan volledige ondersteuning voor \gtab{} worden bekomen.
% 
% \paragraph{\lungo} \score{7}{8}
% Aangezien dezelfde aanpak werd gebruik als \jqm{} is de ondersteuning exact hetzelfde.

%%%%%%%%%%%%%
\subsection{\uit{formulieren}}
\label{sec:evaluatie-ondersteuning-formulieren}

In tabel \ref{tabel:evaluatie-ondersteuning-formulieren} worden de resultaten getoond van de vijf deeluitdagingen.
Onder de tabel wordt per raamwerk verklaard waarom dat resultaat werd behaald.
De tweede deeluitdaging faalt bij ieder raamwerk op de \htc{} en \gtab{} aangezien geen e-mailtoetsenbord getoond wordt.

\begin{table}
\centering
\resizebox{14cm}{!} {
\pgfplotstabletypeset[
	column type=l,
	every head row/.style={
		before row={%
			\toprule
			\textbf{Uitdaging}
			& \multicolumn{2}{c}{\textbf{\sta}}
			& \multicolumn{2}{c}{\textbf{\kendoa}} 
			& \multicolumn{2}{c}{\textbf{\jqma}}
			& \multicolumn{2}{c}{\textbf{\lungoa}} \\
			\cmidrule(r){2-3}
			\cmidrule(r){4-5}
			\cmidrule(r){6-7}
			\cmidrule(r){8-9}
		},
		after row=\midrule,
		},
  	every last row/.style={
  		before row=\toprule,
 		after row=\bottomrule},
	columns={Uitdaging,ST(abs),ST(max),Kendo(abs),Kendo(max),jQM(abs),jQM(max),Lungo(abs),Lungo(max)},
	begin table=\begin{tabular}{lcccccccc},
	end table=\end{tabular},
	header=true,
	skip coltypes=true,
	columns/Uitdaging/.style ={column name=},
	columns/ST(abs)/.style ={column name=Score},
	columns/ST(max)/.style={column name=Max},
	columns/Kendo(abs)/.style ={column name=Score},
	columns/Kendo(max)/.style={column name=Max},
	columns/jQM(abs)/.style ={column name=Score},
	columns/jQM(max)/.style={column name=Max},
	columns/Lungo(abs)/.style ={column name=Score},
	columns/Lungo(max)/.style={column name=Max},
	col sep=comma,
	string type,
]{tabellen/ondersteuning/formulieren.csv}
}
\caption{Ondersteuning van \uit{formulieren}.}
\label{tabel:evaluatie-ondersteuning-formulieren}
\end{table}

\paragraph{\st}
Alle smartphones en de \gtab{} vertoonden een probleem bij de labels van de \code{option}-knoppen.
Deze zijn nodig om het type van uitgave te selecteren bij het aanmaken van een nieuwe uitgave.
Alle labels buiten \term{Hotel} en \term{Restaurant} zijn te lang om in de voorziene ruimte weer te geven.
\st{} kort deze af met ellipsen waardoor niet de volledige naam leesbaar is.
Dit kan tot verwarring leiden.
In liggende mode worden sommige labels wel zichtbaar.
De \term{datepicker} en schakelaar die \st{} voorziet, werkten op elk toestel.


\paragraph{\kendo}
Alle Android-toestellen toonden een wit kader bij \term{read-only} invoervelden waarbij de waarde uit een \term{widgets} kwam.
Deze zijn te zien wanneer het overzicht van een ingevoerde uitgave wordt opgevraagd.
Het raamwerk moet de waarden van de \term{widgets} in de \term{read-only} invoerelementen invullen.
Deze functionaliteit werkte wel op iOS-toestellen.
Het openen van \term{datepicker} bij \kendo{} vindt plaats als de gebruiker op het icoontje in het invoerveld klikt.
Indien op de datum wordt geklikt zal de gebruiker de datum tekstueel aanpassen.  
Hiervoor moet het correcte formaat worden gebruikt.
Ook werd op alle toestellen een virtueel toetsenbord weergegeven ondanks de invoer uit de \term{datepicker} moet worden gekozen.
De schakelaar en optievelden werkten op elk apparaat.

\paragraph{\jqm}
Op iedere deeluitdaging behaalt \jqm{} de maximumscore.
Het scrollen in de \term{datepicker} werkte echter niet vloeiend op de \htc{} en \gtab{}, maar de functionaliteit werkte wel.
Op de \ipadi{} werd opgemerkt dat door de lokale instellingen, een scheidingsteken voor de duizendtallen werd gebruikt voor het getal in het nummerveld.
%Als laatste bleken er verschillende implementaties van de \js{}-functie \code{new Date(dateString)} doorheen de apparaten.
Als \code{dateString} werd het patroon \code{jaar/maand/dag} uiteindelijk gebruikt, want het patroon \code{jaar-maand-dag} gaf op de \htc{} en \gtab{} een ongeldige datum.
De schakelaar en optievelden werkten op elk apparaat.

\paragraph{\lungo}
De slechtste ondersteuning was op de \htc{} en \gtab{}.
Met deze toestellen kon zeer moeilijk op de velden worden geklikt.
Soms lukte dit, soms lukte dit niet.
Dit wordt niet in de punten gereflecteerd.
De aangepaste \term{datepicker} werkte op ieder toestel, wat in schril contrast staat met de schakelaar die op geen enkel toestel correct werkte.
Deze laatste deed verschillende acties naargelang geklikt of ge\term{swipe}d werd op de schakelaar.

%%%%%%%%%%%%%
\subsection{\uit{autoaanvullen}}
\label{sec:evaluatie-ondersteuning-autoaanvullen}

\st{} kon deze uitdaging niet implementeren en dit werd bijgevolg niet op ondersteuning getest.
\kendo{} en \jqm{} behaalde beide de maximum score.
Op de \htc{} werden bij \lungo{} de suggesties dubbel op het scherm aan de gebruiker getoond.

% \paragraph{\st} 
% Deze uitdaging kon niet worden ondersteund omdat het raamwerk geen implementatie toeliet
% 
% \paragraph{\kendo} \score{8}{8}
% Deze uitdaging behaalde op elk toestel de maximum score.
% 
% \paragraph{\jqm} \score{8}{8}
% Ieder toestel ondersteunde de auto-aanvulling.
% 
% \paragraph{\lungo} \score{7}{8}
% Zeven van de acht toestellen ondersteunden auto-aanvulling.
% De \htc{} toonde echter de suggesties dubbel op het scherm aan de gebruiker.

%%%%%%%%%%%%%
\subsection{\uit{afbeelding}}
\label{sec:evaluatie-ondersteuning-afbeelding}

De scores zijn gelijk voor alle toestellen.
\kendo{}, \jqm{} en \lungo{} gebruiken dezelfde aanpak om deze uitdaging te implementeren.
Deze aanpak steunt op de FileReaderAPI en wordt enkel vanaf Android~3 en iOS~6 ondersteund~\cite{Deveria2013c}.
Ook de plug-in die \st{} gebruikt voor deze uitdaging maakt gebruik van deze API.
Hierdoor zullen de \htc{},  \gtab{} en \ipadi{} deze uitdaging in de vier raamwerken niet ondersteunen.

Er dient opgemerkt te worden dat de hoge resolutie camera aan de achterkant van de \ipadiii{}, \iphoneiii{} en \iphoneiv{} niet kon worden gebruikt.
Dit is een gekende limiet van iOS waarbij apparaten met minder dan $256\unit{MB}$ RAM maximaal drie megapixels foto's kunnen importeren op het \code{canvas}.
Apparaten met $256\unit{MB}$ RAM of meer kunnen foto's tot vijf megapixels importeren op het \code{canvas}~\cite{Apple2012}.
Een gelijkaardig probleem gaat ook op voor Android-apparaten als de foto wordt opgeslagen in de lokale opslag.
De exacte limieten voor \code{localStorage} en \code{sessionStorage} zijn browser- en versiespecifiek~\cite{Gonzalez2012}.
Hierdoor moet de camera aan de voorkant worden gebruikt of een afbeelding met lagere resolutie worden gekozen.
De \ipadi{} biedt geen ondersteuning voor het formulierveld van het type \code{file} en zal het opladen van een afbeelding dus ook niet ondersteunen~\cite{ViljamiSalminen2012}.

% \paragraph{\st} \score{5}{8}
% Oudere Android types bieden geen ondersteuning voor de FileReaderAPI.
% Hierdoor was het opladen van een afbeelding niet mogelijk bij de \htc{} en \gtab{}.
% De \ipadi{} brengt geeft geen dialoogvenster naar boven nadat op de knop werd gedrukt omdat \code{file} niet als formulierveld wordt ondersteund.
% 
% Hierdoor moet de camera aan de voorkant worden gebruikt of een afbeelding met lagere resolutie worden gekozen.
% 
% \paragraph{\jqm} \score{5}{8}
% Op vier apparaten kon een afbeelding met succes worden toegevoegd.
% Daarentegen hadden de \htc{} en \gtab{} geen ondersteuning voor de FileReaderAPI, op de \gs{} bleef de applicatie hangen en als laatste bood de \ipadi{} geen ondersteuning voor het formulierveld van het type \code{file}.
% 
% \paragraph{\kendo} \score{5}{8}
% De FileReader API en \code{file} als type van het formulierveld waren de vereisten voor deze uitdaging.
% Hierdoor werd de implementatie voor \htc{},  \gtab{} en \ipadi{} niet ondersteund.
% 
% \paragraph{\lungo} \score{5}{8}
% Aangezien dezelfde aanpak werd gebruik als \jqm{} is de ondersteuning exact hetzelfde.

%%%%%%%%%%%%%
\subsection{\uit{validatie}}
\label{sec:evaluatie-ondersteuning-validatie}

HTML5-formuliervalidatie wordt op geen enkel beschikbaar mobiel besturingssysteem ondersteund~\cite{Deveria2013c}.
Elke implementatie van deze uitdaging werd door alle toestellen ondersteund.
Een specifiek mobiel probleem bij \jqm{} was dat bij het tonen van het dialoogvenster, de plug-in op de achtergrond de cursor op het eerste veld zette. 
Hierdoor verscheen het toetsenbord op het scherm van het mobiele apparaat wanneer het dialoogvenster tevoorschijn kwam, wat niet de bedoeling is. 
Dit werd opgelost door \code{focusInvalid:false} in te stellen.

% \paragraph{\st} \score{8}{8}
% Deze uitdaging werkte op ieder toestel en behaalde zo de maximumscore.
% 
% \paragraph{\kendo} \score{8}{8}
% Deze uitdaging werkte op ieder toestel en behaalde zo de maximumscore.
% 
% \paragraph{\jqm} \score{8}{8}
% Deze uitdaging werkte op ieder toestel en behaalde zo de maximumscore.
% 
% \paragraph{\lungo}
% Aangezien deze uitdaging handmatig werd geïmplementeerd, werd de ondersteuning niet getest.

%%%%%%%%%%%%%
\subsection{\uit{handtekening}}
\label{sec:evaluatie-ondersteuning-handtekening}
\kendo{} en \jqm{} hanteren dezelfde plug-in om deze uitdaging te implementeren.
Onderliggend steunt deze op het \code{canvas} kenmerk van HTML5.
De plug-in die \st{} gebruikt is ook afhankelijk van dit kenmerk.
Alle onderzochte toestellen ondersteunen de \code{canvas}~\cite{Deveria2013c} en de drie raamwerken behalen vervolgens de maximumscore. 
Omdat bij \lungo{} geen plug-in werd gevonden voor deze uitdaging, kon ook de ondersteuning niet worden onderzocht.

% \paragraph{\st} \score{8}{8}
% Deze uitdaging werkte op ieder toestel en behaalde zo de maximumscore.
% 
% \paragraph{\kendo} \score{8}{8}
% Aangezien dezelfde aanpak werd gebruik als \jqm{} is de ondersteuning exact hetzelfde.
% 
% \paragraph{\jqm} \score{8}{8}
% De handtekening kon op ieder apparaat worden getekend, behalve op de \gs{}.
% Daar werd tijdens het tekenen van de handtekening, de handtekening dubbel op het scherm getekend.
% 
% \paragraph{\lungo}
% 
% Doordat geen plug-in werd gevonden voor deze uitdaging, kon ook de ondersteuning niet worden onderzocht.


%%%%%%%%%%%%%
\subsection{\uit{pdf}}
\label{sec:evaluatie-ondersteuning-pdf}

\kendo{}, \jqm{} en \lungo{} hergebruiken dezelfde code voor deze uitdaging.
Drie van de acht apparaten hadden problemen met het tonen van de PDF.
Op de \htc{} werd een wit scherm getoond, op de \gtab{} lukt het niet en op de \gs{} gaf het besturingssysteem de boodschap 'Download unsuccessful' terug.
Bij de vier andere apparaten werd de PDF-lezer geopend en kon het bestand worden bekeken.
De score voor deze drie raamwerken is dus 5/8.

\st{} maakt gebruik van een plug-in voor het tonen van PDF-bestanden.
Deze plug-in maakt gebruik van PDF.JS,  een PDF-renderer van Mozilla~\cite{Gal2010}.
PDF.JS is geïmplementeerd met \js{} en HTML5.
Het \code{canvas} kenmerk en ondersteuning voor de SVG (Scalable Vector Graphics) API is noodzakelijk.
Slechts vanaf Android~3 wordt dit laatste ondersteund.
Alle gecontroleerde versies van iOS ondersteunen dit wel~\cite{Deveria2013c}.
Het tonen van een PDF werkt dus niet op de \htc{} en \gtab{}.
Een belangrijke opmerking is dat de ondersteuning van deze uitdaging werd getest op een implementatie die niet gebouwd was met Sencha Cmd.
Dit was omdat de gebouwde versie een foutboodschap gaf wanneer de plug-in werd aangesproken.
\st{} haalt een score van 6/8 omdat de \gs{} deze uitdaging wel ondersteunt in tegenstelling tot de drie andere raamwerken.

% \paragraph{\st} \score{6}{8}
% 
% 
% \paragraph{\kendo} \score{5}{8}
% Aangezien dezelfde aanpak werd gebruik als \jqm{} is de ondersteuning exact hetzelfde.
% 
% \paragraph{\jqm} \score{5}{8}
% Drie van de vier apparaten hadden problemen met het tonen van de PDF.
% Op de \htc{} werd een wit scherm getoond, op de \gtab{} lukt het niet en op de \gs{} gaf het besturingssysteem de boodschap 'Download unsuccessful' terug.
% Bij de vier andere apparaten werd de PDF-lezer geopend en kon de PDF worden bekeken.
% 
% \paragraph{\lungo} \score{5}{8}
% Aangezien dezelfde aanpak werd gebruik als \jqm{} is de ondersteuning exact hetzelfde.


%%%%%%%%%%%%%
\subsection{\uit{offline}}
\label{sec:evaluatie-ondersteuning-offline}

In tabel \ref{tabel:evaluatie-ondersteuning-offline} worden de resultaten getoond van de twee deeluitdagingen.
Offline en opslag is één element van de acht technologieklassen van HTLM5 (zie sectie \ref{sec:html5-css3-js}).
Om de POC offline beschikbaar te maken is zowel ondersteuning voor \code{manifest} en \code{localStorage} noodzakelijk.
Beide kenmerken worden door alle beschikbare mobiele besturingssystemen ondersteund~\cite{Deveria2013c}. 



\begin{table}
\centering
\resizebox{14cm}{!} {
\pgfplotstabletypeset[
	column type=l,
	every head row/.style={
		before row={%
			\toprule
			\textbf{Uitdaging}
			& \multicolumn{2}{c}{\textbf{\sta}}
			& \multicolumn{2}{c}{\textbf{\kendoa}} 
			& \multicolumn{2}{c}{\textbf{\jqma}}
			& \multicolumn{2}{c}{\textbf{\lungoa}} \\
			\cmidrule(r){2-3}
			\cmidrule(r){4-5}
			\cmidrule(r){6-7}
			\cmidrule(r){8-9}
		},
		after row=\midrule,
		},
  	every last row/.style={
  		before row=\toprule,
 		after row=\bottomrule},
	columns={Uitdaging,ST(abs),ST(max),Kendo(abs),Kendo(max),jQM(abs),jQM(max),Lungo(abs),Lungo(max)},
	begin table=\begin{tabular}{lcccccccc},
	end table=\end{tabular},
	header=true,
	skip coltypes=true,
	columns/Uitdaging/.style ={column name=},
	columns/ST(abs)/.style ={column name=Score},
	columns/ST(max)/.style={column name=Max},
	columns/Kendo(abs)/.style ={column name=Score},
	columns/Kendo(max)/.style={column name=Max},
	columns/jQM(abs)/.style ={column name=Score},
	columns/jQM(max)/.style={column name=Max},
	columns/Lungo(abs)/.style ={column name=Score},
	columns/Lungo(max)/.style={column name=Max},
	col sep=comma,
	string type,
]{tabellen/ondersteuning/offline.csv}
}
\caption{Ondersteuning van \uit{offline}.}
\label{tabel:evaluatie-ondersteuning-offline}
\end{table}

% 
% \paragraph{\st} \score{8}{8}
% Deze uitdaging behaalde bij alle toestellen de maximum score.
% 
% \paragraph{\kendo} \score{8}{8}
% Deze uitdaging behaalde bij alle toestellen de maximum score.
% 
% \paragraph{\jqm} \score{8}{8}
% Alle acht apparaten ondersteunen zowel \code{localStorage} als Application Cache, wat beide HTML5-specificaties zijn.
% 
% \paragraph{\lungo} \score{8}{8}
% Aangezien dezelfde aanpak werd gebruik als \jqm{} is de ondersteuning exact hetzelfde.