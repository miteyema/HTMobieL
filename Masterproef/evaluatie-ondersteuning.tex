\section{Ondersteuning}
\label{sec:evaluatie-ondersteuning}
De ondersteuning van de vier raamwerken wordt samengevat per uitdaging in tabel~\ref{tabel:evaluatie-ondersteuning-u}. 
Voor een gedetailleerd overzicht wordt verwezen naar appendix~\ref{app:ondersteuning}.
In de secties \ref{sec:evaluatie-ondersteuning-toestel} tot \ref{sec:evaluatie-ondersteuning-offline} zal iedere uitdaging per raamwerk uitvoerig worden besproken.

\begin{table}[H]
\centering
\pgfplotstabletypeset[
  begin table=\begin{tabular}{p{8cm} p{1cm} p{1cm} p{1cm} p{1cm}},
  end table=\end{tabular},
  skip coltypes=true,
  col sep=comma,
  string type,
  header=true,
  columns={Uitdaging,ST(rel),Kendo(rel),jQM(rel),Lungo(rel)},
  columns/Uitdaging/.style={column name=\textbf{Uitdaging}, column type={l}},  
  columns/jQM(rel)/.style={column name=\textbf{\jqma}, column type={c}},
  columns/ST(rel)/.style={column name=\textbf{\sta}, column type={c}},
  columns/Lungo(rel)/.style={column name=\textbf{\lungoa}, column type={c}},
  columns/Kendo(rel)/.style={column name=\textbf{\kendoa}, column type={c}},
  every head row/.style={
    before row=\toprule,
    after row=\midrule},
  every last row/.style={
  	before row=\toprule,
 	after row=\bottomrule}
]{tabellen/ondersteuning-u.csv}
\caption{Samenvattende tabel voor ondersteuningscriterium per uitdaging.}
\label{tabel:evaluatie-ondersteuning-u}
\end{table}

%TODO
% De algemene score voor iedere raamwerk is rond de 90% of meer, dus eigenlijk moet je niet kijken naar de ondersteuning wanneer je een raamwerk kiest
% jQM is de duidelijke winnaar over alle devices, gevolgd door een gedeelde tweede plaats van Kendo en ST, daarna door Lungo.
% Op de HTC (android 2.3) haalt iedereen rond de 70, behalve lungo 55
% Galaxytab bieden jQM en Kendo 70% support, terwijl ST en lungo maar rond de 55 beiden
% Op de recenste devices scoren alle raamwerken ruim boven de 90%
% Iedereen support 100% de offline
% Support op iOS zit bij ieder raamwerk rond of boven de 90%, terwijl bij android dat bij ieder raamwerk maar boven de 70% ligt (een gemiddelde van 9 punten beter)
% bij iOS altijd updaten tot de laatste versie (behalve ipad 1), zie maximale versies iOS op http://en.wikipedia.org/wiki/List_of_iOS_devices
% Android: de hardware is zo divers, moeilijk te updaten, dus je blijft vast aan je 'oud' os en daardoor support HTML5 bla bal bal
% Formuliervalidatie worden volledig ondersteund door raamwerk of plugin, in tegenstelling tot HTML5 formuliervalidatie op mobiele apparaten.
% Toevoegen van afbeeldingen, maakt jQM/Kendo/Lungo expliciet gebruik van HTML5 en ST impliciet under the hood in the plugin
% local storage en manifest is ook HTML5, maar wordt wel overal ondersteund

Een ander zicht op diezelfde data wordt bekomen door de ondersteuning van de vier raamwerken samen te vatten per apparaat.
Deze weergave wordt getoond in tabel~\ref{tabel:evaluatie-ondersteuning-a}.

\begin{table}[H]
\centering
\pgfplotstabletypeset[
  begin table=\begin{tabular}{p{8cm} p{1cm} p{1cm} p{1cm} p{1cm}},
  end table=\end{tabular},
  skip coltypes=true,
  col sep=comma,
  string type,
  header=true,
  columns={Apparaat,ST(rel),Kendo(rel),jQM(rel),Lungo(rel)},
  columns/Apparaat/.style={column name=\textbf{Apparaat}, column type={l}},  
  columns/jQM(rel)/.style={column name=\textbf{\jqma}, column type={c}},
  columns/ST(rel)/.style={column name=\textbf{\sta}, column type={c}},
  columns/Lungo(rel)/.style={column name=\textbf{\lungoa}, column type={c}},
  columns/Kendo(rel)/.style={column name=\textbf{\kendoa}, column type={c}},
  every head row/.style={
    before row=\toprule,
    after row=\midrule},
  every last row/.style={
  	before row=\toprule,
 	after row=\bottomrule}
]{tabellen/ondersteuning-a.csv}
\caption{Samenvattende tabel voor ondersteuningscriterium per apparaat.}
\label{tabel:evaluatie-ondersteuning-a}
\end{table}

%%%%%%%%%%%%%
\subsection{\uit{toestel}}
\label{sec:evaluatie-ondersteuning-toestel}

Aangezien deze uitdaging voor ondersteuning geen deeluitdagingen bevat, wordt er geen tabel getoond.
Hieronder wordt per raamwerk verklaard waarom dat resultaat werd behaald.

%TODO Tim: breakpoint CSS3 media query gekozen volgens website 
\paragraph{\jqm}
Zeven van de acht toestellen werden met succes herkend als smartphone of tablet.
Echter door de lage resolutie van de \gtab{}, werd staand de tablet als smartphone herkend.
Als de tablet liggend werd geplaatst, dan werd deze pas als tablet herkend.

\paragraph{\st}
Alle toestellen buiten de \gtab{} werden door het \st{} raamwerk herkend.
Het genereren van de toestelspecifieke lay-out werd op elk toestel wel ondersteund.

\paragraph{\kendo}
Net zoals bij \jqm{} werd de \gtab{} enkel als tablet herkend als de applicatie liggend werd geladen.
De toestelspecifieke lay-out werd correct weergegeven.
Ook werd het besturingssysteem herkend en werd de \term{native look-and-feel} nagebootst.

\paragraph{\lungo}
Aangezien dezelfde aanpak werd gebruik als \jqm{} is de ondersteuning exact hetzelfde.

%%%%%%%%%%%%%
\subsection{\uit{formulieren}}
\label{sec:evaluatie-ondersteuning-formulieren}

In tabel \ref{tabel:evaluatie-ondersteuning-formulieren} worden de resultaten getoond van de vijf deeluitdagingen van \uit{formulieren}.
Onder de tabel wordt per raamwerk verklaard waarom dat resultaat werd behaald.

\begin{table}[H]
\centering
\pgfplotstabletypeset[
  begin table=\begin{tabular}{p{8cm} p{1cm} p{1cm} p{1cm} p{1cm}},
  end table=\end{tabular},
  skip coltypes=true,
  col sep=comma,
  string type,
  header=true,
  skip coltypes=true,
  columns={Uitdaging,ST(rel),Kendo(rel),jQM(rel),Lungo(rel)},
  columns/Uitdaging/.style={column name=\textbf{Uitdaging}, column type={l}},  
  columns/jQM(rel)/.style={column name=\textbf{\jqma}, column type={c}},
  columns/ST(rel)/.style={column name=\textbf{\sta}, column type={c}},
  columns/Lungo(rel)/.style={column name=\textbf{\lungoa}, column type={c}},
  columns/Kendo(rel)/.style={column name=\textbf{\kendoa}, column type={c}},
  every head row/.style={
    before row=\toprule,
    after row=\midrule},
  every last row/.style={
  	before row=\midrule,
    after row=\bottomrule}
]{tabellen/ondersteuning/formulieren.csv}
\caption{Ondersteuning voor \uit{formulieren}}
\label{tabel:evaluatie-ondersteuning-formulieren}
\end{table}

\paragraph{\jqm}
Op iedere deeluitdaging behaalt \jqm{} de maximumscore.
Het scrollen in de \term{datepicker} werkte echter niet vloeiend op de \htc{} en \gtab{}, maar de functionaliteit werkte wel.
Op de \ipadi{} werd opgemerkt dat door de lokale instellingen, een scheidingsteken voor de duizendtallen werd gebruikt voor het getal in het nummerveld.
Als laatste bleken er verschillende implementaties van de \js{}-functie \code{new Date(dateString)} doorheen de apparaten.
Als \code{dateString} werd het patroon \code{jaar/maand/dag} uiteindelijk gebruikt, want het patroon \code{jaar-maand-dag} gaf op de \htc{} en \gtab{} anders problemen.

\paragraph{\st}
Alle smartphones en de \gtab{} vertoonden een probleem bij de labels van de optieknoppen.
Deze zijn nodig om het type van uitgave te selecteren bij het aanmaken van een nieuwe uitgave.
Alle labels buiten \term{Hotel} en \term{Restaurant} zijn te lang om in de voorziene ruimte weer te geven.
\st{} kort deze af met ellipses waardoor niet de volledige naam leesbaar is.
Dit kan tot verwarring leiden.
In \term{landscape} mode worden sommige labels wel zichtbaar %
%TODO afgekorte titel oplossing met css extentie?
De tweede onderuitdaging faalt bij de \htc{} en \gtab{} aangezien geen email toetsenbord ondersteund wordt.
%TODO sander: referentie?
De laatste twee onderuitdagingen halen de maximumscore.


\paragraph{\kendo}
Alle Android toestellen toonden een wit kader bij \term{read-only} widgets.
Deze zijn te zien wanneer het overzicht van een ingevoerde uitgave wordt opgevraagd.
Dit ondankt de invoervelden standaard HTML-invoer zijn.
Het raamwerk moet de waarden van de widgets in de \term{read-only} invoerelementen invullen.
Deze functionaliteit werkt enkel op iOS toestellen.
De tweede onderuitdaging faalt bij de \htc{} en \gtab{} aangezien geen email toetsenbord ondersteund wordt.
%TODO sander: referentie?
Het openen van \term{data-picker} bij \kendo{} vindt plaats als de gebruiker op het icoontje in het invoerveld klikt.
Indien op de datum wordt geklikt zal de gebruiker de datum tektueel aanpassen.  
Hiervoor moet het correcte formaat worden gebruikt!
Nog een opmerking,  op alle toestellen wordt automatisch een virtueel toetsenbord weergegeven ondankt de invoer uit de \term{date-picker} moet worden gekozen.
%TODO nul geven?

\paragraph{\lungo}
De slechtste ondersteuning was op de \htc{} en \gtab{}.
Met deze toestellen kon zeer moeilijk op de velden worden geklikt.
Soms lukte dit, soms lukte dit niet.
Daarenboven kwam er geen aangepast toetsenbord voor een formulierveld van het type e-mail.
De aangepaste \term{datepicker} werkte op ieder toestel, wat in schril contrast staat met de schakelaar die op geen enkel toestel correct werkte.
Deze laatste deed verschillende acties naargelang geklikt of gesleept werd op de schakelaar.

%%%%%%%%%%%%%
\subsection{\uit{autoaanvullen}}
\label{sec:evaluatie-ondersteuning-autoaanvullen}

Aangezien deze uitdaging voor ondersteuning geen deeluitdagingen bevat, wordt er geen tabel getoond.
Hieronder wordt per raamwerk verklaard waarom dat resultaat werd behaald.

\paragraph{\jqm}
Ieder toestel ondersteunde de auto-aanvulling.

\paragraph{\st}
Deze uitdaging kon niet worden ondersteund omdat het raamwerk geen implementatie toeliet

\paragraph{\kendo}
Deze uitdaging behaalde op elk toestel de maximum score.
%TODO check gt en htc!!

\paragraph{\lungo}
Zeven van de acht toestellen ondersteunden auto-aanvulling.
De \htc{} toonde echter de suggesties dubbel op het scherm aan de gebruiker.

%%%%%%%%%%%%%
\subsection{\uit{afbeelding}}
\label{sec:evaluatie-ondersteuning-afbeelding}

Aangezien deze uitdaging voor ondersteuning geen deeluitdagingen bevat, wordt er geen tabel getoond.
Hieronder wordt per raamwerk verklaard waarom dat resultaat werd behaald.

\paragraph{\jqm}
Op vier apparaten kon een afbeelding met succes worden toegevoegd.
Daarentegen hadden de \htc{} en \gtab{} geen ondersteuning voor de FileReaderAPI, op de \gs{} bleef de applicatie hangen en als laatste bood de \ipadi{} geen ondersteuning voor het formulierveld van het type \code{file}.

\paragraph{\st}
Oudere Android types bieden geen ondersteuning voor de FileReaderAPI.
Hierdoor was het opladen van een afbeelding niet mogelijk bij de \htc{} en \gtab{}.
De \ipadi{} brengt geeft geen dialoogvenster naar boven nadat op de knop werd gedrukt omdat \code{file} niet als formulierveld wordt ondersteund.
%TODO referentie?
Er dient opgemerkt te worden dat de hoge resolutie camera aan de achterkant van de \ipadiii{}, \iphoneiii{} en \iphoneiv{} niet kon worden gebruikt.
Dit komt omdat de grootte van de fotos de limiet van de \code{local storage} overschreiden.
%TODO exacte limiet refereren?
Hierdoor moet de camera aan de voorkant worden gebruikt of een afbeelding met lagere resolutie worden gekozen.

\paragraph{\kendo}
De FileReader API en \code{file} als type van het formulierveld waren de vereisten voor deze uitdaging.
Hierdoor werd de implementatie voor \htc{},  \gtab{} en \ipadi{} niet ondersteund.

\paragraph{\lungo}
Aangezien dezelfde aanpak werd gebruik als \jqm{} is de ondersteuning exact hetzelfde.

%%%%%%%%%%%%%
\subsection{\uit{validatie}}
\label{sec:evaluatie-ondersteuning-validatie}

Aangezien deze uitdaging voor ondersteuning geen deeluitdagingen bevat, wordt er geen tabel getoond.
Hieronder wordt per raamwerk verklaard waarom dat resultaat werd behaald.

\paragraph{\jqm}
Deze uitdaging werkte op ieder toestel en behaalde zo de maximumscore.

\paragraph{\st}
Deze uitdaging werkte op ieder toestel en behaalde zo de maximumscore.

\paragraph{\kendo}
Deze uitdaging werkte op ieder toestel en behaalde zo de maximumscore.

\paragraph{\lungo}
Deze uitdaging werkte op ieder toestel en behaalde zo de maximumscore.

%%%%%%%%%%%%%
\subsection{\uit{handtekening}}
\label{sec:evaluatie-ondersteuning-handtekening}

Aangezien deze uitdaging voor ondersteuning geen deeluitdagingen bevat, wordt er geen tabel getoond.
Hieronder wordt per raamwerk verklaard waarom dat resultaat werd behaald.

\paragraph{\jqm}
De handtekening kon op ieder apparaat worden getekend, behalve op de \gs{}.
Daar werd tijdens het tekenen van de handtekening, de handtekening dubbel op het scherm getekend.

\paragraph{\st}
Deze uitdaging werkte op ieder toestel en behaalde zo de maximumscore.

\paragraph{\kendo}
Aangezien dezelfde aanpak werd gebruik als \jqm{} is de ondersteuning exact hetzelfde.

\paragraph{\lungo}
Doordat geen plug-in werd gevonden voor deze uitdaging, kon ook de ondersteuning niet worden onderzocht.

%%%%%%%%%%%%%
\subsection{\uit{pdf}}
\label{sec:evaluatie-ondersteuning-pdf}

Aangezien deze uitdaging voor ondersteuning geen deeluitdagingen bevat, wordt er geen tabel getoond.
Hieronder wordt per raamwerk verklaard waarom dat resultaat werd behaald.

\paragraph{\jqm}
Drie van de vier apparaten hadden problemen met het tonen van de PDF.
Op de \htc{} werd een wit scherm getoond, op de \gtab{} lukt het niet en op de \gs{} gaf het besturingssysteem de boodschap 'Download unsuccessful' terug.
Bij de vier andere apparaten werd de PDF-lezer geopend en kon de PDF worden bekeken.

\paragraph{\st}
De plug-in waarmee PDF-bestanden getoond worden, maakt gebruik van PDF.JS,  een PDF-renderer van Mozilla. %TODO referentie
PDF.JS is geïmplementeerd met \js{} en HTML5.
Het \code{canvas} kenmerk en ondersteuning voor de SVG (Scalable Vector Graphics) API is noodzakelijk.
Vanaf Android 3.0 wordt dit laatste slechts ondersteund.
Deze uitdaging lukt dus niet op de \htc{} en \gtab{}.
Een belangrijke opmerking is dat de ondersteuning van deze uitdaging werd getest op een implementatie van \st{} die niet gebouwd was met Sencha Cmd.
Dit omdat de gebouwde versie een foutenboodschap gaf wanneer de PDF plug-in werd aangesproken.
%TODO verder uitwerken?

\paragraph{\kendo}
Aangezien dezelfde aanpak werd gebruik als \jqm{} is de ondersteuning exact hetzelfde.

\paragraph{\lungo}
Aangezien dezelfde aanpak werd gebruik als \jqm{} is de ondersteuning exact hetzelfde.


%%%%%%%%%%%%%
\subsection{\uit{offline}}
\label{sec:evaluatie-ondersteuning-offline}

In tabel \ref{tabel:evaluatie-ondersteuning-offline} worden de resultaten getoond van de twee deeluitdagingen van \uit{offline}.
Onder de tabel wordt per raamwerk verklaard waarom dat resultaat werd behaald.

\begin{table}[H]
\centering
\pgfplotstabletypeset[
  begin table=\begin{tabular}{p{8cm} p{1cm} p{1cm} p{1cm} p{1cm}},
  end table=\end{tabular},
  skip coltypes=true,
  col sep=comma,
  string type,
  header=true,
  skip coltypes=true,
  columns={Uitdaging,ST(rel),Kendo(rel),jQM(rel),Lungo(rel)},
  columns/Uitdaging/.style={column name=\textbf{Uitdaging}, column type={l}},  
  columns/jQM(rel)/.style={column name=\textbf{\jqma}, column type={c}},
  columns/ST(rel)/.style={column name=\textbf{\sta}, column type={c}},
  columns/Lungo(rel)/.style={column name=\textbf{\lungoa}, column type={c}},
  columns/Kendo(rel)/.style={column name=\textbf{\kendoa}, column type={c}},
  every head row/.style={
    before row=\toprule,
    after row=\midrule},
  every last row/.style={
  	before row=\midrule,
    after row=\bottomrule}
]{tabellen/ondersteuning/offline.csv}
\caption{Ondersteuning voor \uit{offline}}
\label{tabel:evaluatie-ondersteuning-offline}
\end{table}

\paragraph{\jqm}
Alle acht apparaten ondersteunen zowel \code{localStorage} als Application Cache, wat beide HTML5-specificaties zijn.

\paragraph{\st}
Deze uitdaging behaalde bij alle toestellen de maximum score.

\paragraph{\kendo}
Deze uitdaging behaalde bij alle toestellen de maximum score.

\paragraph{\lungo}
Aangezien dezelfde aanpak werd gebruik als \jqm{} is de ondersteuning exact hetzelfde.