\chapter{Besluit}
\label{chap:besluit}

\section{Conclusie} % 1 pagina % Sander
% - wat we gedaan hebben, onze doelen, hebben we de beste gevonden?
% - we hebben dat gevonden, ….

Deze thesistekst bestaat uit twee doelstellingen.
Een eerste doel is het definiëren van een methodologie om HTML5-raamwerken met elkaar te vergelijken.
Het tweede doel omvat de effectieve vergelijking van de raamwerken zelf.

De uitgelichte raamwerken zijn \st{}, \kendo{},  \jqm{} en \lungo{}.
\st{} bouwt op het MVC-ontwerppatroon en is \js-gedreven.
Het raamwerk is gratis en heeft zowel een commerciële als \term{open-source} licentie.
\kendo{} dwingt het MVVM-ontwerppatroon af en is zowel \js- als opmaakgedreven.
Een licentie voor het gebruik van \kendo{} kost $\$699$.
\jqm{} en \lungo{} hebben geen ontwerppatroon en zijn beide opmaakgedreven.
Beide raamwerken zijn \term{open-source}.

Vijf criteria werden gekozen om de vergelijkende studie uit te voeren:  populariteit,  productiviteit,  gebruik,  ondersteuning en performantie.
Elk criterium werd voorzien van een formule om een score te berekenen voor het criterium.
%TODO gonzalo why poc
Een POC die werknemers toelaat onkosten toe te voegen, werd geformaliseerd en op zoveel mogelijk vlakken getest met de opgelegde criteria.
Om populariteit te meten werd naar de activiteit van de raamwerken op sociale netwerken gekeken.
De tijd om een loginapplicatie te ontwikkelen, bepaalde de productiviteit.
De POC werd onderverdeeld in $13$ uitdagingen en $38$ deeluitdagingen om de functionaliteit van het raamwerk te testen en het gebruikscriterium te quoteren.
Vervolgens werd een subset van de uitdagingen getest op acht verschillende mobiele toestellen om de ondersteuning te controleren.
%TODO responsetijden?
Ten slotte bepaalden de download- en responsetijden van de POC en loginapplicatie de performantie.
De responsetijden werden vervangen door een gebruikerservaring.
De scores van de vijf criteria voor de vier raamwerken werden in één spinnenweb ondergebracht.

Na evaluatie is \jqm{} het beste raamwerk op basis van de gekozen criteria, gevolgd door \kendo{}, \lungo{} en \st{}.
Deze volgorde werd bepaald door de scores van alle criteria op te tellen.
\jqm{} heeft als belangrijkste troeven de hoge productiviteit en performantie doordat het enerzijds zeer goed gedocumenteerd is en anderzijds geen ontwerppatroon afdwingt.
Dit laatste is echter een nadeel waardoor het minder scoort op gebruik.
\kendo{} heeft als belangrijkste troef het gebruik doordat het een ontwerppatroon afdwingt.
Het scoort echter ondermaats op performantie door het crashen van lange lijsten op iOS.
\lungo{} behaalt op geen enkel criterium de maximumscore.
Het behaalde echter wel de beste downloadtijd bij performantie doordat het raamwerk is geoptimaliseerd voor mobiel gebruik.
\st{} is het minst productief en minst performant in vergelijking met de andere raamwerken.
Daarentegen scoort \st{} het best op het vlak van gebruikerservaring.
Door het afdwingen van een ontwerppatroon scoort het quasi evengoed als \kendo{} op vlak van gebruik.
Alle onderzochte raamwerken scoren zeer goed op ondersteuning.


\section{Geleerde lessen} % 1 pagina 
Bij het uitvoeren van een vergelijkende studie is de keuze van de vergelijkingscriteria bepalend voor het resultaat van het onderzoek.
Daarbij is het belangrijk om reeds bestaande literatuur grondig te controleren.
Zo werd er in het begin te specifiek naar papers gezocht omtrent het vergelijken van HTML5-raamwerken.
Meer algemene methodologieën om software te vergelijken moesten worden gezocht waaruit criteria konden worden hergebruikt.
%TODO ahp hier toevoegen?

Bij het opzetten van criteria is het van groot belang dat de manier waarop deze criteria zullen worden getoetst, zeer gedetailleerd worden neergeschreven.
De exacte formules van de criteria werden pas tijdens de evaluatie vastgelegd.
Pas dan werd de nood van een formele notatie duidelijk.
Ook is het belangrijk om op voorhand de elementen van een vergelijkende studie uit te testen alvorens de vergelijkingscriteria vast te leggen.
Hierdoor kunnen iteraties over de criteria vermeden worden als blijkt dat ze niet toepasbaar zijn.
%TODO nog relevant?
De nieuwe formules voor het productiviteits- en performantiecriterium hadden hierdoor vermeden kunnen worden.
Een initiële vertrouwdheid met de raamwerken had er ook voor gezorgd dat de criteria meer kenmerken van de raamwerken zouden bevatten.
De raamwerken die steunen op een ontwerppatroon konden op die manier meer bevoordeeld worden door bijvoorbeeld een uitbreidbaarheidscriterium te introduceren.

%TODO 3x het woordje tijd, herschrijven
Een andere geleerde les is dat er in het uitvoeren van de evaluatie veel tijd kruipt.
Echter, het interpreteren en begrijpen van de resultaten duurde zowaar nog langer.
Ook het opsporen en verbeteren van eigen fouten is zeer tijdsrovend.
Door op een consistente en gestructureerde methode de evaluatie te voltooien, moet het aantal fouten tot een minimum worden beperkt.

Er werden tot slot nog twee praktische zaken geleerd.
Het opmeten van tijd en het gebruik van logboeken vereist een zekere vorm van discipline.
Het eerste was noodzakelijk om de productiviteit van het raamwerk te toetsen.
De tijdbudgetten die nodig waren moesten gemakkelijk kunnen worden gereconstrueerd.
De logboeken werden gebruikt bij de implementatie van de POC en moesten bij de evaluatie het gebruikscriterium drijven.
Een tweede praktisch punt gaat over de samenwerking tussen beide auteurs.
Het is geen gemakkelijke opgave om als twee individuen continu op gelijke hoogte te zitten.
Dit vraagt enorm veel onderlinge gesprekken met duidelijke communicatie.
Discipline, sociale netwerken en andere technologieën zoals Google Drive en \gh{} hielpen de communicatie te verbeteren.

\section{Verder onderzoek} % 1 pagina % Tim
Er kan verder gezocht worden naar de oorzaak waarom bepaalde zaken zeer opmerkelijk waren of waarom ze niet lukten tijdens de vergelijking.
Zo was er enerzijds de crash van de 850 lijstitems van \kendo{} op iOS.
Hier kan gezocht worden naar de oorzaak van de crash, maar tevens kan ook gezocht worden naar de grens van het aantal lijstitems waarbij dat het wel lukt.
Anderzijds was er een opmerkbare waarneming bij de performantie van de applicaties uit cache. 
Zo ligt bijvoorbeeld de gemiddelde downloadtijd van de login uit cache hoger dan deze van de POC voor \jqm{} en \lungo{}. 

Ook kunnen nieuwe raamwerken worden toegevoegd aan de vergelijking.
Hierdoor vergroot ten eerste de grootte van de vergelijking, maar kan ten tweede ook de methode telkens opnieuw worden getoetst met deze nieuwe raamwerken.
Daarnaast komen van de reeds vergeleken raamwerken geregeld nieuwe versies uit.
Zo is het ook mogelijk om de evolutie in de rangschikking van de vier vergeleken raamwerken over de tijd te bekijken.
Mogelijk kan de rangschikking veranderen bij het uitbrengen van nieuwe versies of plug-ins.

De huidige methode omvat vijf vergelijkingscriteria die worden gedreven door de POC.
Verder onderzoek kan deze POC uitbreiden met extra kenmerken zoals Pull\&Refresh.
Dit loopt in de lijn om ook andere gebeurtenissen te gebruiken dan alleen maar de \term{tap} gebeurtenis.
Andere gebeurtenissen zijn bijvoorbeeld \term{double tap}, \term{swipe}, \term{hold}, maar ook gebeurtenissen waar meerdere vingers voor nodig zijn zoals \term{rotate}.
Daarnaast is ook de integratie van HTML5-kenmerken zoals GPS, \term{push events}, \term{drag and drop}, video en audio in de raamwerken zeker het onderzoeken waard.

Naast het toevoegen van extra kenmerken aan de POC, kunnen criteria ook op andere manieren gecontroleerd worden.
Nu worden bij ondersteuning enkel apparaten met een Android- of iOS-besturingssysteem gebruikt.
Dit kan worden vervangen of uitgebreid naar andere besturingssystemen zoals Windows Phone en BlackBerry~OS.
Een andere voorbeeld is dat voor de downloadtijd bij performantie Wifi werd gebruikt voor de verbinding.
Andere verbindingsmogelijkheden zoals 3G kunnen worden gebruikt en hierdoor kunnen andere resultaten bekomen worden.
Een derde voorbeeld is de rendertijden die worden gebruikt bij performantie.
Deze konden niet worden opgemeten bij \st{} en \lungo{}.
Aanpassingen aan de broncode zouden het opmeten van rendertijden wel kunnen toelaten.
%TODO alternatief nogeens vermelden?
Daarnaast kan het voorgestelde alternatief ook verbeterd worden.
Als laatste kunnen ook de lijnen effectief geschreven code worden gebruikt bij productiviteit. 

Een andere onderzoekspiste is om nieuwe criteria toe te voegen.
Zo kan bijvoorbeeld het criterium uitbreidbaarheid worden onderzocht.
%TODO ref naar ISO?
Dit criterium omvat hoe gemakkelijk het gaat om de bestaande applicatie geïmplementeerd in een bepaald raamwerk uit te breiden.
Een te onderzoeken hypothese hierbij is dat raamwerken die een ontwerppatroon afdwingen beter zullen scoren dan raamwerken zonder ontwerppatroon.
Een bijkomende hypothese is dat de totale score van raamwerken die een ontwerppatroon afdwingen zal stijgen en deze van raamwerken zonder ontwerppatroon zal dalen.
Dit komt doordat de ene worden afgestraft op productiviteit en de andere op uitbreidbaarheid.
Mogelijk kan de rangschikking van de vier onderzochte raamwerken veranderen.
Anderzijds kan ook een criterium worden toegevoegd die kijkt naar het finale resultaat van het raamwerk.
Zo kunnen bepaalde raamwerken de \term{native look-and-feel} van mobiele besturingssystemen  nabootsen, andere raamwerken bieden dan weer standaard een frisse hedendaagse lay-out.

Andere onderzoeksvragen kunnen een stap terugnemen door bijvoorbeeld af te vragen of het baterijverbruik door webapplicaties een probleem vormt.
De bekomen data kan worden vergeleken met \term{native} en hybride applicaties.
Deze laatste vergelijking kan zelfs veralgemeend worden waardoor een vergelijking tussen web-, \term{native} en hybride applicaties zich opdringt. 


%%% Local Variables: 
%%% mode: latex
%%% TeX-master: "masterproef"
%%% End: 
