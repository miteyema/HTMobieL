\chapter{Besluit}
\label{chap:besluit}

\section{Conclusie} % 
% - wat we gedaan hebben, onze doelen, hebben we de beste gevonden?
% - we hebben dat gevonden, ….

\section{Geleerde lessen}
% beter en vollediger uitwerken van criteria op voorhand! niet enkel bekijken als je effectie moet gaan evalueren
% het aantal ongewenste verrassingen blijft zo beperkt
% POC updaten (pull-to-refresh,  meer items laden,  ...) HTML5 features meer toevoegen (GPS, audio,  drag and drop (herorden lijst, lang duwen), carousel met swipe, push eventes 
% toggle entries beter bijhouden, veel meer algemener 


\section{Toekomstig werk} % 1 pagina
% ook het criterium uitbreidbaarheid erbij betrekken, want nu komen ST en Kendo niet helemaal tot uiting in onze spidergraph
% onderzoeken ophalen icons in cache 
% onderzoeken crash kendo op ios
% nieuwe frameworks toevoegen + updates van huidige frameworks blijven controleren (resultaten ook updaten)
% methodologie blijven verder toetsen
% subjectieve gebruikservaringstesten met > 5 mensen
% het finale resultaat van de framework bekijken (look-and-feel van kendo, nice dialogs van lungo,...)
% is battery use an issue for web applications?
% hoe zit het met de downloadsnelheid als je iets anders dan WiFi gebruikt? 3G 4G…?
% vergelijking web / hybrid / native
% windows en blackberry ondersteunen?

%%% Local Variables: 
%%% mode: latex
%%% TeX-master: "masterproef"
%%% End: 
