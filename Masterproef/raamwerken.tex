\chapter{Mobiele HTML5-raamwerken}
\label{chap:raamwerken}

Dit hoofdstuk vergelijkt op een passieve manier de vier gekozen mobiele HTML5-raamwerken, namelijk \st{}~(\ref{sec:raamwerk-st}), \kendo{}~(\ref{sec:raamwerk-kendo}),\jqm{}~(\ref{sec:raamwerk-jqm}) en \lungo{}~(\ref{sec:raamwerk-lungo}).
Dit is een vergelijking gebaseerd op informatie van de raamwerken die in de literatuur kon worden gevonden.
Omkadering, code en ontwikkeling, functionele kenmerken en niet-functionele kenmerken zullen voor ieder raamwerk worden besproken.
In de laatste sectie (\ref{sec:raamwerken-tabel}) wordt een overzicht van deze passieve vergelijking weergegeven.

In samenspraak met Capgemini werd gekozen om \jqm{}, \st{} en \kendo{} te onderzoeken.
De twee eerstgenoemde werden gekozen om hun enorme populariteit~\cite{Firtman2013,Hales2012,Oeflman2011,David2011}.
%TODO refereer interview?
De laatstgenoemde had het voordeel dat deze de \term{native look-and-feel} kan nabootsen, wat ook een handig onderzoekspunt leek voor Capgemini.
Het tijdsbudget liet toe om ook een vierde raamwerk te vergelijken.
Doordat \tmp{} een volledige make-over ondergaat in de zomer van 2013 leek het de auteurs niet nuttig om de huidige versie te onderzoeken als deze toch volledig wordt vernieuwd.
\moobile{} werd ook uitgesloten door de nog zeer jonge versie van het raamwerk tijdens het onderzoek.
Als laatste werd besloten om \davinci{} uit te sluiten door de gebrekkige Engelstalige documentatie van dit Koreaans raamwerk.
%TODO waarom jqtouch niet:  \jqt{} is slechts een plug-in voor \jqm{} en \st{} is hierop verdergebouwd.
Uiteindelijk bleef \lungo{} over en dit werd het vierde raamwerk.

Zoals besproken in \ref{sec:mobiele-html5-raamwerken} kunnen raamwerken worden gecategoriseerd volgens twee aanpakken, namelijk opmaakgedreven en \js{}-gedreven~\cite{Oeflman2011}. 
Eerst wordt gestart met het bespreken van een volledig \js{}-gedreven raamwerk, namelijk \st{}~(\ref{sec:raamwerk-st}).
Daarna volgt \kendo{} die zowel \js{}- als opmaakgedreven is (\ref{sec:raamwerk-kendo}).
De laatste twee besproken raamwerken, \jqm{}~(\ref{sec:raamwerk-jqm}) en \lungo{}~(\ref{sec:raamwerk-lungo}), zijn beide opmaakgedreven.

%%%%%%%%%%%%%%%%%%%%%%%%%%%%%%%%%%%%%%%%%%%%%%%%%%%%%%%%%%%%%%%%%%
%%%%%%%%%%%%%%%%%%%%%%%%%%%%%%%%%%%%%%%%%%%%%%%%%%%%%%%%%%%%%%%%%%

\section{\st}
\label{sec:raamwerk-st}

\st{} wordt ontwikkeld door Sencha,  een bedrijf dat in 2010 is ontstaan als een samensmelting van Ext JS,  jQuery Touch en Raphaël.  
Ext JS is een \js{}-raamwerk voor de ontwikkeling van webapplicaties. 
jQuery Touch is een jQuery plug-in voor mobiele webontwikkeling.  
Het steunt op WebKit en voegt \term{touch events} toe aan jQuery.  
Raphaël,  ten slotte,  is een \js{}-bibliotheek voor vectortekeningen. 
Op het moment van schrijven is \st{} aan versie 2.2~\cite{Inc.}.  

\subsection{Omkadering}
\paragraph{Programmeertaal}
\st{} is \js{}-gedreven, dus alle functionaliteit wordt in \js{} geïmplementeerd. 
Alle HTML-code wordt bij het bekijken van de pagina gegenereerd.  

\paragraph{Tools}
Naast \st{} levert Sencha nog producten die \st{} uitbreiden of het leven van de ontwikkelaar makkelijker maken.  
Deze worden hieronder opgelijst~\cite{Inc.}.  

\subparagraph{Sencha Architect}
Dit is een andere desktopapplicatie die het ontwikkelingsproces vergemakkelijkt met een GGI en \term{drag-and-drop} commando's~\cite{Sencha2012a}.  
De applicatie kan worden gedownload voor Mac, Windows en Linux distributies.
Sencha Architect kan voor 30 dagen uitgeprobeerd worden,  daarna moet een licentie worden aangekocht.
De prijs voor één ontwikkelaar bedraagd $\$399$~\cite{Inc.}.

\subparagraph{Sencha Cmd}
Deze tool vergemakkelijkt de ontwikkeling van \st{} applicaties door middel van commando's die vanuit de terminal kunnen worden uitgevoerd~\cite{Sencha2012}.
Sencha Cmd is beschikbaar voor Mac, Windows en Linux distributies.
Het kan initiële applicaties opzetten,  bestanden toevoegen en de applicatie bouwen en uitrollen.
De applicatie kan ook gebouwd worden naar een \term{native} applicatie voor iOS en Android door de webapplicatie om te vormen.
De tool is gratis op de Sencha website te downloaden~\cite{Inc.}.

\subparagraph{Sencha Animator}
Dit is een desktopapplicatie om CSS3-animaties te ontwerpen~\cite{Sencha2012b}.  
De applicatie kan worden gedownload voor Mac, Windows en Linux distributies.
Deze animaties worden enkel in WebKit-browsers ondersteund.
De prijs voor één ontwikkelaar bedraagd $\$99$~\cite{Inc.}.

% \subparagraph{Sencha GXT}
% Sencha GXT is een uitbreiding op Google Web Toolkit (GWT).  
% De compiler van GWT laat toe applicaties in Java te schrijven en ze te compileren naar geoptimaliseerde,  \term{cross-browser} HTML5 en \js{}.  
% Sencha GXT voegt grafieken,  \term{widgets}, etc. toe aan GWT.
% 
% \subparagraph{Sencha.IO}
% Deze uitbreiding zorgt voor \term{cloud services} binnen mobiele applicaties.  

\paragraph{Documentatie}
Alle documentatie voor \st{}~2.1.1 is te vinden op \exturl{docs.sencha.com/touch/2-0}.  
Een zoekfunctie voor objecten,  eigenschappen en methoden is aanwezig om snel zaken op te zoeken.  
De meeste functionaliteiten zijn voorzien van codevoorbeelden samen met het resultaat hoe de browser de code rendert.  
Verder biedt de Sencha website ook hulpmiddelen om \st{} te leren gebruiken.  
Hier staan handleidingen, introductievideos, enz.

Een ander handig naslagwerk is de Kitchen Sink~\cite{Inc.2013}.  
Dit is een webapplicatie,  geschreven in \st{},  die de belangrijkste functionaliteiten bevat samen met de bijhorende code.  

\paragraph{Marktadoptatie}
Volgens de Sencha website is 50\% van de Fortune~100 - een lijst van de grootste Amerikaanse bedrijven gerangschikt op jaaromzet - een Sencha-klant~\cite{Inc.}.  
Enkele van hun grootste klanten zijn CNN,  Samsung,  Cisco en  Visa.

\paragraph{Licenties}
\st{} is gratis binnen een commerciële context waarbij het bedrijf in kwestie de broncode niet deelt voor zijn gebruikers.  
De gratis \term{open-source} versie van \st{} laat dit wel toe.  
Deze komt met een GNU GPLv3 \term{open-source} licentie wat wil zeggen dat de vrijheid bestaat om aanpassingen aan de broncode te maken en te verspreiden,  zolang de code ook maar gratis verspreid wordt voor alle gebruikers.
  
Voor de ontwikkeling van eigen raamwerken of SDKs wordt een \term{original equipment manufacturer} (OEM) licentie voorzien.  
Dit wil zeggen dat bedrijven hun producten gaan verkopen onder hun eigen merk en naam, maar gebruik maken van Sencha.  
Omdat het gebruik hiervan per gebruiker verschilt,  worden OEM-licenties op maat gemaakt~\cite{Inc.}.

\subsection{Code en ontwikkeling}
Alle code moet in \js{} worden geschreven.
Eén HTML-bestand dient slechts als container om de bestanden in te laden.  
\st{} valt dus onder \js{}-gebaseerde raamwerken.  
De keuze voor deze aanpak heeft twee belangrijke motivaties.  
Enerzijds is \st{} gebouwd op Ext JS,  wat op zich een \js{}-raamwerk is.  
Anderzijds zorgt het voor een betere ondersteuning voor toestellen met verschillende resoluties.  
Samen met SASS en Compass kan \st{} lay-outs definiëren per apparaat (zie sectie \ref{sec:sencha-aanpasbaarheid}).  
De \code{Ext.env.Browser} en \code{Ext.env.OS} eigenschappen en \code{Ext.Viewport.getOrientation} en \code{Ext.feature.has} methoden kunnen de vereisten bepalen en de juiste lay-out kiezen~\cite{JohnEClark2012}.

Zoals besproken biedt Sencha de terminal tool Sencha Cmd aan om het de ontwikkelingsproces te vergemakkelijken.  

\paragraph{Debugging}
De broncode van \st{} kan ingeladen worden met \code{sencha-touch-debug.js} als bibliotheek.  
Deze versie is niet gecomprimeerd en bevat commentaar en documentatie om makkelijker te zoeken waar in de code de fout zich juist bevindt.
%TODO is algemeen voor alle frameworks:
%Het debuggen van code gebeurt voornamelijk in de browser zelf.  
% Tools als de Safari Web Inspector,  Chrome Developer Tools of Firebug moeten de fouten kunnen opsporen. 

\subsection{Functionele kenmerken}
%Net \st{} heeft functionaliteiten om eenvoudig GI-elementen te genereren.  
\st{} bevat alle elementen van de GI als \js{}-objecten.  
Net zoals alle objectgerichte programmeertalen maken deze objecten gebruik van een klassensysteem,  iets wat slechts vanaf \st{} 2 werd ingevoerd.  
Op die manier kunnen klassen worden gedefinieerd (\code{Ext.define}) en aangemaakt (\code{Ext.create}).  
Hierbij is ook overerving mogelijk.  
De basisklasse van alle objecten is \code{Ext.Component}.  
Componenten kunnen worden gerenderd, zichzelf tonen of verbergen,  centreren op het scherm en zichzelf aan- of uitzetten.   
Het aanmaken van componenten kan ook door de component als \code{xtype} te definiëren.  
Andere componenten kunnen deze \code{xtype} dan hergebruiken.

Een andere belangrijke component is \code{Ext.Container}.  
Containers kunnen subcomponenten bevatten en een lay-out specificeren.  
Alle componenten krijgen een naam die verwijst naar een \term{namespace}.  
Dit is handig om conflicten te vermijden tussen eigen objecten en standaard objecten van het raamwerk.  
Voor een opsomming van alle componenten wordt verwezen naar de documentatie~\cite{Inc.2013a}.

\paragraph{Model}
Data kan intern worden voorgesteld met \code{Models}.  
Dit is iets wat hoort bij de MVC-architectuur (zie sectie \ref{sec:sencha-programeerbaarheid}).  
Een model specificeert een lijst van velden waarbij een veld een naam en een type heeft.  
Optioneel kunnen validaties bij de velden worden toegevoegd om data consistent te houden.  

\paragraph{Store}
\code{Ext.data.Store} is de klasse om instanties van een model op te slaan.  
Een \code{Store} wordt voorzien van een \code{Proxy}.  
Deze kan data aan de klant of server zijde opslaan.  
Een \code{Proxy} voor opslag aan klant zijde kan zowel in het RAM-geheugen als in de \term{local storage} en \term{session storage} van de browser opslaan.  
Een \code{Proxy} voor server opslag kan data verzenden via AJAX (zelfde domein) of JSONP (verschillende domeinen).  
JSONP staat voor JavaScript Object Notation with Padding en is een methode om data op een server in een ander domein op te vragen.
Een \code{Proxy} kan ook nog voorzien worden van een \code{Reader} die aangeeft hoe de ontvangen data gelezen moet worden.

\paragraph{View}
Een \code{View} is de benaming voor objecten die aan de gebruiker kunnen worden getoond.  
Een voorbeeld hiervan zijn lijsten,  waar instanties van een \code{Store} kan worden weergegeven.  
Zo'n lijst kan makkelijk gefilterd of gesorteerd worden op basis van velden uit het model.
Hiervoor moeten \code{Filters} of \code{Sorters} aan de \code{Store} worden toegevoegd. 
De lay-out van één lijstitem bepalen kan via een \code{XTemplate}.  
Het sjabloon bepaalt de HTML-structuur van elk item.  
Alle gedefinieerde velden van het model kunnen in de template worden opgeroepen of gemanipuleerd.

\paragraph{Controllers} 
De \code{Controllers} zijn maken de binding tussen \code{Models} en \code{Views}.
Ze kunnen gebeurtenissen opvangen en bijhorende operaties uitvoeren.
Hierdoor is navigatie door de applicatie mogelijk en kunnen alle componenten gemanipuleerd worden.
\code{Controllers} worden ook gebruikt bij de initialisatie van de applicatie.

\subsection{Niet-functionele kenmerken}
\paragraph{Performantie}
In vergelijking met versie~1.1 van \st{} is de performantie gestegen om wille van verschillende factoren.  
De introductie van het klassensysteem,  zoals besproken in de vorige sectie,  laat toe objecten dynamisch te laden. 
Het grote verschil tussen \code{Ext.define} en \code{Ext.create} is dat objecten enkel in het geheugen worden geladen na creatie.  
Het is dus de taak van de programmeur om objecten enkel te construeren wanneer ze nodig zijn.

Verder kwam versie~2.0 met een nieuwe lay-out \term{engine} die vooral het verwisselen van oriëntatie van het toestel versnelde.  
Ook een verbetering in performantie op Android-toestellen,  voornamelijk bij scrollen en animaties,  werd ingevoerd~\cite{Inc.}.

\paragraph{Aanpasbaarheid}
\label{sec:sencha-aanpasbaarheid}
Elke component binnen het raamwerk moet overerven van \code{Ext.Component}.  
Deze voorziet een attribuut \code{ui}.  
De waarde hiervan is een CSS-klasse die bepaald hoe de component er zal uitzien.  
\st{} heeft al twee CSS-klassen voorzien:  \code{light} en \code{dark}.  
Andere componenten kunnen deze lijst uitbreiden.  
Een knop kan bijvoorbeeld \code{normal},  \code{back},  \code{round},  \code{small},  \code{action} of \code{forward} als \code{ui} waarde hebben.

Het is ook mogelijk om eigen waarden voor \code{ui} te definiëren of de standaarden van \st{} aan te passen.  
SASS en Compass maken dit mogelijk door eigen CSS-bestanden aan te maken.  
SASS staat voor Syntactically Awesome Stylesheets en breidt CSS uit met variabelen,  geneste structuren, \term{mixins} en overerving~\cite{Eppstein2013}.  
\term{Mixins} groeperen enkele CSS-eigenschappen en kunnen worden hergebruikt.  
Compass is een raamwerk bovenop SASS en CSS.  
Het compileert SCSS (Sassy CSS) naar CSS-bestanden~\cite{Eppstein2013a}.        

\st{} thema's bestaan allemaal uit een set van \term{mixins}.  
Door zelf \term{mixins} te creëren of reeds bestaande te manipuleren kunnen eigen thema's gecreëerd worden en ze aan de \code{ui}-waarde van een component toegekend worden.

\paragraph{Programmeerbaarheid}
\label{sec:sencha-programeerbaarheid}
Zoals reeds aangehaald ondersteund \st{} de MVC-architectuur.  
Deze architectuur vermijdt lange \js{}-bestanden door ze logisch op te delen.  
Modellen groeperen velden tot een beschrijving van data-objecten, \code{Views} definiëren de weergave van componenten en controllers verbinden beide op basis van gebeurtenissen.

In theorie zou het verschil tussen mobiele websites en applicaties enkel in de \code{Views} terug te vinden zijn.  
Echter,  dit wordt nog niet volledig ondersteund en worden aparte projecten voor deze functionaliteit gepromoot.

\paragraph{Ondersteuning browser}
\st{} steunt op de WebKit-browser \term{engine} dus moet de browser deze bevatten.  
Hoewel dit bij de meeste browsers geen probleem meer vormt, vallen toch enkele populaire browsers uit de boot.  
\st{} is bijvoorbeeld niet compatibel met FireFox Mobile en Opera Mobile~\cite{JohnEClark2012}.
De volgende versie van de Opera browser zal de WebKit \term{engine} wel bevatten~\cite{Wokke2013}, een trend dat de meeste browser verkopers zullen (moeten) volgen.

Zoals reeds vermeld zijn er ook methoden voorzien om informatie op te vragen over de context die gehanteerd wordt (browser, besturingssysteem, toestel, etc.).  
Verder kan \st{} ook vragen naar de ondersteuning van specifieke kenmerken (audio,  canvas,  CSS3, ...)  analoog als Modernizr~\cite{Modernizr2012}.  

Op de Sencha website zijn voor de belangrijkste browsers en bijhorend besturingssystemen \term{scorecards} voorzien om hun compatibiliteit met HTLM5 en \st{} te bespreken~\cite{Inc.}.

%%%%%%%%%%%%%%%%%%%%%%%%%%%%%%%%%%%%%%%%%%%%%%%%%%%%%%%%%%%%%%%%%%
%%%%%%%%%%%%%%%%%%%%%%%%%%%%%%%%%%%%%%%%%%%%%%%%%%%%%%%%%%%%%%%%%%

\section{\kendo}
\label{sec:raamwerk-kendo}
\kendo{} is een raamwerk van de hand van Telerik.
Het bestaat uit drie luiken:  Web, Mobile en DataViz.  
Het eerste is gericht op de ontwikkeling van desktop- en mobiele applicaties,  het tweede voegt een \term{native look-and-feel} toe aan mobiele applicaties en het laatste zorgt voor datavisualisatie met HTML5- en \js{}-technologie.
\kendo{} is een zowel een \js{}- als opmaakgedreven raamwerk en heeft een MVVM-architectuur (zie infra) dat steunt op de jQuery bibliotheek.
Verder heeft de ontwikkelaar ook de mogelijkheid om eenvoudig de \term{backend} te integreren aan de klantzijde.
.NET,  PHP en JSP zijn momenteel de ondersteunde technologieën voor \term{backend} integratie.
Op het moment van schrijven is \kendo{} aan versie 2013 Q1~\cite{Telerik}. 

\subsection{Omkadering}
\label{sec:kendo-omkadering}

\paragraph{Programmeertaal}
\kendo{} kan zowel als \js{}- en opmaakgedreven beschouwd worden. 
Data-attributen in kunnen een HTML-element associëren met \kendo{} of het overeenkomstige jQuery object kan in \js{} het raamwerk oproepen en het element initialiseren.
Alle GI-elementen van \kendo{} Mobile kunnen met data-attributen worden opgebouwd.
\term{Widgets} van \kendo{} Web kunnen zowel met \js{} als data-attributen geïnitialiseerd worden.

Het laden van het raamwerk kan door heel \kendo{} op te roepen of enkel \kendo{} Mobile met respectievelijk \code{kendo.all.js} en \code{kendo.mobile.js}.
Elk van de drie luiken - Web, Mobile en DataViz - kan op zichzelf functioneren door hun \js{}-bestand in te laden.
Er kan wel slecht één van de drie script gelijktijdig gebruikt worden.  
Wanneer elementen uit verschillende luiken gebruikt worden, moet \code{kendo.all.js} worden gebruikt.
Een alternatieve oplossing is de keuze van het script van één luik en alle benodigde \js{}-bestanden te genereren met de \js{} Builder op \url{http://www.kendoui.com/custom-download.aspx}.
Hier kunnen de vereiste elementen geselecteerd worden en wordt het vereiste \js{}-bestand gegenereerd.
Op een analoge wijze als de \js{}-bestanden, kan de programmeur kiezen tussen verschillende \term{stylesheets}:  \code{kendo.all.css} en \code{kendo.mobile.css}.

\paragraph{Tools}
Op de \kendo{}-website staan drie webtools vermeld die Telerik aanbiedt om de programmeur te ondersteunen.
De eerste is \kendo{} Dojo~\cite{Telerika},  een interactieve leeromgeving om met \kendo{} vertrouwd te raken.
De gebruiker kan de basis van \kendo{} leren kennen met geleide handleidingen en uitvoerbare voorbeelden.
De twee andere webapplicaties zijn een ThemeBuilder voor Web en Mobile die op een grafische manier CSS-bestanden kunnen genereren~\cite{Telerikb,Telerikc}.
Voor \kendo{} Mobile kan een verschillende lay-out bepaald worden voor alle ondersteunde platformen.

\paragraph{Documentatie}
Alle documentatie kan gevonden worden op \url{http://docs.kendoui.com}~\cite{Telerikd}.
Twee belangrijke secties binnen de documentatie zijn de API en Getting Started.
Beide kunnen op elkaar gemapt worden omdat alle objecten van \kendo{} die in de API worden aangehaald ook in een pagina onder \term{Getting Started} worden besproken.
Deze laatste probeert met meer woorden en voorbeelden uit te leggen wat het object juist inhoudt.
Verder staan er bij de documentatie nog handleidingen die complexere functionaliteit uit de doeken doet.
Ook zijn er demo's die live voorbeelden tonen samen met de code die nodig is om het voorbeeld te maken.

\paragraph{Marktadoptatie}
Enkele van de populairste klanten van \kendo{} zijn Nikon,  Fujifilm en Symantec~\cite{Telerike}.

\paragraph{Licenties}
Een licentie voor \kendo{} Complete kost $\$699$ per ontwikkelaar.
Voor \term{backend} ondersteuning in PHP,  JSP of ASP.NET MVC moet $\$300$ meer betaald worden.
Hierbij zijn één jaar updates mogelijk en wordt professionele ondersteuning aangeboden met een responsetijd onder 48 uur.
Bij een licentie met \term{backend} integratie is support zelfs gegarandeerd na 24 uur.
Voor \kendo{} Web,  Mobile en DataViz bestaan ook een aparte licenties voor respectievelijk $\$399$,  $\$199$ en $\$399$~\cite{Telerik}.

\subsection{Code en ontwikkeling}
Zoals reeds vermeld moet de programmeur zowel \js{}- als HTML-code schrijven. 
De \js{}- en CSS-bestanden van het raamwerk moeten in de projectfolder worden gekopieerd respectievelijk in een \term{js}- en \term{styles}-map.
\kendo{} steunt op de jQuery-bibliotheek en deze moet ingeladen worden voor het \kendo{} raamwerk zelf wordt aangeroepen.
De initialisatie van een applicatie moet via \code{var app = new kendo.mobile.Application()}.
Hier kunnen parameters meegegeven worden die bijvoorbeeld de stijl van één platform vastlegt voor alle toestellen of het initiële scherm bepalen.

Net zoals bij \jqm{} zijn er drie strategieën om webapplicaties te maken:  volledige applicatie in één webpagina,  elk scherm in een aparte pagina of een combinatie van beide.
De navigatie naar een scherm gebeurt op basis van de \term{identifier} van dat scherm.
Standaard navigeert \kendo{} naar het eerste gedefinieerde scherm van een webpagina.
Een ander scherm kan in dezelfde pagina of in een ander bestand staan.
Een lokale navigatie wordt herkend door een \term{hashtag} die voor de id van het scherm wordt geplaatst als parameter van de \code{navigate}-methode.
Navigatie naar een ander bestand kan door de bestandsnaam als parameter op te geven.

\subsection{Functionele kenmerken}
\label{sec:kendo-functioneel}
%TODO klassensysteem
\kendo{} is zowel opmaak- als \js{}-gedreven en steunt op de MVVM-architectuur.
Dit beïnvloedt sterk alle functionele kenmerken.

\paragraph{UI-elementen}
Formulieren volgen de de HTML5-norm. 
Deze elementen zijn wel enkel functioneel op iOS 5.x en Android 4.x en hoger.  
Het stijl van de elementen op andere platformen zal werken, maar is beperkt tot  enkel tekstinvoer~\cite{Telerike}.

Het toevoegen van knoppen kan zowel met de \code{button}-tag als met standaard hyperlinks (\code{<a>}).
Knoppen kunnen ook samengevoegd worden tot een \code{ButtonGroup}.
Dit maakt het mogelijk om gemeenschappelijke acties aan een groep van knoppen toe te kennen om bijvoorbeeld een menu te maken.
Een \code{TabStrip} is een alternatief waar tabs in de voettekst het scherm kunnen laten variëren.

\paragraph{View}
Schermen worden voorgesteld met \code{Views},  analoog als bij de MVC-architectuur.
Een \code{View} aanduiden gebeurt door het attribuut \code{data-role} aan \code{View} gelijk te stellen.
\code{Views} kunnen met een lay-out worden voorzien met de \code{data-layout}-tag.
Een \code{Layout} bepaalt de vormgeving van een \code{View} en kan hergebruikt worden.

Een \code{ListView} is een specifieke \code{View} voor lijsten.
De \code{data-template} kan bij lijsten de \term{identifier} van een sjabloon bevatten die de opmaak van de lijstelementen definieert.
Deze sjablonen zijn specifieke \kendo{} scripts die HTML-tags en \js{}-code kunnen bevatten.
Ook kunnen ze verwijzen naar velden van het model dat aan de lijst is toegekend (zie infra).

Twee andere instanties van \code{Views} zijn \code{SplitView} en \code{ScrollView}.
De eerste kan het scherm in twee \code{Views} splitsen,  vaak gebruikt bij tabletapplicaties.
De tweede definieert een verzameling van pagina's die met een \term{swipe} bewegingen gelinkt zijn.

\paragraph{View-Model}
Het \code{View-Model} behoort tot de kern van \kendo{} en wordt \code{ObservableObject} genoemd.
Dit is een \js{}-object dat kan gebonden worden aan abonnees.
Het ondersteunt het monitoren van wijzigingen en verwittigt elke abonnee wanneer een wijziging zich voordoet.
Een \code{ObservableObject} kan aan een \code{View} worden toegekend door het in de \code{data-model}-tag te vermelden.

Er zijn verschillende bindingen	 mogelijk tussen een \code{View} en \code{ObservableObject}.
Deze wordt aangegeven in de \code{data-bind}-tag.
\kendo{} ondersteunt een binding met volgende eigenschappen:  \code{attr,  checked, clicked, custom, disabled, enabled, events, html, invisible, source, style, text, value} en \code{visible}.
Als een gebonden eigenschap wijzigt - door gebruikersinvoer of programmatisch - zal het overeenkomstige veld in het \code{ObservableObject} ook wijzigen.

\paragraph{Model}
Het \code{Model}-object erft over van \code{ObservableObject} en breidt het uit met de mogelijkheid om schema's,  velden en methoden te definiëren.  
Velden kunnen van het type \code{string, number, boolean} en \code{date} zijn.
Ook kunnen de velden verder beschreven worden door bijvoorbeeld een standaard waarde of validatie toe te voegen.
Een schema is een eigenschap van een \code{DataSource},  een \kendo{} object voor de opslag van lokale of externe data.  
Een \code{DataSource} ondersteunt alle CRUD (\term{Create, Read, Update en Delete}) operaties en het sorteren, pagineren, filteren, groeperen en aggregeren van data.
Het schema attribuut legt de structuur van de data in de \code{DataSource} vast.
Bij externe databronnen bepaalt het hoe binnenkomende data geparset moet worden om aan de opgelegde structuur te voldoen.
Een \code{Model} kan ook als waarde van het attribuut worden gezet.
Dit wil zeggen dat de bijhorende \code{DataSource} instanties zal bevatten van het toegekende \code{Model}.

\subsection{Niet-functionele kenmerken}
\label{sec:kendo-niet-functioneel}

\paragraph{Performantie}
De performantie van een \kendo{} applicatie wordt deels bepaald door de programmeur.
Deze moet er voor zorgen dat de data op het juiste moment geladen wordt.
Bij het weergeven van een \code{View} gaan drie gebeurtenissen vooraf,  namelijk \code{beforeShow,  init} en \code{show}.
De eerste wordt uitgevoerd voor een \code{View} zichtbaar wordt,  de tweede na initialisatie en de laatste bij het tonen van een \code{View}.
Het initialiseren van een \code{View} vindt maar één keer plaats nadat de volledige applicatie geladen is.
Bij de ontwikkeling van een \code{ListView} met data van een externe databron kan best de \code{DataSource} geladen worden bij het initialiseren van de applicatie,  de lijst gemaakt worden bij de \code{init}-gebeurtenis en de lijst ververst worden bij een \code{show}-gebeurtenis.

\paragraph{Aanpasbaarheid}
\kendo{} probeert de \term{native look-and-feel} van verschillende besturingssystemen na te bootsen.
Het \kendo{} pakket bevat ook tien extra thema's die een alternatieve lay-out bepalen.
Deze zijn elk nog persoonlijk aan te passen met de Mobile ThemeBuilder zoals beschreven in de sectie \ref{sec:kendo-omkadering}.

\paragraph{Programeerbaarheid}
\kendo{} is zowel JavaScipt- als opmaakgedreven.
Een kennis van zowel HTML als \js{} is vereist om met dit raamwerk aan de slag te kunnen.
Het raamwerk is gebouwd op de jQuery Core en maakt dus vaak van jQuery's \term{selectors} gebruik.
Ook kan een AJAX verzoek met jQuery syntax geformuleerd worden om externe data op te halen voor een \code{DataSource}.

\paragraph{Browserondersteuning}
Zoals reeds vermeld herkent \kendo{} het platform waarop de applicatie wordt uitgevoerd.
De lay-out van de applicatie zal de \term{native look-and-feel} van het besturingssysteem vervolgens nabootsen.
Ondersteunde systemen zijn iOS, Android, BlackBerry en Windows Phone~8.

Alle widgets, zoals gebruikt in het raamwerk, ondersteunen \term{progressive enhancement}.
Oudere browsers kunnen zo bestaande inhoud en functionaliteit raadplegen met \term{native} HTML-types indien bepaalde elementen niet worden ondersteund.
Ook de HTML5 formulierelementen worden opgebouwd met \term{progressive enhancement}.

%%%%%%%%%%%%%%%%%%%%%%%%%%%%%%%%%%%%%%%%%%%%%%%%%%%%%%%%%%%%%%%%%%
%%%%%%%%%%%%%%%%%%%%%%%%%%%%%%%%%%%%%%%%%%%%%%%%%%%%%%%%%%%%%%%%%%

\section{\jqm}
\label{sec:raamwerk-jqm}
\jqm{} is een opmaakgedreven raamwerk dat werd aangekondigd in 2010 en hoofdzakelijk gebruikersinterface-elementen (GI-elementen) aanbiedt~\cite{Resig2010}.
In november 2011 werd versie~1.0 uitgebracht~\cite{Parker2011} en een jaar later werd in oktober versie~1.2 uitgebracht~\cite{Parker2012}. 
Op het moment van schrijven zit \jqm{} aan versie~1.3.1~\cite{Parker2013b}. 
Het raamwerk wordt beheerd door het jQuery Project dat onder andere jQuery Core beheert en waar \jqm{} afhankelijk van is~\cite{JQuery2012}. 
\jqm{} wordt door onder andere Adobe, BlackBerry en Mozilla gesponsord~\cite{JQuery2012a}.

\subsection{Omkadering}
\paragraph{Programmeertaal}
Om met \jqm{} aan de slag te kunnen, is niets meer nodig dan kennis over HTML, CSS en \js{}. 
Alle GI-elementen worden geschreven in HTML en aangeduid met \code{data-}* attributen.

\paragraph{Tools}
Een standaard teksteditor voldoet om met \jqm{} aan de slag te kunnen. 
Natuurlijk kan het gemakkelijk zijn om van \term{integrated development environments}~(IDE's) zoals Aptana Studio~\cite{Aptana2012} of WebStorm~\cite{JetBrains2012} gebruik te maken, waardoor handige kenmerken zoals automatische code-aanvulling beschikbaar zijn.

Het is ook mogelijk om gebruik te maken van Codiqa~\cite{Sperry2012} om de GI-elementen op het scherm te slepen en neer te zetten. 
Codiqua zal automatisch op de achtergrond de HTML-code voorzien.

\paragraph{Documentatie}
Op de documentatiesite van versie~1.2~\cite{JQuery2012b} is een catalogus te vinden van alle mogelijke elementen waarover \jqm{} beschikt. 
Door de broncode van een voorbeeld te bekijken, kan worden gekeken welke code moet worden geschreven om tot dat resultaat te komen.

Naast de GI-elementen is er ook documentatie over de API. 
Deze gaat over initiële configuraties, \term{events} en methodes die kunnen worden gebruikt.

\paragraph{Marktadoptatie}
Op de website van \jqm{} wordt een reeks applicaties getoond die gemaakt zijn met hun raamwerk. 
Enkele voorbeelden zijn webapplicaties voor Ikea, Disney World, Stanford University en Moulin Rouge~\cite{JQuery2012a}. 

\paragraph{Licenties}
Sinds september 2012 is het enkel nog mogelijk om \jqm{} onder de Massachusetts Institute of Technology (MIT) licentie te verkrijgen~\cite{Dmethvin2012}. 
Dit betekent dat de code wordt vrijgegeven als \term{open-source} en dat deze tegelijkertijd kan worden gebruikt in propriëtaire projecten en applicaties~\cite{PhilDutson2012}.

\subsection{Code en ontwikkeling}
Zoals werd aangehaald, wordt voornamelijk HTML5-code geschreven voorzien van \code{data-}* attributen. 
Daarna zal het raamwerk door middel van \term{progressive enhancement} allerhande code toevoegen om de beoogde GI-elementen correct te tonen in de browser. 
Dit wordt verder uitgelegd in de sectie browserondersteuning (zie \ref{sec:jqm-browser-support}).

Er zijn drie strategieën om webapplicaties te maken in \jqm{}~\cite{Broulik2012}. 
Een eerste is om de volledige applicatie in één webpagina te schrijven. 
De vele schermen van de webapplicatie zijn dan allemaal samengebracht op eenzelfde webpagina. 
Het voordeel bij deze aanpak is dat er initieel minder verzoeken zijn naar de server omdat alles in één bestand wordt opgehaald. 
Dit geldt ook zo voor de geïmporteerde CSS- en \js{}-bestanden. 

Een tweede strategie is om voor ieder scherm een aparte webpagina aan te maken. 
Het voordeel hierbij is dat de eerste pagina waar de gebruiker op terecht komt, sneller wordt gedownload. 
Bij iedere navigatie naar een ander scherm, moet dit scherm via Asynchronous \js{} and XML~(AJAX) worden opgehaald, waardoor dit vertragend kan werken. 

Een laatste strategie is om een mix tussen beide te maken. 
Men kan bijvoorbeeld alle schermen die de gebruiker vaak nodig heeft op één webpagina plaatsen. 
De schermen die de gebruiker zelden nodig heeft, worden op aparte webpagina's geplaatst.   

\subsection{Functionele kenmerken}
\jqm{} is een raamwerk dat voornamelijk GI-elementen aanbiedt, met name pagina's en dialoogvensters, werkbalken, knoppen, inhoud vormgeven, elementen voor formulieren en lijsten~\cite{JQuery2012b}.
Deze kenmerken zijn gebaseerd op versie~1.2.

\begin{enumerate}
\item \textbf{Pagina's en dialoogvensters}
De basisstructuur van een pagina bestaat uit een koptekst, inhoud en voettekst. 
Bij het overgaan naar een andere pagina wordt gekozen uit tien overgangseffecten. 
Voordat deze overgang gebeurt, zal \jqm{} altijd eerst die pagina ophalen via AJAX en inladen in het DOM. 
Zo kan een soepel overgangseffect worden getoond aan de gebruiker. 
Daarnaast is het ook mogelijk om gelinkte pagina's op voorhand op te halen. 
Als laatste biedt \jqm{} ook dialoogvensters en pop-ups aan. 

\item \textbf{Werkbalken}
Het is mogelijk om zowel knoppen bij de koptekst als bij de voettekst te plaatsen. 
Bij deze laatste kunnen typisch meer knoppen geplaatst worden, bij de koptekst slechts twee. 
Daarnaast is het ook mogelijk om navigatiebalken te maken. 
Aan zowel de werk- als navigatiebalken kunnen iconen worden toegevoegd.

\item \textbf{Knoppen}
Het is ook mogelijk om knoppen te plaatsen in het inhoud gedeelde. 
Ook hier is er terug een variëteit aan mogelijkheden: grote of kleine, met iconen of zonder, gegroepeerd of niet. 

\item \textbf{Inhoud vormgeven}
De inhoud van de pagina kan worden vormgegeven door gebruik te maken van een rooster. 
\jqm{} laat roosters tot vijf kolommen toe. 
Daarnaast zijn er ook nog opklapbare blokken ter beschikking. 
Als laatste kunnen deze blokken ook samengevoegd worden tot een accordeon. 

\item \textbf{Elementen voor formulieren}
\jqm{} biedt alle gangbare elementen voor formulieren aan zoals tekstinvoer, een selectie uit een lijst, een zoekveld, een \term{slider} en een \term{switch}. 
Het raamwerk verplicht zelfs om de \code{<label>}-tag te gebruiken. 
Zo wordt de applicatie toegankelijker gemaakt voor bijvoorbeeld mensen met een \term{e-reader}.

\item \textbf{Lijsten}
Een laatste categorie GI-elementen die \jqm{} aanbiedt, zijn lijsten. 
Deze gaan van standaard ongeordende lijsten tot lijsten met alle soorten decoraties als iconen, afbeeldingen, telbubbels en verdelers. 
Ook is het mogelijk om in deze lijsten te zoeken. 
Hiervoor dient de gebruiker enkel één data attribuut toe te voegen, waarna het raamwerk de implementatie voorziet. 
\end{enumerate}

\subsection{Niet-functionele kenmerken}
\paragraph{Performantie}
Zoals gezegd schrijft de ontwikkelaar HTML5-code met specifieke data-attributen en zal het raamwerk daarna de code verder aanvullen. 
Dit gebeurt enkel op de pagina die de gebruiker op dat moment bekijkt. 
Dit gaat dus ook op voor een webapplicatie waarbij alle schermen op één webpagina zijn geschreven. 
Deze webpagina bevat allemaal \code{<div>}-verpakkingen voor ieder scherm. 
\jqm{} zal enkel die \code{<div>} verder aanvullen die op dat moment getoond wordt aan de gebruiker. 

\paragraph{Aanpasbaarheid}
Als \jqm{} \term{out-of-the-box} wordt gebruikt, zit alles goed qua kleur en design. 
Er is keuze uit vijf kleurenthema's die kunnen worden toegepast op de gehele applicatie of enkel op bepaalde elementen. 
Om een applicatie echt te laten onderscheiden van de andere, is een eigen kleurthema noodzakelijk. 
Hier is \jqm{} op voorzien door hun \term{stylesheet} op te delen in twee delen: thema's en structuur. 
Een ontwikkelaar kan ook enkel de structuur downloaden en zelf het thema in CSS schrijven. 
Doordat dit laatste heel wat inspanning vraagt, hebben de ontwikkelaars van \jqm{} ook een tool ter beschikking gesteld, namelijk ThemeRoller~\cite{JQuery2012c}. 
Hiermee worden de kleuren naar een voorbeeldapplicatie gesleept, waarna de overeenkomstige \term{stylesheet} kan worden gedownload.

\paragraph{Programmeerbaarheid}
Bij het programmeren in \jqm{} wordt geen enkel ontwerppatroon afgedwongen. 
De code voor de GI-elementen wordt tenslotte als HTML5-code geschreven. 
Voor de echte functionaliteit wordt beroep gedaan op \js{} en meer bepaald op de jQuery Core bibliotheek. 
Ook deze dwingt geen ontwerppatroon af.

\paragraph{Browserondersteuning}
\label{sec:jqm-browser-support}

\jqm{} maakt gebruikt van \emph{progressive enhancement} (zie \ref{par:progressive-enhancement}).
Hierdoor wordt in principe ieder apparaat ondersteunt door het raamwerk, maar zal de applicatie op een ouder apparaat minder functionaliteit krijgen dan diezelfde applicatie op een nieuw apparaat.
Om dit te verduidelijken deelt \jqm{} browsers op in drie verschillende klassen: A, B en C~\cite{JQuery2012d}. 
Hierbij krijgt een klasse~A browser de volledige ervaring met AJAX-gebaseerde paginaovergangen.
Bij een browser van klasse~B wordt geen AJAX ondersteund.
Indien met een klasse~C browser heeft, wordt enkel een basiservaring aangeboden die nog steeds functioneel is.

%%%%%%%%%%%%%%%%%%%%%%%%%%%%%%%%%%%%%%%%%%%%%%%%%%%%%%%%%%%%%%%%%%
%%%%%%%%%%%%%%%%%%%%%%%%%%%%%%%%%%%%%%%%%%%%%%%%%%%%%%%%%%%%%%%%%%

\section{\lungo}
\label{sec:raamwerk-lungo}
\lungo{} is een opmaakgedreven raamwerk waarbij versie~1.0 uitkwam in 2011~\cite{TapQuo2011}.
Het raamwerk wordt onderhouden door TapQuo dat een Spaans bedrijf is, gespecialiseerd rond mobiele gebruikerservaring~\cite{TapQuo2013a}.
\lungo{} is afhankelijk van een \js{}-bibliotheek, namelijk \quo{}.
\lungo{} biedt vooral GI-elementen aan, maar daarnaast zijn er ook \term{wrappers} voor cache, opslag en SQL beschikbaar~\cite{TapQuo2013}.
Er wordt geen programmeerstijl zoals MVC afgedwongen.
Op het moment van schrijven zit \lungo{} aan versie~2.1~\cite{TapQuo2013}.

\subsection{Omkadering}
\paragraph{Programmeertaal}
Er is niets meer nodig dan kennis over HTML, CSS en \js{}.
De GI-elementen worden geschreven in HTML en aangeduid aan de hand van zowel CSS-klassen en \code{data-*}-attributen.
Functionele vereisten kunnen worden geschreven in \js{}.

\paragraph{Tools}
Er worden geen specifieke tools door TapQuo aangeboden om de ontwikkeling te versnellen.
Wel kan er gebruik worden gemaakt van Twitter Bower~\cite{Twitter2013} die helpt om de bestanden van het \lungo{}-raamwerk tezamen met de \quo{}-bibliotheek te beheren.
Dit komt door de aanwezigheid van het bestand \code{component.json} op \gh{} dat Bower gebruikt.

\paragraph{Documentatie}
De documentatiesite~\cite{Lungo2013} wordt eerst getoond hoe het geraamte van een typische \lungo{}-applicatie er uitziet.
Vervolgens zijn er nog acht andere pagina's die de aangeboden GI-elementen en API bondig uitleggen.
Op de documentatiepagina's is altijd eerst de broncode te zien.
Er kan dan boven de broncode op een knop geklikt worden, waarna een livevoorbeeld wordt getoond.
De documentatiesite~\cite{TapQuo2013c} van de onderliggende \js{}-bibliotheek bestaat uit één pagina met een zeer summiere samenvatting van de API.

\paragraph{Marktadoptatie}
Doordat geen informatie te vinden is op de site van \lungo{} werd een e-mail gestuurd met de vraag welke bedrijven \lungo{} gebruiken of in welke projecten van TapQuo zelf \lungo{} werd gebruikt.
Jammer genoeg hebben de auteurs tot op het moment van schrijven geen antwoord ontvangen.

\paragraph{Licenties}
Het raamwerk wordt onder de GPLv3-licentie vrijgegeven.
Daarnaast is een commerciële versie ook mogelijk.
Voor duiding omtrent deze laatste genoemde werd contact opgenomen met TapQuo via e-mail.
Jammer genoeg hebben de auteurs tot op het moment van schrijven geen antwoord ontvangen.

\subsection{Code en ontwikkeling}
Zoals gezegd wordt een \lungo{}-applicatie geprogrammeerd vanuit geannoteerde HTML-code door middel van CSS-klassen en \code{data-*}-attributen.
Er wordt geen programmeerstijl zoals MVC afgedwongen.

Om de verschillende schermen te scheiden wordt gebruik gemaakt van \code{<article>}- en \code{<section>}-tags die specifiek zijn voor HTML5.
Analoog wordt ook voor de kop- en voettekst de \code{<header>}- en \code{<footer>}-tags gebruikt.

Enerzijds kan de volledige applicatie op één webpagina worden geprogrammeerd.
Daarnaast biedt \lungo{} ook de mogelijkheid om de verschillende schermen op afzonderlijke pagina's op te slaan.
Hierbij dient enkel de code binnen de \code{<body>}-tag opgeslaan te worden.
Daarna wordt bij de initialisatie van \lungo{} gedefinieerd welke afzonderlijke pagina's asynchroon dienen te worden opgehaald.

\subsection{Functionele kenmerken}
De documentatiepagina is opgedeeld in negen groepen die hieronder kort worden aangehaald.

\begin{enumerate}

\item \textbf{Prototype } 
Hierop worden GI-elementen, navigatie, formulieren, scrollen, lijsten en data-attributen aangehaald. 
Opmerkelijk is de directe ondersteuning door het raamwerk bij het naar beneden trekken van een lijst waarbij de data automatisch wordt herladen.

\item \textbf{Core }
Hier worden functies aangehaald die het raamwerk zelf intern gebruikt.
Een voorbeeld hiervan is de \code{Lungo.Core.isMobile()}-functie die controleert of de applicatie in een mobiele omgeving wordt uitgevoerd.
Een ander voorbeeld is de aanwezig van een sorteer- en zoekfunctie voor arrays.

\item \textbf{Data }
Zoals aangehaald in de inleiding biedt het raamwerk \term{wrappers} voor cache, opslag en SQL.
De ontwikkelaar dient dus zelf geen \code{if-then-else}-constructies op te stellen of een bepaald kenmerk ondersteund wordt of niet, het raamwerk neemt dat voor zijn rekening.
Indien het apparaat het kenmerk niet ondersteund, voorziet \lungo{} een \term{fallback}.

\item \textbf{DOM }
Het manipuleren van de DOM gebeurt door de \quo{}-bibliotheek.
Hiervoor wordt gerefereerd naar de documentatie van \quo{}.
Zelf biedt \lungo{} enkel ondersteuning voor \term{events} bij het laden en ontladen van pagina's binnen de applicatie.

\item \textbf{Element }
Toevoegen van een telbubbel, vooruitgangsbalk, laadbalk, automatisch herladen bij het naar beneden trekken van lijst en een fotocarousel zijn programmeerbaar en manipuleerbaar via de \lungo{} API vanuit \js{}. 

\item \textbf{Notification }
Er worden standaard verschillende dialoogvensters aangeboden voor fout-, succes- en infomeldingen.
Deze dialoogvensters zijn voorzien van een specifieke lay-out.
Zo zal een fout- en succesboodschap respectievelijk een rode en groene achtergrondkleur hebben.
Een ander type venster dat wordt aangeboden is een bevestigingsvenster waarbij een vraag wordt gesteld aan de gebruiker waar ja of nee op geantwoord kan worden.
Het raamwerk voorziet voor beide acties de mogelijkheid om een eigen functie hieraan aan te haken.
 
\item \textbf{Routing }
Hier wordt uitgebreid ingegaan op de navigatie doorheen de applicatie vanuit \js{}.
Het is mogelijk om binnen eenzelfde artikel tussen verschillende secties te navigeren alsook van artikel te veranderen.
Daarnaast is het mogelijk om vanuit \js{} de zijnavigatie te tonen.
Als laatste voorziet \lungo{} een terugkeerfunctionaliteit, vergelijkbaar met de functie \code{history.back()} uit \js{}.

\item \textbf{Service }
Versturen van HTTP-oproepen kan zonder meer vanuit het \lungo{}.
Zowel HTTP GET als HTTP POST zijn mogelijk.
Het raamwerk kan omgaan met tekst, XML, JSON en HTML als antwoord.
Er is ook een verkorte notatie indien het antwoord JSON is.
Als laatste biedt \lungo{} de mogelijkheid om het antwoord van de HTTP-oproep te cachen, waarbij de maximale tijd moeten worden opgegeven.
Indien de oproep nog eens gebeurt en de maximale tijd is niet verstreken, zal \lungo{} geen HTTP-oproep versturen, maar het opgeslagen antwoord gebruiken.

\item \textbf{View }
Hier worden \js{}-methoden aangeboden om het uitzicht binnen een \code{<article>} te bepalen.
Zo kan de titel worden aangepast of van sectie binnen het artikel worden veranderd.
Daarnaast wordt ook ingegaan op het tonen en verbergen van de zijnaviatie.
 
\end{enumerate}

\subsection{Niet-functionele kenmerken}
\paragraph{Performantie}
De \js{}-bibliotheek waarop \lungo{} steunt is geoptimaliseerd voor mobiel gebruik.
Hierdoor bevat het geen methodes voor desktopgebruikers, waardoor het bestand kleiner is dan traditionele \js{}-bibliotheken.

\paragraph{Aanpasbaarheid}
\label{sec:lungo-aanpasbaarheid}
Standaard zit een \lungo{}-applicatie goed qua lay-out.
TapQuo biedt zelf geen tools aan om de kleuren of het uitzicht van de applicatie te veranderen.
Hierdoor zal een ontwikkelaar zelf aangewezen zijn om eigen CSS-code te schrijven.

\paragraph{Programmeerbaarheid}
\label{sec:lungo-programeerbaarheid}
\lungo{} dwingt geen enkel ontwerppatroon af.
Voor de echte functionaliteit wordt beroep gedaan op \js{}.
Daarbij is de ontwikkelaar vrij hoe hij te werk gaat en kan gebruik maken van de \lungo{} API. 
Voor aanpassingen die van toepassing zijn op DOM-manipulatie, is de ontwikkelaar aangewezen op \quo{}.
Ook deze dwingt geen ontwerppatroon af.

\paragraph{Ondersteuning browser}
TapQuo geeft aan op hun website dat ze ondersteuning bieden voor iOS, Android, Blackberry en FirefoxOS.
Verder halen ze aan dat ze een zelfde ervaring willen hebben voor een gemaakte applicatie gaande van mobiele apparaten, tv's tot desktopapparaten.

%%%%%%%%%%%%%%%%%%%%%%%%%%%%%%%%%%%%%%%%%%%%%%%%%%%%%%%%%%%%%%%%%%
%%%%%%%%%%%%%%%%%%%%%%%%%%%%%%%%%%%%%%%%%%%%%%%%%%%%%%%%%%%%%%%%%%

\section{Overzicht}
\label{sec:raamwerken-tabel}

In tabel~\ref{tabel:raamwerken-tabel} wordt de passieve vergelijking van de raamwerken getoond.
In het volgende hoofdstuk worden de vijf actieve vergelijkingscriteria besproken.
De resultaten hiervan kunnen niet worden bekomen door enkel naar de literatuur te kijken, maar moeten actief onderzocht worden.
De resultaten hiervan zullen in hoofdstuk \ref{chap:evaluatie} besproken worden.

\begin{landscape}
\begin{table}[H]
\centering
\pgfplotstabletypeset[
  begin table=\begin{tabular}{p{5cm} p{3.5cm} p{3.5cm} p{3.5cm} p{3.5cm}},
  end table=\end{tabular},
  col sep=comma,
  string type,
  header=true,
  skip coltypes=true,
  columns={Criterium,ST,Kendo,jQM,Lungo},
  columns/Criterium/.style={column name=\textbf{Criterium}},  
  columns/jQM/.style={column name=\textbf{\jqm}},
  columns/ST/.style={column name=\textbf{\st}},
  columns/Kendo/.style={column name=\textbf{\kendo}},
  columns/Lungo/.style={column name=\textbf{\lungo}},
  every head row/.style={
    before row=\toprule,
    after row=\midrule},
  every last row/.style={
    after row=\bottomrule}
]{tabellen/raamwerken.csv}
\caption{Passieve vergelijking van \st{}, \kendo{}, \jqm{} en \lungo{}}
\label{tabel:raamwerken-tabel}
\end{table}
\end{landscape}
