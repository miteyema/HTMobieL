\chapter{Evaluatie}
\label{chap:evaluatie}

In dit hoofdstuk voeren we de vergelijking uit en bekijken we de bekomen resultaten.
Enerzijds vergelijken we in \ref{sec:evaluatie-poc} de implementatie van de POC in de verschillende raamwerken.
Anderzijds vergelijken we in \ref{sec:evaluatie-criteria} de raamwerken op basis van de vergelijkscriteria opgesteld in .

%%%%%%%%%%%%%%%%%%%%%%%%%%%%%%%%%%%%%%%%%%%%%%%%%%%%%%%%%%%%%%%%%%%%%%%%

\section{POC}
\label{sec:evaluatie-poc}
In deze sectie bekijken we de problemen die we hebben tegengekomen bij het implementeren van de POC in de verschillende raamwerken.
Er wordt een onderscheid gemaakt tussen enerzijds de functionele vereisten als anderzijds vereisten met betrekking tot lay-out.

\subsection{Functionele vereisten}
Hieronder bespreken we de functionele vereisten van de POC.

\subsubsection{Formulieren}

\paragraph{jQuery Mobile} 
Voor het toevoegen van placeholders in de formuliervelden kon beroep worden gedaan op het \code{placeholder} attribuut in HTML5. Er dienden geen labels te worden gezet bij de velden. Deze labels zijn echter wel verplicht in jQuery Mobile, maar kunnen onzichtbaar worden gemaakt met de \code{ui-hide-label} CSS klasse~\cite{JQuery2013}.

Sommige velden waren verplicht in te vullen, terwijl andere niet. Hiervoor werd eerst gedacht om het \code{required} attribuut in HTML5 te gebruiken. Het probleem is echter dat er geen ondersteuning is voor mobiele browsers~\cite{Deveria2013}. Daarnaast was het ook nodig om de velden te valideren op hun waarde. Validatie is echter niet standaard aanwezig in jQuery Mobile. Als oplossing werd de plugin van Jörn Zaefferer gebruikt~\cite{Zaefferer2013}. Deze plugin loste ook het probleem met de verplichte velden op. Deze plugin kan op twee manieren gebruikt worden: enerzijds schrijven van CSS klassen in de HTML-code ofwel anderzijds door programmatie in de JavaScript-code. Beide aanpakken werden getest doorheen de POC. 
% TODO verder uitleggen hoe slim de plugin is

\subsubsection{Opladen van bewijs}

\paragraph{jQuery Mobile} 
Het opladen van een bestand kan gebeuren door \code{file} als invoertype van het formulierveld te gebruiken. Voor het kan worden doorgestuurd naar de backend, moet het bewijs eerst lokaal worden omgevormd naar base64. Dit werd geïmplementeerd met de FileReaderAPI en het canvas, wat beide HTML5 specificaties zijn. Het aangeklikte bestand wordt gelezen door middel van de FileReaderAPI, waarna het als afbeelding wordt opgeslagen en geïmporteerd wordt op het canvas. Eenmaal geïmporteerd, kan men de \code{.toDataURL()} oproepen op het canvas om de geïmporteerde afbeelding om te vormen naar base64. Deze aanpak werkt correct op recente mobiele apparaten. De FileReaderAPI wordt echter niet ondersteund op Android versies 2.3 en lager of iOS versies lager dan 6.0~\cite{Deveria2013a}.

\subsubsection{Handtekening}

\paragraph{jQuery Mobile} 
Er werd gezocht naar een plugin om deze functionaliteit te bekomen, doordat jQuery Mobile dit niet standaard aanbiedt. Eerst werd gewerkt met Signature Pad van Thomas Bradley~\cite{Bradley2013}. Door de lange tijd die werd besteed aan het aanpassen van layout, werd overgestapt naar jSignature van Willow Systems~\cite{Systems2013}. Deze laatste gaf ook het voordeel dat de breedte van het gebied om te handtekening in te zetten, zich automatisch naar 100\% schaalde. De plugin maakt gebruik van het HTML5 canvas element en de \code{.toDataURL()} methode, maar deze wordt niet ondersteund op Android versies 2.3 en lager~\cite{Systems2013}.

\subsubsection{AJAX}

\paragraph{jQuery Mobile} 
Het maken van oproepen via AJAX gebeurt via de jQuery bibliotheek waar jQuery Mobile op steunt. Dit gebeurt met de functie \code{\$.ajax}. 

\subsubsection{XML}

\subsubsection{JSON}

\subsubsection{PDF tonen}

\paragraph{jQuery Mobile} 
Het is niet aangeraden om ruwe data op te halen via AJAX. Hierdoor werd gebruik gemaakt van een verborgen formulier met de nodige parameters. Bij het klikken op lijstitem, wordt dit verborgen formulier opgestuurd naar de backend die dan een PDF teruggeeft.

\subsubsection{Automatische aanvulling}

\paragraph{jQuery Mobile} 
Doordat jQuery Mobile standaard geen automatische aanvulling ter beschikking heeft, werd de plugin van Andy Matthews gebruikt~\cite{Matthews2013}. Dit is een zeer gemakkelijk te integreren plugin die zowel met lokale data als data op afstand kan werken. Daarnaast dienden enkel vijf suggesties getoond te worden. Deze functionaliteit zat niet in de plugin, maar werd geïmplementeerd met de JavaScript \code{slice} functie.

\subsection{Lay-out}
Hieronder bespreken we de vereisten met betrekking tot lay-out.

\subsubsection{Tablet en smartphone}

\paragraph{jQuery Mobile} 
In jQuery Mobile is er standaard geen splitview aanwezig om een menu te tonen voor tablets, maar niet voor smartphones. Eerst werden hiervoor gezocht naar plugins aan de hand van~\cite{Deering2012}, wat leidde tot: Splitview~\cite{Rahman2013}, SimpleSplitView~\cite{Yared2013} en Multiview~\cite{Franck2012}. Deze drie mogelijke kanshebbers hadden elk hun tekorten. Zo was de eerste destructief ten opzichte van het raamwerk. Dit betekent dat de bestanden van jQuery Mobile zelf werden aangepast, wat het moeilijker zou maken als men wil updaten naar een nieuwe versie. De tweede plugin werkte enkel tot versie 1.0.1 van jQuery Mobile. De laatste plugin had moeite met het zich  aanpassen aan veranderde afmetingen van de browser. 

Uiteindelijk werd van een plugin afgestapt door \cite{Hadlock2012} waarbij werd aangetoond hoe men via CSS3 media queries hetzelfde kan bereiken. Daarnaast gebruikt de documentatie van jQuery Mobile~\cite{JQuery2012b} een gelijkaardige layout. De uiteindelijke oplossing voor het probleem kwam uit te combinatie van deze twee voorgaande oplossingen.

%%%%%%%%%%%%%%%%%%%%%%%%%%%%%%%%%%%%%%%%%%%%%%%%%%%%%%%%%%%%%%%%%%%%%%%%

\section{Vergelijkscriteria}
\label{sec:evaluatie-criteria}