\chapter{Vergelijking}
\label{chap:vergelijking}

In dit hoofdstuk bekijken we hoe we de mobiele HTML5 raamwerken vergelijken.
Hoofdzakelijk zullen we dit doen aan de hand van een \term{proof of concept} (POC).
Deze wordt geïntroduceerd in sectie \ref{sec:vergelijking-poc} en zal hoofdzakelijk de gekozen vergelijkingscriteria in sectie \ref{sec:vergelijking-criteria} drijven.


\section{POC} % TIM (iets van 2 bladzijden)
\label{sec:vergelijking-poc}
In samenspraak met Capgemini werd gekozen voor een applicatie die het mogelijk maakt voor werknemers om hun onkosten via hun mobiel apparaat door te sturen.
Een werknemer kan een foto nemen met de camera 

\section{Vergelijkingscriteria} % SANDER (iets van 2x de bloglengte)
\label{sec:vergelijking-criteria}

In deze sectie zullen we de criteria toelichten die we zullen toepassen om raamwerken te vergelijken.
In sectie \ref{sec:vergelijken-raamwerken} werden reeds technieken besproken die in de literatuur worden toegepast.
Elementen van deze technieken zullen terugkomen in onze methode om de raamwerken te evalueren.

Vijf grote criteria zullen gebruikt worden:  omkadering(\ref{sec:vergelijking-gemeenschap}), productiviteit(\ref{sec:vergelijking-gemeenschap}), gebruik(\ref{sec:vergelijking-gemeenschap}), ondersteuning(\ref{sec:vergelijking-gemeenschap}) en performantie(\ref{sec:vergelijking-gemeenschap}). Elk raamwerk krijgt voor elk criterum een score. 
Als besluit zullen we een \term{spiderweb} van de criteria creëren waar de scores van de raamwerken opstaan. 
Zoals hierboven vermeld zal een POC gebruikt worden bij de vergelijking.
De implementatie van deze POC zal het productiviteit-, gebruik- en ondersteuningcriterium drijven.  
De performantie zal deels bepaald worden door de laadtijden van de POC.

%TODO spiderweb

\subsection{Gemeenschap}
\label{sec:vergelijking-gemeenschap}
De gemeenschap en populariteit van een raamwerk is makkelijk in cijfers uit te drukken. 
We voorzien een tabel waar we per raamwerk het aantal volgers op Twitter, watchers/forkers van GitHub en aantal likes van Facebook zullen onderbrengen~\cite{Sarrafi2012a,Ayuso2012}. 
Een grafiek van Google Trends, die het zoekvolume op Google uitzet per tijd, zal voor elk raamwerk na deze tabel worden toegevoegd.
%TODO zoals bij onze presentatie gezegd:  ook cijfers per tijd bekijken?
\paragraph{}
De som van Twitter volgers($T_i$), GitHub watchers($G_i$) en Facebook likes($F_i$) vormt de score voor het gemeenschapscriteria:
\begin{equation}
  G_f={T_f+G_f+F_f}
  \label{eq:gemeenschap}
\end{equation}.

\subsection{Productiviteit}
\label{sec:vergelijking-gemeenschap}
De productiviteit moet een indicatie geven hoe lang het duurt om met het raamwerk vertrouwd te raken. 
Hiervoor gaan we het feit uitbuiten dat we deze thesis met twee maken. 
Elk zullen we de POC in twee verschillende raamwerken maken wat het totaal op 4 raamwerken brengt. 
De tijd die nodig is om de volledige POC te implementeren is een indicatie voor de productiviteit. 
Ook zal elk van ons een loginscherm maken in de andere twee raamwerken~\cite{Burris}. 
Als Tim de POC implementeert in raamwerken A en B en Sander in raamwerken C en D, zal dus Tim ook een loginscherm maken in raamwerken C en D en vice versa.  
Dit scherm bevat UI-elementen, validaties en backend integratie. 
We kunnen dit dus als voldoende steekproef beschouwen om ervaring met een raamwerk te testen. 
De tijd die ieder nodig heeft om dit scherm te bouwen, geeft ook een indicatie van de nodige leertijd.
\paragraph{}
De som van de uren voor het implementeren van de POC($t_{POC}$) en het loginscherm($t_{login}$) vormt de score voor de productiviteit:
\begin{equation}
  Pr_f = {t_{POC} + t_{login}}
  \label{eq:productiviteit}
\end{equation}.

\subsection{Gebruik}
\label{sec:vergelijking-gemeenschap}
Om het verschil in gebruik van de raamwerken te onderzoeken, zullen we steunen op de logboeken die we hebben bijgehouden tijdens het implementeren van de POC. 
We gaan er vanuit dat de POC relevante kenmerken bevat om de raamwerken zoveel mogelijk uit te buiten. 
We zullen bij het vergelijken de moeilijkheden groeperen zodat het snel duidelijk wordt welke functionaliteit in een bepaald raamwerk moeilijker/makkelijker te implementeren is. 
\paragraph{}
Een hogere score wordt toegekend wanneer bepaalde functionaliteit reeds aangeboden wordt door het raamwerk. 
Een lagere score betekent dat een plugin werd gezocht. 
Wanneer een hack noodzakelijk was om de functionaliteit te beogen, zal de laagste score worden toegekend.
\begin{table}[h]	
  \centering
  \begin{tabular}{ll}
    \toprule
    \textbf{Score} & \textbf{Assessment criteria}\\
    \midrule
    2 & Ondersteund door het raamwerkk\\
    1 & Een plugin is nodig\\
    0 & Eigen implementatie of hack\\
    \bottomrule
  \end{tabular}
  \caption{Beoordeling uitdagingen gebruikcriterium}
  \label{tabel:scores-uitdagingen}
\end{table}
\begin{equation}
  G_f = \sum_{i=1}^{13}{\left(S_j\right)}
  \label{eq:gebruik}
\end{equation}
waar $i$ het aantal uitdagingen voorstelt.

\subsection{Ondersteuning}
\label{sec:vergelijking-gemeenschap}
We zullen een scenario uitwerken waarbij we het gebruik van de POC doorlopen. 
Dit scenario zullen we op een reeks van verschillende mobiele apparaten herhalen~\cite{Sarrafi2012a}. 
Bij het mislukken van een stap in het scenario, zal een punt worden afgetrokken. 
Hierdoor kunnen we een score bekomen hoe goed de applicatie, geïmplementeerd in een bepaald raamwerk, scoort op een bepaald apparaat. 
De gebruikte apparaten en versies van besturingssystemen zullen worden bepaald aan de hand van de marktaandeel en de beschikbaarheid van deze apparaten aan het departement HCI.

\begin{table}[h]
  \centering
  \begin{tabular}{llll}
    \toprule
    \textbf{Toestel} & \textbf{Besturingssysteem} & \textbf{Browser} & \textbf{Gewicht}\\
    \midrule
    HTCDesireZ & Android 2.3.3 &  & \\
    GalaxyTab & Android 2.3.6 &  & \\
    Galaxy Gio & Android 2.3.6 &  & \\
    Galaxy Ace 2 & Android 2.3.6 &  & \\
    GalaxySII & Android 4.1.2 &  & \\
    Nexus 7 & Android 4.2.1  &  & \\
    iPad3 4GWiFi & iOS 6.0.1 &  & \\
    iPhone 3GS & iOS 6.0.1 &  & \\
    \bottomrule
  \end{tabular}
  \caption{Beschikbare toestellen met hun besturingssysteem, browser en gewicht}
  \label{tabel:toestellen-hci}
\end{table}

\paragraph{}
\begin{equation}
  O_f = \sum_{d=1}^{N}{\left(\sum_{i=1}^{13}S_j*W_d\right)}
  \label{eq:ondersteuning}
\end{equation}
waar $d$ het aantal toestellen voorstelt en $i$ het aatal uitdagingen. 
\paragraph{}
Een kanttekening hierbij is of het raamwerk ondersteunt de webapplicatie om te vormen tot een native applicatie. 
Ook bekijken we of de \term{native look-and-feel} per besturingssysteem door het raamwerk kan worden benaderd.

\subsection{Performantie}
\label{sec:vergelijking-gemeenschap}
Een laatste criterium is de performantie. 
Dit zal worden uitgedrukt in de tijd nodig om te applicatie te renderen. 
Hierbij zullen we verschillende testen doen. 
Ten eerste zullen we gebruik maken van de volledige POC($t_{POC}$). 
We zullen kijken naar de rendertijd van de applicatie aan de hand van Google Page Speed~\cite{Morgan2011}. 
Daarnaast zullen we een geïsoleerde test doen met het verzenden van een AJAX request, omdat sommige raamwerken altijd eerst een OPTION request sturen($t_{AJAX}$). Verdergaand op het versturen van requests zullen we kijken hoe lang het duurt vooraleer een expense daadwerkelijk verzonden is naar de backend. Vervolgens zullen we ook geïsoleerde testen doen met UI-elementen. 
Hierbij zullen we een pagina voorzien van 1000 buttons en meten hoe lang het duurt tot deze pagina gerendert wordt. 
Als laatste zullen we een test op de aparte loginschermen uitvoeren($t_{login}$). 
Hier kunnen we dan ook het aantal lijnen code bekijken en de bijhorende performantie. 
De redenen waarom we ook deze geïsoleerde applicatie gebruiken is omdat we ervan uitgaan dat niet de volledige POC in ieder raamwerk zal kunnen worden geïmplementeerd. 
Dit is in tegenstelling tot de geïsoleerde login applicatie, waarbij we exact kunnen vergelijken.
De formule voor het performantiecriterium is 
\paragraph{}
\begin{equation}
  Pe_f=t_{POC}+t_{AJAX}+t_{login} 
  \label{eq:performantie}
\end{equation}.
