\chapter{Evaluatie}
\label{chap:evaluatie}

In dit hoofdstuk wordt de vergelijking uitgevoerd op basis van de vijf vergelijkingscriteria uit hoofdstuk \ref{chap:vergelijkingscriteria}, namelijk gemeenschap~(\ref{sec:evaluatie-gemeenschap}), productiviteit~(\ref{sec:evaluatie-productiviteit}), gebruik~(\ref{sec:evaluatie-gebruik}), ondersteuning~(\ref{sec:evaluatie-ondersteuning}) en performantie~(\ref{sec:evaluatie-performantie}). 

%%%%%%%%%%%%%%%%%%%%%%%%%%%%%%%%%%%%%%%%%%%%%%%%%%%%%%%%%%%%%%%%%%%%%%%%

\section{Gemeenschap}
\label{sec:evaluatie-gemeenschap}

De gemeenschap van de vier raamwerken wordt samengevat in tabel~\ref{tabel:evaluatie-gemeenschap}. 
Daarna zal per raamwerk de gemeenschap worden besproken.

\begin{table}[H]
\centering
\pgfplotstabletypeset[
  begin table=\begin{tabular}{p{8cm} p{1cm} p{1cm} p{1cm} p{1cm}},
  end table=\end{tabular},
  skip coltypes=true,
  col sep=comma,
  string type,
  header=true,
  columns={Gemeenschap,jQM,ST,Kendo,Lungo},
  columns/Gemeenschap/.style={column name=\textbf{Gemeenschap}, column type={l}},  
  columns/jQM/.style={column name=\textbf{\jqma}, column type={c}},
  columns/ST/.style={column name=\textbf{\sta}, column type={c}},
  columns/Lungo/.style={column name=\textbf{\lungoa}, column type={c}},
  columns/Kendo/.style={column name=\textbf{\kendoa}, column type={c}},
  every head row/.style={
    before row=\toprule,
    after row=\midrule},
  every last row/.style={
  	before row=\midrule,
    after row=\bottomrule}
]{tabellen/gemeenschap.csv}
\caption{Samenvattende tabel voor gemeenschapscriterium}
\label{tabel:evaluatie-gemeenschap}
\end{table}

% \paragraph{\jqm}
% % Met 7.400 volgers op GitHub~\cite{GitHub2012} en 11.200 volgers op Twitter~\cite{Twitter2012} komt de grote kracht van jQuery Mobile van zijn community . Dit heeft grotendeels te maken met het feit dat jQuery Mobile geniet van het succes van jQuery, dat ook zeer populair is~\cite{Hales2012}.
% %TODO Sander: dit komt in de vergelijkingscriteria
% 
% \paragraph{\st}
% % We kunnen vaststellen dat Sencha over een grote community beschikt.  Met meer dan 2 miljoen ontwikkelaars wereldwijd is Sencha de grootste provider van een open-source web applicatie~\cite{Inc.}.  
% 
% \paragraph{\kendo}
% 
% \paragraph{\lungo}

