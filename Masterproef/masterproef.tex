\documentclass[master=cws,dutch]{kulemt}
\setup{title={Vergelijkende studie van frameworks voor de ontwikkeling van mobiele HTML5 applicaties},
  author={Tim Ameye\and Sander Van Loock},
  promotor={Prof.\,dr.\,ir.\ E. Duval},
  assessor={Ir.\,W. Eetveel\and W. Eetrest},
  assistant={Ir.\ A.~Assistent \and D.~Vriend}}
% De volgende \setup mag verwijderd worden als geen fiche gewenst is.
\setup{filingcard,
  translatedtitle={The best master thesis ever},
  udc=621.3,
  shortabstract={Hier komt een heel bondig abstract van hooguit 500
    woorden. \LaTeX\ commando's mogen hier gebruikt worden. Blanco lijnen
    (of het commando \texttt{\string\pa r}) zijn wel niet toegelaten!
    \endgraf \lipsum[2]}}
% Verwijder de "%" op de volgende lijn als je de kaft wil afdrukken
%\setup{coverpageonly}
% Verwijder de "%" op de volgende lijn als je enkel de eerste pagina's wil
% afdrukken en de rest bv. via Word aanmaken.
%\setup{frontpagesonly}

% Kies de fonts voor de gewone tekst, bv. Latin Modern
\setup{font=lm}

% Hier kun je dan nog andere pakketten laden of eigen definities voorzien
\usepackage{kulemtx}
\headstyles{kulemtman}
\kulemtmanToC

% Tenslotte wordt hyperref gebruikt voor pdf bestanden.
% Dit mag verwijderd worden voor de af te drukken versie.
\usepackage[pdfusetitle,colorlinks,plainpages=false]{hyperref}

%%%%%%%
% Om wat tekst te genereren wordt hier het lipsum pakket gebruikt.
% Bij een echte masterproef heb je dit natuurlijk nooit nodig!
\IfFileExists{lipsum.sty}%
 {\usepackage{lipsum}\setlipsumdefault{11-13}}%
 {\newcommand{\lipsum}[1][11-13]{\par Hier komt wat tekst: lipsum ##1.\par}}
%%%%%%%

%\includeonly{hfdst-n}
\begin{document}

\begin{preface}
  Dit is mijn dankwoord om iedereen te danken die mij bezig gehouden heeft.
  Hierbij dank ik mijn promotor, mijn begeleider en de voltallige jury.
  Ook mijn familie heeft mij erg gesteund natuurlijk.
\end{preface}

\tableofcontents*

\begin{abstract}
  In dit \texttt{abstract} environment wordt een al dan niet uitgebreide
  samenvatting van het werk gegeven. De bedoeling is wel dat dit tot
  1~bladzijde beperkt blijft.

  \lipsum[1]
\end{abstract}

% Een lijst van figuren en tabellen is optioneel
%\listoffigures
%\listoftables
% Bij een beperkt aantal figuren en tabellen gebruik je liever het volgende:
\listoffiguresandtables
% De lijst van symbolen is eveneens optioneel.
% Deze lijst moet wel manueel aangemaakt worden, bv. als volgt:
\chapter{Lijst van afkortingen en symbolen}
\section*{Afkortingen}
\begin{flushleft}
  \renewcommand{\arraystretch}{1.1}
  \begin{tabularx}{\textwidth}{@{}p{12mm}X@{}}
    LoG   & Laplacian-of-Gaussian \\
    MSE   & Mean Square error \\
    PSNR  & Peak Signal-to-Noise ratio \\
  \end{tabularx}
\end{flushleft}
\section*{Symbolen}
\begin{flushleft}
  \renewcommand{\arraystretch}{1.1}
  \begin{tabularx}{\textwidth}{@{}p{12mm}X@{}}
    42    & ``The Answer to the Ultimate Question of Life, the Universe,
            and Everything'' volgens de \cite{h2g2} \\
    $c$   & Lichtsnelheid \\
    $E$   & Energie \\
    $m$   & Massa \\
    $\pi$ & Het getal pi \\
  \end{tabularx}
\end{flushleft}

% Nu begint de eigenlijke tekst
\mainmatter

\chapter{Inleiding} %3 blz (0.75 per sectie)
\label{inleiding}
%In dit hoofdstuk wordt het werk ingeleid. Het doel wordt gedefinieerd en er wordt uitgelegd wat de te volgen weg is (beter bekend als de rode draad).

%\section{Achtergrondinformatie} %tim
% - web apps (cross platform)
% - HTML5/js/css3
% - wat doet framework?
% - om de ontwikkeling vergemakkelijk worden frameworks aangeboden... 

Het gebruik van smartphones en tablets stijgt ontzettend snel in onze samenleving.
Waar voorheen met de gsm enkel kon worden gebruik gemaakt van de voorgeïnstalleerde applicaties op het apparaat, kunnen op deze mobiele apparaten ook extra applicaties vanuit een winkel worden gedownload en geïnstalleerd.
Ontwikkelaars van deze mobiele applicaties worden geconfronteerd met de variëteit aan mobiele besturingssystemen die op deze apparaten aanwezig zijn.
% Er bestaat niet enkel verschillende besturingssystemen, maar ook verschillende versies van besturingssystemen.
Dit komt doordat een applicatie dient te worden geprogrammeerd in een programmeertaal die specifiek is tot het besturingssysteem.
Ontwikkelaars zullen dus eenzelfde applicatie in verschillende programmeertalen dienen te programmeren om een zo groot mogelijk publiek te bereiken.
Niet enkel het programmeren, maar ook het onderhoud van de applicaties in verschillende programmeertalen brengt een grote kost met zich mee.

Een oplossing hiervoor is het maken van een mobiele webapplicatie, gebruikmakend van HTML5.
Ten eerste wordt deze rechtstreeks in een webbrowser geopend en moet dus niet langer worden geïnstalleerd vanuit een winkel.
Dit betekent dus dat ieder mobiel apparaat die een webbrowser heeft, de webapplicatie kan openen ongeacht zijn mobiel besturingssysteem.
Ten tweede wordt de applicatie slechts in één programmeertaal geschreven, wat de kost omlaag brengt.
Om het ontwikkelingsproces van deze mobiele HTML5-applicaties te versnellen worden raamwerken aangeboden die helpen bij de functionaliteit van de applicatie en de elementen voor de gebruikersinterface. 

\section{Probleembeschrijving} %tim
% - heel veel frameworks, nog geen literatuur die vergelijkt (alleen blogs e.d.)
% - eerder voorstelling ipv objectieve verglijking.
% - beste framework?
% - hoe vergelijken?
Mobiele HTML5-raamwerken schieten als paddenstoelen uit de grond en ook de verschillende versies van eenzelfde raamwerk volgen elkaar in snel op.
In de huidige literatuur worden er vaak raamwerken aangehaald en besproken, maar niet vergeleken.
Indien deze toch worden vergeleken, gebeurt dit vaak subjectief of worden punten gegeven zonder een gestaafde methode te gebruiken.
Geen enkele literatuur bouwt verder op reeds bestaande literatuur van vergelijkingen voor HTML5-raamwerken of aggregeert de losstaande vergelijkingen tot één geheel.

\section{Doelstellingen} %sander
% Is er een beste framework?
% contribuite:  methodologie uitwerken om OBJECTIEF en VISUEEL raamwerken te vergelijken
Deze thesistekst bestaat uit twee grote doelstellingen die beide in de titel verborgen zitten.
Een eerste doel is het creëren van een methodologie om HTML5-raamwerken met elkaar te vergelijken.
Deze methodologie moet alle belangrijke aspecten van de raamwerken tegen het licht houden.
Ook moet er geprobeerd worden de werkwijze zo objectief mogelijk te laten verlopen en het resultaat van de studie op een eenvoudige,  visuele manier aan de lezer te presenteren.
Het tweede doel omdat de effectieve vergelijking van de raamwerken zelf.
Door de grote verscheidenheid van HTML5-raamwerken moeten de bestudeerde raamwerken zo worden gekozen dat ze zoveel mogelijk aspecten van alle raamwerken bevatten.
Hier komt ook de afweging tussen het aantal bestudeerde raamwerken en diepte van de vergelijkende studie de kop op steken.
De raamwerken die worden gekozen moeten vervolgens worden vergeleken met de vooropgestelde methodologie.
Het resultaat moet alle positieve en negatieve aspecten van de raamwerken bevatten.
Vervolgens moeten er gekeken worden of er één raamwerk het best is of er verschillende raamwerken in verschillende situaties als best kunnen worden bestempeld.

\section{Toepassingsgebied} %sander
% mobiele wereld (mobile = booming)
% web (web = booming)
% kruising tussen web en mobile = super booming!
% bedrijfswereld (capgemini) HTML5 iets nieuws,  bedrijven kunnen nu een met een gerust hart een goede keuze maken (mss beter doelstellingen)
HTML5-raamwerken vergemakkelijken de ontwikkeling van HTML5-applicaties.
Deze applicaties zijn toepasbaar op twee domeinen: web en mobiel.
Ze zijn toegankelijk via het web en geoptimaliseerd om op mobiele apparaten te kunnen werken.
Het aanspreken van mobiele applicaties via het web heeft zowel voor- als nadelen.
Zeker wanneer er wordt vergeleken met \term{native} of hybriede applicaties.
De focus van deze studie ligt echter niet bij het onderzoeken van deze voor- of nadelen!
Wel zullen de verschillende technologieën besproken worden om mobiele applicaties te maken.

Omdat Capgemini deze thesis mee ondersteunt zullen beide domeinen vanuit een bedrijfscontext worden benaderd.
Dit zal vooral naar boven komen in de keuze van vergelijkingscriteria en de methode om deze criteria te testen.


%\section{Rode draad} %tim
% - geen nadruk op passief actief
% - passieve en actieve vergelijkingscriteria al hier vermelden.
Eerst wordt in hoofdstuk~\ref{chap:literatuurstudie} de basis van dit werk uitgelegd.
Vervolgens wordt er in hoofdstuk~\ref{chap:raamwerken} een passieve vergelijking gedaan van de gekozen raamwerken.
Daarna zullen in hoofdstuk~\ref{chap:vergelijkingscriteria} de gekozen vergelijkingscriteria voor de actieve vergelijking aan bod komen en verantwoord worden.
Hieropvolgend wordt in hoofdstuk~\ref{chap:evaluatie} deze actieve vergelijking uitgevoerd op de gekozen raamwerken aan de hand van de gekozen criteria.
Als laatste wordt dit alles besloten in hoofdstuk~\ref{chap:besluit}.

%%% Local Variables: 
%%% mode: latex
%%% TeX-master: "masterproef"
%%% End: 

\chapter{Het eerste hoofdstuk}
\label{hoofdstuk:1}
Een hoofdstuk behandelt een samenhangend geheel dat min of meer op zichzelf
staat. Het is dan ook logisch dat het begint met een inleiding, namelijk
het gedeelte van de tekst dat je nu aan het lezen bent.

\section{Eerste onderwerp in dit hoofdstuk}
De inleidende informatie van dit onderwerp.

\subsection{Een item}
De bijbehorende tekst. Denk eraan om de paragrafen lang genoeg te maken en
de zinnen niet te lang.

Een paragraaf omvat een gedachtengang en bevat dus steeds een paar zinnen.
Een paragraaf die maar \'e\'en lijn lang is, is dus uit den boze.

\section{Tweede onderwerp in dit hoofdstuk}
Er zijn in een hoofdstuk verschillende onderwerpen. We zullen nu
veronderstellen dat dit het laatste onderwerp is.

\subsection{Een item}
Maak ook geen misbruik van opsommingen. Voor korte opsommingen gebruik je
geen ``\verb|itemize|'' of ``\texttt{enumerate}'' commando's. Doe dus
\emph{niet} het volgende:
\begin{quote}
  De Eiffeltoren bevat drie verdiepingen:
  \begin{itemize}
  \item de eerste;
  \item de tweede;
  \item de derde.
  \end{itemize}
\end{quote}
Maar doe:
\begin{quote}
  De Eiffeltoren bevat drie verdiepingen: de eerste, de tweede en de derde.
\end{quote}

\section{Besluit van dit hoofdstuk}
Als je in dit hoofdstuk tot belangrijke resultaten of besluiten gekomen
bent, dan is het ook logisch om het hoofdstuk af te ronden met een
overzicht ervan. Voor hoofdstukken zoals de inleiding en het
literatuuroverzicht is dit niet strikt nodig.

%%% Local Variables: 
%%% mode: latex
%%% TeX-master: "masterproef"
%%% End: 

\chapter{Een volgend hoofdstuk}
\label{hoofdstuk:2}
Een hoofdstuk behandelt een samenhangend geheel dat min of meer op zichzelf
staat. Het is dan ook logisch dat het begint met een inleiding, namelijk
het gedeelte van de tekst dat je nu aan het lezen bent.

\section{Eerste onderwerp in dit hoofdstuk}
De inleidende informatie van dit onderwerp.

\subsection{Een item}
Een tekst staat nooit alleen. Dit wil zeggen dat er zeker ook referenties
nodig zijn. Dit kan zowel naar on-line documenten\cite{wiki} als naar
boeken\cite{pratchett06:_good_omens}.

\section{Figuren}
Figuren worden gebruikt om illustraties toe te voegen. Dit is dan ook de
manier om beeldmateriaal toe te voegen zoals getoond wordt in
figuur~\ref{fig:logo}.

\begin{figure}
  \centering
  \includegraphics{logokul}
  \caption{Het KU~Leuven logo.}
  \label{fig:logo}
\end{figure}

\section{Tabellen}
Tabellen kunnen gebruikt worden om informatie op een overzichtelijke te
groeperen. Een tabel is echter geen rekenblad! Vergelijk maar eens
tabel~\ref{tab:verkeerd} en tabel~\ref{tab:juist}. Welke tabel vind jij het
duidelijkst?

\begin{table}
  \centering
  \begin{tabular}{||l|lr||} \hline
    gnats     & gram      & \$13.65 \\ \cline{2-3}
              & each      & .01 \\ \hline
    gnu       & stuffed   & 92.50 \\ \cline{1-1} \cline{3-3}
    emu       &           & 33.33 \\ \hline
    armadillo & frozen    & 8.99 \\ \hline
  \end{tabular}
  \caption{Een tabel zoals het niet moet.}
  \label{tab:verkeerd}
\end{table}

\begin{table}
  \centering
  \begin{tabular}{@{}llr@{}} \toprule
    \multicolumn{2}{c}{Item} \\ \cmidrule(r){1-2}
    Animal    & Description & Price (\$)\\ \midrule
    Gnat      & per gram    & 13.65 \\
              & each        & 0.01 \\
    Gnu       & stuffed     & 92.50 \\
    Emu       & stuffed     & 33.33 \\
    Armadillo & frozen      & 8.99 \\ \bottomrule
  \end{tabular}
  \caption{Een tabel zoals het beter is.}
  \label{tab:juist}
\end{table}

\section{Lorem ipsum}
Tenslotte gaan we hier nog wat tekst voorzien zodat er minstens een
bijkomende bladzijde aangemaakt wordt. Dat geeft de gelegenheid om eens te
zien hoe de koptekst en de voettekst zich gedragen.

\subsection{Lorem ipsum dolor sit amet, consectetur adipiscing elit}
Sed nec tortor id felis tristique sodales. Nulla nec massa eu dui fermentum
tincidunt. Integer ullamcorper ante eget eros posuere faucibus. Nam id
ligula ut augue pulvinar vulputate id at purus. Aenean condimentum tortor
eu mi placerat eget eleifend massa mollis. Nam est mi, sagittis quis
euismod eget, sagittis in nibh. Proin elit turpis, aliquam et imperdiet
sed, volutpat eu turpis.

Pellentesque vel enim tellus, vitae egestas turpis. Praesent malesuada elit
non nisi sollicitudin non blandit lacus tincidunt. Morbi blandit urna at
lectus ornare laoreet. Suspendisse turpis diam, lobortis dictum luctus
quis, commodo at lorem. Integer lacinia convallis ultricies. Sed quis augue
neque, eu malesuada arcu. Nullam vehicula, purus vitae sagittis pulvinar,
erat eros semper massa, eu egestas nibh erat quis magna. Cras pellentesque,
nisl eu dapibus volutpat, urna augue ornare quam, quis egestas lectus nulla
a lectus.

Vivamus dictum libero in massa cursus sed vulputate eros imperdiet. Donec
lacinia, libero ac lobortis egestas, nibh dui ornare arcu, luctus porttitor
velit massa sit amet quam. Maecenas scelerisque laoreet diam, vitae congue
quam adipiscing vitae. Aliquam cursus nisl a leo convallis eleifend
fermentum massa porta. Nunc libero quam, dapibus dapibus molestie sit amet,
faucibus vel nunc.

\subsection{Praesent auctor venenatis posuere}
Sed tellus augue, molestie in pulvinar lacinia, dapibus non ipsum. Fusce
vitae mi vitae enim ullamcorper hendrerit eu malesuada est. Proin iaculis
ante sed nibh tincidunt vel interdum libero posuere. Vivamus accumsan metus
quis felis congue suscipit dapibus enim mattis. Fusce mattis tortor eget
ipsum interdum sagittis auctor id metus.

Integer diam lacus, pharetra sit amet tempor et, tristique non lorem.
Aenean auctor, nisi eu interdum fermentum, lectus massa adipiscing elit,
sed facilisis orci odio a lectus. Proin mi nibh, tempus quis porta a,
viverra quis enim. In sollicitudin egestas libero, quis viverra velit
molestie eget. Nulla rhoncus, dolor a mollis vestibulum, lacus elit semper
nisi, nec sollicitudin sem urna eu magna. Nunc sed est urna, euismod congue
mi.

\subsection{Cras vulputate ultricies venenatis}
Vivamus eros urna, sodales accumsan semper vel, lobortis sit amet mauris.
Etiam condimentum eleifend lorem, ullamcorper ornare lectus aliquet vitae.
Praesent massa enim, interdum sit amet semper et, venenatis ut elit.
Quisque faucibus, quam ac lacinia imperdiet, nulla neque elementum purus,
tempus rutrum justo massa porta sapien. Vestibulum ante ipsum primis in
faucibus orci luctus et ultrices posuere cubilia Curae; Sed ultrices
interdum mi, et rhoncus sapien rutrum sed.

Duis elit orci, molestie quis sollicitudin sed, convallis non ante.
Maecenas tincidunt condimentum justo, et ultricies leo tristique vitae.
Vestibulum quis quam non lectus dapibus eleifend a vitae nibh. Nam nibh
justo, pharetra quis iaculis consequat, elementum quis justo. Etiam mollis
lacinia lacus, nec sollicitudin urna lobortis ac. Nulla facilisi.

Proin placerat risus eleifend erat ultricies placerat. Etiam rutrum magna
nec turpis euismod consectetur. Phasellus tortor odio, lacinia imperdiet
condimentum sed, faucibus commodo erat. Phasellus sed felis id ante
placerat ultrices. Aenean tempor justo in tortor volutpat eu auctor dolor
mollis. Aenean sit amet risus urna. Morbi viverra vehicula cursus.

\subsection{Donec nibh ante, consectetur et posuere id, tempus nec arcu}
Curabitur a tellus aliquet ipsum pellentesque scelerisque. Etiam congue,
risus et volutpat rutrum, est purus dapibus leo, non cursus metus felis
eget ligula. Vivamus facilisis tristique turpis, ut pretium lectus luctus
eleifend. Fusce magna sapien, ullamcorper vitae fringilla id, euismod quis
ante.

Phasellus volutpat, nunc et pharetra semper, sem justo adipiscing mauris,
id blandit magna quam et orci. Vestibulum a erat purus, ut molestie ante.
Vestibulum ante ipsum primis in faucibus orci luctus et ultrices posuere
cubilia Curae; Proin turpis diam, consequat ut ullamcorper ut, consequat eu
orci. Sed metus risus, fringilla nec interdum vel, interdum eu nunc.
Suspendisse vel sapien orci.

\subsection{Morbi et mauris tempus purus ornare vehicula}
Mauris sit amet diam quam, eget luctus purus. Sed faucibus, risus semper
eleifend iaculis, mi turpis bibendum nisl, quis cursus nibh nisl sit amet
ipsum. Vestibulum tempor urna vitae mi auctor malesuada eget non ligula.
Nullam convallis, diam vel ultrices auctor, eros eros egestas elit, sed
accumsan arcu tortor eget leo. Vestibulum orci purus, porttitor in pharetra
eget, tincidunt eget nisl. Nullam sit amet nulla dui, facilisis vestibulum
dui.

Donec faucibus facilisis mauris ac cursus. Duis rhoncus quam sed nisi
laoreet eu scelerisque massa tincidunt. Vivamus sit amet libero nec arcu
imperdiet tempor quis non libero. Sed consequat dignissim justo. Phasellus
ullamcorper, velit quis posuere vulputate, felis erat tincidunt mauris, at
vestibulum justo lectus et turpis. Maecenas lacinia convallis euismod.
Quisque egestas fermentum sapien eu dictum. Sed nec lacus in purus dictum
consequat quis vel nisl. Fusce non urna sem. Curabitur eu diam vitae elit
accumsan blandit. Nullam fermentum nunc et leo dictum laoreet. Donec semper
varius velit vel fringilla. Vivamus eu orci nunc.

\section{Besluit van dit hoofdstuk}
Als je in dit hoofdstuk tot belangrijke resultaten of besluiten gekomen
bent, dan is het ook logisch om het hoofdstuk af te ronden met een
overzicht ervan. Voor hoofdstukken zoals de inleiding en het
literatuuroverzicht is dit niet strikt nodig.

%%% Local Variables: 
%%% mode: latex
%%% TeX-master: "masterproef"
%%% End: 

% ... en zo verder tot
\chapter{Het laatste hoofdstuk}
\label{hoofdstuk:n}
Een hoofdstuk behandelt een samenhangend geheel dat min of meer op zichzelf
staat. Het is dan ook logisch dat het begint met een inleiding, namelijk
het gedeelte van de tekst dat je nu aan het lezen bent.

\section{Eerste onderwerp in dit hoofdstuk}
De inleidende informatie van dit onderwerp.

\subsection{Een item}
De bijbehorende tekst. Denk eraan om de paragrafen lang genoeg te maken en
de zinnen niet te lang.

Een paragraaf omvat een gedachtengang en bevat dus steeds een paar zinnen.
Een paragraaf die maar \'e\'en lijn lang is, is dus uit den boze.

\section{Tweede onderwerp in dit hoofdstuk}
Er zijn in een hoofdstuk verschillende onderwerpen. We zullen nu
veronderstellen dat dit het laatste onderwerp is.

\section{Besluit van dit hoofdstuk}
Als je in dit hoofdstuk tot belangrijke resultaten of besluiten gekomen
bent, dan is het ook logisch om het hoofdstuk af te ronden met een
overzicht ervan. Voor hoofdstukken zoals de inleiding en het
literatuuroverzicht is dit niet strikt nodig.

%%% Local Variables: 
%%% mode: latex
%%% TeX-master: "masterproef"
%%% End: 

\chapter{Besluit}
\label{chap:besluit}

\section{Conclusie} % 1 pagina % Sander
% - wat we gedaan hebben, onze doelen, hebben we de beste gevonden?
% - we hebben dat gevonden, ….

Deze thesistekst bestaat uit twee doelstellingen.
Een eerste doel is het definiëren van een methodologie om HTML5-raamwerken met elkaar te vergelijken.
Het tweede doel omvat de effectieve vergelijking van de raamwerken zelf.

De uitgelichte raamwerken zijn \st{}, \kendo{},  \jqm{} en \lungo{}.
\st{} bouwt op het MVC-ontwerppatroon en is \js-gedreven.
Het raamwerk is gratis en heeft zowel een commerciële als \term{open-source} licentie.
\kendo{} dwingt het MVVM-ontwerppatroon af en is zowel \js- als opmaakgedreven.
Een licentie voor het gebruik van \kendo{} kost $\$699$.
\jqm{} en \lungo{} hebben geen ontwerppatroon en zijn beide opmaakgedreven.
Beide raamwerken zijn \term{open-source}.

Vijf criteria werden gekozen om de vergelijkende studie uit te voeren:  populariteit,  productiviteit,  gebruik,  ondersteuning en performantie.
Elk criterium werd voorzien van een formule om een score te berekenen voor het criterium.
In samenspraak met Capgemini werd een POC opgesteld die werknemers toelaat onkosten toe te voegen.
Deze werd gebruikt om het gebruik en de ondersteuning te drijven.  
Om populariteit te meten werd naar de activiteit van de raamwerken op sociale netwerken gekeken.
De tijd om een loginapplicatie te ontwikkelen, bepaalde de productiviteit.
De POC werd onderverdeeld in $13$ uitdagingen en $38$ deeluitdagingen om de functionaliteit van het raamwerk te testen en het gebruik te quoteren.
Vervolgens werd een subset van de uitdagingen getest op acht verschillende mobiele toestellen om de ondersteuning te controleren.
Ten slotte bepaalden de downloadtijd en de gebruikerservaring van de loginapplicatie de performantie.
De gebruikerservaring bepaalt hoe vlot het gaat om door een lange lijst te scrollen.
De scores van de vijf criteria voor de vier raamwerken werden in één spinnenweb ondergebracht.

Na evaluatie is \jqm{} het beste raamwerk op basis van de gekozen criteria, gevolgd door \kendo{}, \lungo{} en \st{}.
Deze volgorde werd bepaald door de scores van alle criteria op te tellen.
\jqm{} heeft als belangrijkste troeven de hoge productiviteit en performantie doordat het enerzijds zeer goed gedocumenteerd is en anderzijds geen ontwerppatroon afdwingt.
Dit laatste is echter een nadeel waardoor het minder scoort op gebruik.
\kendo{} heeft als belangrijkste troef het gebruik doordat het een ontwerppatroon afdwingt.
Het scoort echter ondermaats op performantie door het crashen van lange lijsten op iOS.
\lungo{} behaalt op geen enkel criterium de maximumscore.
Het behaalde echter wel de beste downloadtijd bij performantie doordat het raamwerk is geoptimaliseerd voor mobiel gebruik.
\st{} is het minst productief en minst performant in vergelijking met de andere raamwerken.
Daarentegen scoort \st{} het best op het vlak van gebruikerservaring.
Door het afdwingen van een ontwerppatroon scoort het quasi evengoed als \kendo{} op vlak van gebruik.
Alle onderzochte raamwerken scoren zeer goed op ondersteuning.


\section{Geleerde lessen} % 1 pagina 
Bij het uitvoeren van een vergelijkende studie is de keuze van de vergelijkingscriteria bepalend voor het resultaat van het onderzoek.
Daarbij is het belangrijk om reeds bestaande literatuur grondig te controleren.
Zo werd er in het begin te specifiek naar papers gezocht omtrent het vergelijken van HTML5-raamwerken.
Meer algemene methodologieën om software te vergelijken moesten worden gezocht waaruit criteria konden worden hergebruikt.

Bij het opzetten van criteria is het van groot belang dat de manier waarop deze criteria zullen worden getoetst, zeer gedetailleerd worden neergeschreven.
De exacte formules van de criteria werden pas tijdens de evaluatie vastgelegd.
Pas dan werd de nood van een formele notatie duidelijk.
Ook is het belangrijk om op voorhand de elementen van een vergelijkende studie uit te testen alvorens de vergelijkingscriteria vast te leggen.
Hierdoor kunnen iteraties over de criteria vermeden worden als blijkt dat ze niet toepasbaar zijn.
Een initiële vertrouwdheid met de raamwerken had er ook voor gezorgd dat de criteria meer kenmerken van de raamwerken zouden bevatten.
De raamwerken die steunen op een ontwerppatroon konden op die manier meer bevoordeeld worden door bijvoorbeeld uitbreidbaarheid te introduceren, zoals dat ook in de ISO-25010 wordt gebruikt.

Een andere geleerde les is dat er in het uitvoeren van de evaluatie veel werk kruipt.
Echter, het interpreteren en begrijpen van de resultaten duurde zowaar nog langer.
Ook het opsporen en verbeteren van eigen fouten is zeer tijdsrovend.
Door op een consistente en gestructureerde methode de evaluatie te voltooien, moet het aantal fouten tot een minimum worden beperkt.

Er werden tot slot nog twee praktische zaken geleerd.
Het opmeten van tijd en het gebruik van logboeken vereist een zekere vorm van discipline.
Het eerste was noodzakelijk om de productiviteit van het raamwerk te toetsen.
De tijdbudgetten die nodig waren moesten gemakkelijk kunnen worden gereconstrueerd.
De logboeken werden gebruikt bij de implementatie van de POC en moesten bij de evaluatie het gebruikscriterium drijven.
Een tweede praktisch punt gaat over de samenwerking tussen beide auteurs.
Het is geen gemakkelijke opgave om als twee individuen continu op gelijke hoogte te zitten.
Dit vraagt enorm veel onderlinge gesprekken met duidelijke communicatie.
Discipline, sociale netwerken en andere technologieën zoals Google Drive en \gh{} hielpen de communicatie te verbeteren.

\section{Verder onderzoek} % 1 pagina % Tim
Er kan verder gezocht worden naar de oorzaak waarom bepaalde zaken zeer opmerkelijk waren of waarom ze niet lukten tijdens de vergelijking.
Zo was er enerzijds de crash van de 850 lijstitems van \kendo{} op iOS.
Hier kan gezocht worden naar de oorzaak van de crash, maar tevens kan ook gezocht worden naar de grens van het aantal lijstitems waarbij dat het wel lukt.
Anderzijds was er een opmerkelijke waarneming bij de performantie van de applicaties uit cache. 
Zo ligt bijvoorbeeld de gemiddelde downloadtijd van de login uit cache hoger dan deze van de POC voor \jqm{} en \lungo{}. 

Ook kunnen nieuwe raamwerken worden toegevoegd aan de vergelijking.
Hierdoor vergroot ten eerste de grootte van de vergelijking, maar kan ten tweede ook de methode telkens opnieuw worden getoetst met deze nieuwe raamwerken.
Daarnaast komen van de reeds vergeleken raamwerken geregeld nieuwe versies uit.
Zo is het ook mogelijk om de evolutie in de rangschikking van de vier vergeleken raamwerken over de tijd te bekijken.
Mogelijk kan de rangschikking veranderen bij het uitbrengen van nieuwe versies of plug-ins.

De huidige methode omvat vijf vergelijkingscriteria die worden gedreven door de POC.
Verder onderzoek kan deze POC uitbreiden met extra kenmerken zoals Pull\&Refresh.
Dit loopt in de lijn om ook andere gebeurtenissen te gebruiken dan alleen maar de \term{tap} gebeurtenis.
Andere gebeurtenissen zijn bijvoorbeeld \term{double tap}, \term{swipe}, \term{hold}, maar ook gebeurtenissen waar meerdere vingers voor nodig zijn zoals \term{rotate}.
Daarnaast is ook de integratie van HTML5-kenmerken zoals GPS, \term{push events}, \term{drag and drop}, video en audio in de raamwerken zeker het onderzoeken waard.

Naast het toevoegen van extra kenmerken aan de POC, kunnen criteria ook op andere manieren gecontroleerd worden.
Nu worden bij ondersteuning enkel apparaten met een Android- of iOS-besturingssysteem gebruikt.
Dit kan worden vervangen of uitgebreid naar andere besturingssystemen zoals Windows Phone en BlackBerry~OS.
Een andere voorbeeld is dat voor de downloadtijd bij performantie Wifi werd gebruikt voor de verbinding.
Andere verbindingsmogelijkheden zoals 3G kunnen worden gebruikt en hierdoor kunnen andere resultaten bekomen worden.
Een derde voorbeeld is de rendertijden die worden gebruikt bij performantie.
Deze konden niet worden opgemeten bij \st{} en \lungo{}.
Aanpassingen aan de broncode zouden het opmeten van rendertijden wel kunnen toelaten.
Daarnaast kan het gebruikte alternatief dat bepaalt hoe vlot het gaat om door een lange lijst te scrollen, ook verbeterd worden.
Als laatste kunnen ook de lijnen effectief geschreven code worden gebruikt bij productiviteit. 

Een andere onderzoekspiste is om nieuwe criteria toe te voegen.
Zo kan bijvoorbeeld het criterium uitbreidbaarheid worden onderzocht.
Dit criterium omvat hoe gemakkelijk het gaat om de bestaande applicatie geïmplementeerd in een bepaald raamwerk uit te breiden.
Een te onderzoeken hypothese hierbij is dat raamwerken die een ontwerppatroon afdwingen beter zullen scoren dan raamwerken zonder ontwerppatroon.
Een bijkomende hypothese is dat de totale score van raamwerken die een ontwerppatroon afdwingen zal stijgen en deze van raamwerken zonder ontwerppatroon zal dalen.
Dit komt doordat de ene worden afgestraft op productiviteit en de andere op uitbreidbaarheid.
Mogelijk kan de rangschikking van de vier onderzochte raamwerken veranderen.
Anderzijds kan ook een criterium worden toegevoegd die kijkt naar het finale resultaat van het raamwerk.
Zo kunnen bepaalde raamwerken de \term{native look-and-feel} van mobiele besturingssystemen  nabootsen, andere raamwerken bieden dan weer standaard een frisse hedendaagse lay-out.

Andere onderzoeksvragen kunnen een stap terugnemen door bijvoorbeeld af te vragen of het baterijverbruik door webapplicaties een probleem vormt.
De bekomen data kan worden vergeleken met \term{native} en hybride applicaties.
Deze laatste vergelijking kan zelfs veralgemeend worden waardoor een vergelijking tussen web-, \term{native} en hybride applicaties zich opdringt. 


%%% Local Variables: 
%%% mode: latex
%%% TeX-master: "masterproef"
%%% End: 


% Indien er bijlagen zijn:
\appendixpage*          % indien gewenst
\appendix
\chapter{De eerste bijlage}
\label{app:A}
In de bijlagen vindt men de data terug die nuttig kunnen zijn voor de
lezer, maar die niet essentieel zijn om het betoog in de normale tekst te
kunnen volgen. Voorbeelden hiervan zijn bronbestanden,
configuratie-informatie, langdradige wiskundige afleidingen, enz.

In een bijlage kunnen natuurlijk ook verdere onderverdelingen voorkomen,
evenals figuren en referenties\cite{h2g2}.

\section{Meer lorem}
\lipsum[50]

\subsection{Lorem 15--17}
\lipsum[15-17]

\subsection{Lorem 18--19}
\lipsum[18-19]

\section{Lorem 51}
\lipsum[51]

%%% Local Variables: 
%%% mode: latex
%%% TeX-master: "masterproef"
%%% End: 

% ... en zo verder tot
\chapter{De laatste bijlage}
\label{app:n}
In de bijlagen vindt men de data terug die nuttig kunnen zijn voor de
lezer, maar die niet essentieel zijn om het betoog in de normale tekst te
kunnen volgen. Voorbeelden hiervan zijn bronbestanden,
configuratie-informatie, langdradige wiskundige afleidingen, enz.

\section{Lorem 20-24}
\lipsum[20-24]

\section{Lorem 25-27}
\lipsum[25-27]

%%% Local Variables: 
%%% mode: latex
%%% TeX-master: "masterproef"
%%% End: 


\backmatter
% Na de bijlagen plaatst men nog de bibliografie.
% Je kan de  standaard "abbrv" bibliografiestijl vervangen door een andere.
\bibliographystyle{abbrv}
\bibliography{referenties}

\end{document}

%%% Local Variables: 
%%% mode: latex
%%% TeX-master: t
%%% End: 
