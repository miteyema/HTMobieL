\chapter{Inleiding} 
\label{inleiding}
%In dit hoofdstuk wordt het werk ingeleid. Het doel wordt gedefinieerd en er wordt uitgelegd wat de te volgen weg is (beter bekend als de rode draad).

\section{Achtergrondinformatie}
% - web apps (cross platform)
% - HTML5/js/css3
% - wat doet framework?
% - om de ontwikkeling vergemakkelijk worden frameworks aangeboden... 

Het gebruik van smartphones en tablets stijgt ontzettend snel in onze samenleving.
Voorheen was er de \term{feature phone} waarop enkel de voorgeïnstalleerde applicaties kon worden gebruikt.
Nu kunnen smartphones en tablets ook extra applicaties vanuit een winkel downloaden en installeren.
Ontwikkelaars van deze mobiele applicaties worden geconfronteerd met de variëteit aan mobiele besturingssystemen die op deze apparaten aanwezig zijn.
Dit komt doordat een applicatie dient te worden geprogrammeerd aan de hand van een SDK (Software Development Kit) die specifiek is voor het besturingssysteem.
Ontwikkelaars zullen dus eenzelfde applicatie in verschillende programmeertalen dienen te programmeren om een zo groot mogelijk publiek te bereiken.
Niet enkel het programmeren, maar ook het onderhoud van de applicaties in verschillende programmeertalen brengt een grote kost met zich mee.

Een oplossing hiervoor is het maken van een mobiele webapplicatie, gebruikmakend van HTML5.
Ten eerste wordt deze rechtstreeks in een webbrowser geopend en dus niet langer vanuit een winkel geïnstalleerd.
Dit betekent dus dat ieder mobiel apparaat dat een webbrowser heeft, de webapplicatie kan openen ongeacht zijn mobiel besturingssysteem.
Ten tweede wordt de applicatie slechts in één programmeertaal geschreven, wat de kost verlaagd.
Om het ontwikkelingsproces van deze mobiele HTML5-applicaties te versnellen worden raamwerken aangeboden die helpen bij de functionaliteit van de applicatie en de elementen voor de gebruikersinterface. 

\section{Probleembeschrijving}
% - heel veel frameworks, nog geen literatuur die vergelijkt (alleen blogs e.d.)
% - eerder voorstelling ipv objectieve verglijking.
% - beste framework?
% - hoe vergelijken?

Mobiele HTML5-raamwerken zijn er in overvloed en ook de verschillende versies van eenzelfde raamwerk volgen elkaar in snel tempo op.
In de huidige literatuur worden er vaak raamwerken aangehaald en besproken, maar niet vergeleken.
Indien deze toch worden vergeleken, gebeurt dit vaak subjectief of worden punten gegeven zonder een gestaafde methode te gebruiken.
Ook bestaat er geen literatuur die vergelijkingen van mobiele HTML5-raamwerken aggregeert.

\section{Doelstellingen}
% Is er een beste framework?
% contribuite:  methodologie uitwerken om OBJECTIEF en VISUEEL raamwerken te vergelijken

Deze thesistekst bestaat uit twee doelstellingen.
Een eerste doel is het definiëren van een methodologie om HTML5-raamwerken met elkaar te vergelijken.
Deze methodologie moet alle belangrijke aspecten van de raamwerken tegen het licht houden.
Ook moet er geprobeerd worden de werkwijze zo objectief mogelijk te laten verlopen en het resultaat van de studie op een eenvoudige,  visuele manier aan de lezer te presenteren.
Het tweede doel omvat de effectieve vergelijking van de raamwerken zelf.
Door de grote verscheidenheid van HTML5-raamwerken moeten de bestudeerde raamwerken zo worden gekozen dat ze zoveel mogelijk aspecten bevatten.
Hier komt ook de afweging tussen het aantal bestudeerde raamwerken en de diepte van de vergelijkende studie de kop op steken.
De raamwerken die worden gekozen moeten vervolgens worden vergeleken met de vooropgestelde methodologie.
Het resultaat moet alle positieve en negatieve aspecten van de raamwerken bevatten.
Vervolgens moet er gekeken worden of er één raamwerk het beste is of er verschillende raamwerken in verschillende situaties als beste kunnen worden bestempeld.

\section{Toepassingsgebied}
% mobiele wereld (mobile = booming)
% web (web = booming)
% kruising tussen web en mobile = super booming!
% bedrijfswereld (capgemini) HTML5 iets nieuws,  bedrijven kunnen nu een met een gerust hart een goede keuze maken (mss beter doelstellingen)

Mobiele HTML5-raamwerken vergemakkelijken de ontwikkeling van mobiele HTML5-applicaties.
Deze applicaties zijn toegankelijk via het web en geoptimaliseerd om op mobiele apparaten te kunnen werken.
Het aanspreken van mobiele applicaties via het web heeft zowel voor- als nadelen.
Zeker wanneer er wordt vergeleken met \term{native} of hybride applicaties.
De focus van deze studie ligt echter niet op het onderzoeken van deze voor- of nadelen.
Wel zullen de verschillende technologieën besproken worden om mobiele applicaties te maken.

Omdat Capgemini deze thesis mee ondersteunt zullen de applicaties gemaakt met HTML5-raamwerken vanuit een bedrijfscontext worden benaderd.
Dit zal vooral naar boven komen in de keuze van vergelijkingscriteria en de methode om deze criteria te testen.

\section{Overzicht}
Eerst wordt in hoofdstuk~\ref{chap:literatuurstudie} de basis van dit werk uitgelegd.
Vervolgens worden in hoofdstuk~\ref{chap:raamwerken} de vier gekozen raamwerken uitvoerig besproken.
Daarna zullen in hoofdstuk~\ref{chap:vergelijkingscriteria} de gekozen vergelijkingscriteria aan bod komen en verantwoord worden.
Hieropvolgend wordt in hoofdstuk~\ref{chap:evaluatie} deze vergelijking uitgevoerd op de gekozen raamwerken aan de hand van de gekozen criteria.
Als laatste wordt in hoofdstuk~\ref{chap:besluit} het besluit geformuleerd.

%%% Local Variables: 
%%% mode: latex
%%% TeX-master: "masterproef"
%%% End: 
