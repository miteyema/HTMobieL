%TODO: koen zegt geen secties in de inleiding van thesistekst

\chapter{Inleiding} %3 blz (0.75 per sectie)
\label{inleiding}
%In dit hoofdstuk wordt het werk ingeleid. Het doel wordt gedefinieerd en er wordt uitgelegd wat de te volgen weg is (beter bekend als de rode draad).

%\section{Achtergrondinformatie} %tim
% - web apps (cross platform)
% - HTML5/js/css3
% - wat doet framework?
% - om de ontwikkeling vergemakkelijk worden frameworks aangeboden... 

Het gebruik van smartphones en tablets stijgt ontzettend snel in onze samenleving.
Waar voorheen met de gsm enkel kon worden gebruik gemaakt van de voorgeïnstalleerde applicaties op het apparaat, kunnen op deze mobiele apparaten ook extra applicaties vanuit een winkel worden gedownload en geïnstalleerd.
Ontwikkelaars van deze mobiele applicaties worden geconfronteerd met de variëteit aan mobiele besturingssystemen die op deze apparaten aanwezig zijn.
% Er bestaat niet enkel verschillende besturingssystemen, maar ook verschillende versies van besturingssystemen.
Dit komt doordat een applicatie dient te worden geprogrammeerd in een programmeertaal die specifiek is tot het besturingssysteem.
Ontwikkelaars zullen dus eenzelfde applicatie in verschillende programmeertalen dienen te programmeren om een zo groot mogelijk publiek te bereiken.
Niet enkel het programmeren, maar ook het onderhoud van de applicaties in verschillende programmeertalen brengt een grote kost met zich mee.

Een oplossing hiervoor is het maken van een mobiele webapplicatie, gebruikmakend van HTML5.
Ten eerste wordt deze rechtstreeks in een webbrowser geopend en moet dus niet langer worden geïnstalleerd vanuit een winkel.
Dit betekent dus dat ieder mobiel apparaat die een webbrowser heeft, de webapplicatie kan openen ongeacht zijn mobiel besturingssysteem.
Ten tweede wordt de applicatie slechts in één programmeertaal geschreven, wat de kost omlaag brengt.
Om het ontwikkelingsproces van deze mobiele HTML5-applicaties te versnellen worden raamwerken aangeboden die helpen bij de functionaliteit van de applicatie en de elementen voor de gebruikersinterface. 

\section{Probleembeschrijving} %tim
% - heel veel frameworks, nog geen literatuur die vergelijkt (alleen blogs e.d.)
% - eerder voorstelling ipv objectieve verglijking.
% - beste framework?
% - hoe vergelijken?
Mobiele HTML5-raamwerken schieten als paddenstoelen uit de grond en ook de verschillende versies van eenzelfde raamwerk volgen elkaar in snel op.
In de huidige literatuur worden er vaak raamwerken aangehaald en besproken, maar niet vergeleken.
Indien deze toch worden vergeleken, gebeurt dit vaak subjectief of worden punten gegeven zonder een gestaafde methode te gebruiken.
Geen enkele literatuur bouwt verder op reeds bestaande literatuur van vergelijkingen voor HTML5-raamwerken of aggregeert de losstaande vergelijkingen tot één geheel.

\section{Doelstellingen} %sander
% Is er een beste framework?
% contribuite:  methodologie uitwerken om OBJECTIEF en VISUEEL raamwerken te vergelijken
Deze thesistekst bestaat uit twee grote doelstellingen die beide in de titel verborgen zitten.
Een eerste doel is het creëren van een methodologie om HTML5-raamwerken met elkaar te vergelijken.
Deze methodologie moet alle belangrijke aspecten van de raamwerken tegen het licht houden.
Ook moet er geprobeerd worden de werkwijze zo objectief mogelijk te laten verlopen en het resultaat van de studie op een eenvoudige,  visuele manier aan de lezer te presenteren.
Het tweede doel omdat de effectieve vergelijking van de raamwerken zelf.
Door de grote verscheidenheid van HTML5-raamwerken moeten de bestudeerde raamwerken zo worden gekozen dat ze zoveel mogelijk aspecten van alle raamwerken bevatten.
Hier komt ook de afweging tussen het aantal bestudeerde raamwerken en diepte van de vergelijkende studie de kop op steken.
De raamwerken die worden gekozen moeten vervolgens worden vergeleken met de vooropgestelde methodologie.
Het resultaat moet alle positieve en negatieve aspecten van de raamwerken bevatten.
Vervolgens moeten er gekeken worden of er één raamwerk het best is of er verschillende raamwerken in verschillende situaties als best kunnen worden bestempeld.

\section{Toepassingsgebied} %sander
% mobiele wereld (mobile = booming)
% web (web = booming)
% kruising tussen web en mobile = super booming!
% bedrijfswereld (capgemini) HTML5 iets nieuws,  bedrijven kunnen nu een met een gerust hart een goede keuze maken (mss beter doelstellingen)
HTML5-raamwerken vergemakkelijken de ontwikkeling van HTML5-applicaties.
Deze applicaties zijn toepasbaar op twee domeinen: web en mobiel.
Ze zijn toegankelijk via het web en geoptimaliseerd om op mobiele apparaten te kunnen werken.
Het aanspreken van mobiele applicaties via het web heeft zowel voor- als nadelen.
Zeker wanneer er wordt vergeleken met \term{native} of hybriede applicaties.
De focus van deze studie ligt echter niet bij het onderzoeken van deze voor- of nadelen!
Wel zullen de verschillende technologieën besproken worden om mobiele applicaties te maken.

Omdat Capgemini deze thesis mee ondersteunt zullen beide domeinen vanuit een bedrijfscontext worden benaderd.
Dit zal vooral naar boven komen in de keuze van vergelijkingscriteria en de methode om deze criteria te testen.


%\section{Rode draad} %tim
% - geen nadruk op passief actief
% - passieve en actieve vergelijkingscriteria al hier vermelden.
Eerst wordt in hoofdstuk~\ref{chap:literatuurstudie} de basis van dit werk uitgelegd.
Vervolgens wordt er in hoofdstuk~\ref{chap:raamwerken} een passieve vergelijking gedaan van de gekozen raamwerken.
Daarna zullen in hoofdstuk~\ref{chap:vergelijkingscriteria} de gekozen vergelijkingscriteria voor de actieve vergelijking aan bod komen en verantwoord worden.
Hieropvolgend wordt in hoofdstuk~\ref{chap:evaluatie} deze actieve vergelijking uitgevoerd op de gekozen raamwerken aan de hand van de gekozen criteria.
Als laatste wordt dit alles besloten in hoofdstuk~\ref{chap:besluit}.

%%% Local Variables: 
%%% mode: latex
%%% TeX-master: "masterproef"
%%% End: 
