\chapter{Inleiding}
\label{inleiding}
%TODO
%In dit hoofdstuk wordt het werk ingeleid. Het doel wordt gedefinieerd en er wordt uitgelegd wat de te volgen weg is (beter bekend als de rode draad).

%\section{Background info}
%\section{Problem description}
%\section{Objectives}
%\section{Scope}

In hoofdstuk~\ref{chap:literatuurstudie} wordt de basis van deze thesis uitgelegd.
Vervolgens wordt er in hoofdstuk~\ref{chap:raamwerken} dieper ingegaan op de gekozen raamwerken die dit werk zal vergelijken.
Daarna zullen in hoofdstuk~\ref{chap:vergelijkingscriteria} de gekozen vergelijkingscriteria aan bod komen en verantwoord worden.
Hieropvolgend wordt in hoofdstuk~\ref{chap:evaluatie} de evaluatie uitgevoerd op de gekozen raamwerken aan de hand van de gekozen criteria.
Als laatste wordt dit alles besloten in hoofdstuk~\ref{chap:besluit}.

%%% Local Variables: 
%%% mode: latex
%%% TeX-master: "masterproef"
%%% End: 
