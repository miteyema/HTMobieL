\chapter{Inleiding} %2 blz (0.5 per sectie)
\label{inleiding}
%TODO
%In dit hoofdstuk wordt het werk ingeleid. Het doel wordt gedefinieerd en er wordt uitgelegd wat de te volgen weg is (beter bekend als de rode draad).
%passieve en actieve vergelijkingscriteria al hier vermelden.

\section{Achtergrondinformatie} %tim
% web apps (cross platform)
% HTML5/js/css3
% wat doet framework?
% om de ontwikkeling vergemakkelijk worden frameworks aangeboden... 

\section{Probleembeschrijving} %tim
% heel veel frameworks, nog geen literatuur die vergelijkt (alleen blogs e.d.)
% eerder voorstelling ipv objectieve verglijking.
% beste framework?
% hoe vergelijken?

\section{Doelstellingen} %sander
% Is er een beste framework?
% contribuite:  methodologie uitwerken om OBJECTIEF en VISUEEL raamwerken te vergelijken

\section{Toepassingsgebied} %sander
% mobiele wereld (mobile = booming)
% web (web = booming)
% kruising tussen web en mobile = super booming!
% bedrijfswereld (capgemini) HTML5 iets nieuws,  bedrijven kunnen nu een met een gerust hart een goede keuze maken (mss beter doelstellingen)

% In hoofdstuk~\ref{chap:literatuurstudie} wordt de basis van deze thesis uitgelegd.
% Vervolgens wordt er in hoofdstuk~\ref{chap:raamwerken} dieper ingegaan op de gekozen raamwerken die dit werk zal vergelijken.
% Daarna zullen in hoofdstuk~\ref{chap:vergelijkingscriteria} de gekozen vergelijkingscriteria aan bod komen en verantwoord worden.
% Hieropvolgend wordt in hoofdstuk~\ref{chap:evaluatie} de evaluatie uitgevoerd op de gekozen raamwerken aan de hand van de gekozen criteria.
% Als laatste wordt dit alles besloten in hoofdstuk~\ref{chap:besluit}.

%%% Local Variables: 
%%% mode: latex
%%% TeX-master: "masterproef"
%%% End: 
