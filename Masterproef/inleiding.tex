\chapter{Inleiding} %3 blz (0.75 per sectie)
\label{inleiding}
%TODO
%In dit hoofdstuk wordt het werk ingeleid. Het doel wordt gedefinieerd en er wordt uitgelegd wat de te volgen weg is (beter bekend als de rode draad).

%\section{Achtergrondinformatie} %tim
% - web apps (cross platform)
% - HTML5/js/css3
% - wat doet framework?
% - om de ontwikkeling vergemakkelijk worden frameworks aangeboden... 
De vraag om applicaties te maken die op ieder mobiel apparaat werken, is zeer groot.
Een oplossing hiervoor is om mobiele webapplicaties te maken, gebruikmakend van HTML5.
Deze applicaties worden geopend in de browser, waardoor ieder mobiel apparaat die een browser heeft, ze kan openen.
Om het ontwikkelingsproces te versnellen worden raamwerken aangeboden die helpen bij de functionaliteit en elementen voor de gebruikersinterface. 

% \section{Probleembeschrijving} %tim
% - heel veel frameworks, nog geen literatuur die vergelijkt (alleen blogs e.d.)
% - eerder voorstelling ipv objectieve verglijking.
% - beste framework?
% - hoe vergelijken?
Deze HTML5-raamwerken schieten als paddenstoelen uit de grond, maar ook de verschillende versies van eenzelfde raamwerk volgen elkaar in snel tempo op.
In de huidige literatuur worden er vaak raamwerken aangehaald, maar niet vergeleken.
Hierdoor rijst de vraag of er een beste raamwerk bestaat die in alle situaties het beste scoort.
Een opmerking hierbij is dat bijvoorbeeld het beste raamwerk in een bepaalde situatie slechter scoort dan een ander.
Sommige literatuur vergelijkt wel degelijk raamwerken, maar dit gebeurt subjectief of zonder een gestaafde methode.
Geen enkele literatuur bouwt verder op reeds bestaande literatuur van vergelijkingen voor HTML5-raamwerken of aggregeert de losstaande vergelijkingen tot één geheel.

\section{Doelstellingen} %sander
% Is er een beste framework?
% contribuite:  methodologie uitwerken om OBJECTIEF en VISUEEL raamwerken te vergelijken

\section{Toepassingsgebied} %sander
% mobiele wereld (mobile = booming)
% web (web = booming)
% kruising tussen web en mobile = super booming!
% bedrijfswereld (capgemini) HTML5 iets nieuws,  bedrijven kunnen nu een met een gerust hart een goede keuze maken (mss beter doelstellingen)

%\section{Rode draad} %tim
% - geen nadruk op passief actief
% - passieve en actieve vergelijkingscriteria al hier vermelden.
Eerst wordt in hoofdstuk~\ref{chap:literatuurstudie} de basis van dit werk uitgelegd.
Vervolgens wordt er in hoofdstuk~\ref{chap:raamwerken} een passieve vergelijking gedaan van de gekozen raamwerken.
Daarna zullen in hoofdstuk~\ref{chap:vergelijkingscriteria} de gekozen vergelijkingscriteria voor de actieve vergelijking aan bod komen en verantwoord worden.
Hieropvolgend wordt in hoofdstuk~\ref{chap:evaluatie} deze actieve vergelijking uitgevoerd op de gekozen raamwerken aan de hand van de gekozen criteria.
Als laatste wordt dit alles besloten in hoofdstuk~\ref{chap:besluit}.

%%% Local Variables: 
%%% mode: latex
%%% TeX-master: "masterproef"
%%% End: 
