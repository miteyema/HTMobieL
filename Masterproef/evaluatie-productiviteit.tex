\section{Productiviteit}
\label{sec:evaluatie-productiviteit}
De productiviteit van de vier raamwerken wordt samengevat in tabel~\ref{tabel:evaluatie-productiviteit}. 
Daarna zullen de geïmplementeerde POC's~(zie \ref{sec:evaluatie-productiviteit-poc}) en loginschermen~(zie \ref{sec:evaluatie-productiviteit-login}) per raamwerk uitvoerig worden besproken.

\begin{table}[H]
\centering
\pgfplotstabletypeset[
  begin table=\begin{tabular}{p{8cm} p{1cm} p{1cm} p{1cm} p{1cm}},
  end table=\end{tabular},
  skip coltypes=true,
  col sep=comma,
  string type,
  header=true,
  columns={Productiviteit,jQM,ST,Kendo,Lungo},
  columns/Productiviteit/.style={column name=\textbf{Productiviteit}, column type={l}},  
  columns/jQM/.style={column name=\textbf{\jqma}, column type={c}},
  columns/ST/.style={column name=\textbf{\sta}, column type={c}},
  columns/Lungo/.style={column name=\textbf{\lungoa}, column type={c}},
  columns/Kendo/.style={column name=\textbf{\kendoa}, column type={c}},
  every head row/.style={
    before row=\toprule,
    after row=\midrule},
  every last row/.style={
  	before row=\midrule,
    after row=\bottomrule}
]{tabellen/productiviteit.csv}
\caption{Samenvattende tabel voor productiviteitscriterium}
\label{tabel:evaluatie-productiviteit}
\end{table}

%%%%%%%%%%%%%%%%%
% 
% \subsection{POC}
% \label{sec:evaluatie-productiviteit-poc}
% 
% \paragraph{\jqm}
% 
% \paragraph{\st}
% 
% \paragraph{\kendo}
% 
% \paragraph{\lungo}
% 
% %%%%%%%%%%%%%%%%%
% 
% \subsection{Loginscherm}
% \label{sec:evaluatie-productiviteit-login}
% 
% \paragraph{\jqm}
% 
% \paragraph{\st}
% Eerst werden Getting Started en Building Your First App gevolgd, wat tutorials van \st{} zelf zijn.
% Hiermee worden de initiële configuraties bekomen voor een werkende applicatie.
% Nadelen aan deze twee tutorials waren vele inconsistenties en soms foute links.
% 
% Daarna werd er gericht gezocht hoe een koptekst en formulier konden worden gemaakt.
% Hiervoor werd terug gebruik gemaakt van de documentatiesite van \st.
% Ook hierop werd beroep gedaan om een model voor de gebruiker te maken met de nodige validaties.
% De verzendknop voor het formulier werd gevonden in een antwoord op een StackOverflow-vraag.
% 
% \paragraph{\kendo}
% Er werd gestart met een inleidende pagina op de documentatiesite van \kendo{} zelf.
% Hiermee werd de minimale code bekomen voor een werkende applicatie.



\paragraph{\lungo}