\section{Productiviteit}
\label{sec:evaluatie-productiviteit}
% De productiviteit van de vier raamwerken wordt samengevat in tabel
% Daarna zullen de geïmplementeerde POC's~(zie \ref{sec:evaluatie-productiviteit-poc}) en loginschermen~(zie \ref{sec:evaluatie-productiviteit-login}) per raamwerk uitvoerig worden besproken.

In deze sectie zal de productiviteit van de vier raamwerken worden onderzocht.
Voor de score van productiviteit wordt naar formule \ref{eq:productiviteit} verwezen.
Tabel~\ref{tabel:evaluatie-productiviteit} bevat een overzicht van de behaalde scores.

\begin{table}[H]
\centering
\pgfplotstabletypeset[
  begin table=\begin{tabular}{p{8cm} p{1cm} p{1cm} p{1cm} p{1cm}},
  end table=\end{tabular},
  skip coltypes=true,
  col sep=comma,
  string type,
  header=true,
  columns={Productiviteit,ST,Kendo,jQM,Lungo},
  columns/Productiviteit/.style={column name=\textbf{Productiviteit}, column type={l}},  
  columns/ST/.style={column name=\textbf{\sta}, column type={c}},
  columns/ST/.style={column name=\textbf{\sta}, column type={c}},
  columns/jQM/.style={column name=\textbf{\jqma}, column type={c}},
  columns/Lungo/.style={column name=\textbf{\lungoa}, column type={c}},
  columns/Kendo/.style={column name=\textbf{\kendoa}, column type={c}},
  every head row/.style={
    before row=\toprule,
    after row=\midrule},
  every last row/.style={
  	before row=\midrule,
    after row=\bottomrule}
]{tabellen/productiviteit.csv}
\caption{Overzicht van productiviteit voor \st{}~(\sta), \kendo{}~(\kendoa), \jqm{}~(\jqma) en \lungo{}~(\lungoa).}
\label{tabel:evaluatie-productiviteit}
\end{table}

\lungo{} is de afgetekende winnaar,  gevold door \kendo{} en \jqm{}. 
\st{} blijkt het minst productief te zijn.
Het type raamwerk (\js-gedreven of opmaakgedreven) en het al dan niet afdwingen van een architectuur weerspiegeld zich in deze resultaten.
De verschillen tussen \jqm{} en \lungo{} enerzijds en \st{} en \kendo{} anderzijds zijn groter want ze bieden een ander type - respectievelijk \js-gedreven en zowel \js- als opmaakgedreven - en hanteren een andere architectuur - respectievelijk MVC en MVVM.
De productiviteit van \lungo{} stijgt met $71\%$ ten opzichte van \jqm{}.
De productiviteit van \kendo{} stijgt met $44\%$ ten opzichte van \st{}.

De belangrijkste factor waarom \lungo{} en \kendo{} beter scoren is omdat beide raamwerken als tweede werden behandeld.
De POC werd eerst in \jqm{} en \st{} geïmplementeerd.
Er zijn vijf redenen waarom dit de data sterk beinvloedt.
%TODO deze vijf punten met itemize ofzo?
%3 en 5
De eerste twee reden omvatten een betere ervaring met de POC en HTML5-raamwerken in het algemeen.
De vertrouwdheid met HTML5-raamwerken weerspiegeld zich vooral tussen \jqm{} en \lungo{}.
Hoewel ze beide op een verschillende \js{}-bibliotheek steunen - respectievelijk jQuery en QuoJS - zijn de gelijkenissen tussen deze twee raamwerken niet te ontkennen.
%4
Verder kon er ook veel code,  zoals gebruikt bij \jqm{},  overgenomen worden bij de implementatie van de POC met \lungo.
%2
Ook kwamen er bij de eerste implementatie problemen met de \term{backend} naar boven.
Deze waren bij de tweede implementatie reeds opgelost.
Door het onnauwkeurig opmeten van de tijd met Toggl kan er geen schatting worgen gemaakt van de tijd die aan de \term{backend} werd besteed.
Er kan echter niet ontkend worden dat dit een probleem was.%TODO ok zo?
%1
Een laatste reden die de productiviteitsgegevens sterk beinvloed is dat niet de volledige POC met \lungo{} kon worden ontwikkeld.
Hierop zal in sectie \ref{sec:evaluatie-gebruik} verder worden ingegaan.

De werkuren van de login applicatie bevestigen voorgaande resulaten niet.
Ze zijn dan ook niet onderheven aan de vijf zonet opgenoemde redenen.
Bij elke implementatie werd met dezelfde achtergrondkennis gestart.  
% 3
De implementatie van de login applicatie is triviaal en eenduidig en dus geldt er voor alle raamwerken dat de ervaring met de applicatie reeds hoog was.
% 5
Ook werd eerst de implementatie van de POC gemaakt voordat aan de login applicatie werd begonnen.
Hierdoor was de algemene ervaring met HTML5-raamwerken gelijkgesteld.
% 4 en 2
Verder werd er geen code gekopieerd en waren er geen problemen met de \term{backend}
%  1
Ten slotte werd alle functionaliteit van de login applictie met alle vier raamwerken gebouwd.
Om al deze redenen werd beslist de score voor productiviteit aan te passen door enkel de uren van de login applicatie ($t_{r,login}$) te beschouwen.
De verbeterde formule voor productiviteit is dan
\begin{equation}
  \text{Productiviteit}'_r = t_{r,login}
  \label{eq:productiviteit-enhanced}
\end{equation}
waar $r$ de verschillende raamwerken voorstellen.

Uit deze data blijkt \jqm{} productiever dan \lungo{}. 
De volgorde van \kendo{} en \st{} blijft echter wel behouden.
De relatieve verschillen tussen \jqm{} en \lungo{} en tussen \st{} en \kendo{} zijn ook verkleind.  
Respectievelijk een stijging van $11.7\%$ en een daling met $31.8\%$.
Toch is er een trend vast te stellen:  opmaakgedreven raamwerken blijken productiever dan \js-gedreven raamwerken.

In wat volgt zullen andere factoren worden besproken die de verschillen in productiviteit kunnen verklaren.

\subsection{Tools}
Enkel bij de ontwikkeling met \st{} kon beroep worden gedaan op een grafische tool om het ontwikkelingsproces te vergemakkelijken:  Sencha Architect~\cite{Sencha2012a}.
Dit is een desktopapplicatie die het ontwikkelingsproces vergemakkelijkt met een GGI en \term{drag-and-drop} commando's.  
Bij de ontwikkeling van de POC werd Sencha Architect versie 2.1 gebruikt voor een tijdsduur van $21$ dagen.
Het grootste voordeel van Sencha Architect kon bij de ontwikkeling van \code{Views} worden gevonden.
De \code{Views} kunnen specifiek voor mobiele schermen worden geoptimaliseerd.
Dit zowel voor staande als liggende apparaten.
Na $21$ dagen werd overgeschakeld op PhpStorm als IDE.
Dezelfde IDE werd ook gebruikt bij de ontwikkeling met andere raamwerken en biedt ondersteuning voor zowel Windows, Mac als Linux.
Een voordeel bij het gebruik van deze IDE is onder andere de automatische code-aanvulling.
Daarnaast toont de IDE hints voor het optimaliseren van \js{}-code die specifiek gebruikt maakt van de jQuery bibliotheek.

%TODO: vermelden van Yeoman + twitter bower?

%TODO: incorperen als we dan over Yeoman vertellen
%Dit is bij \jqm{} niet mogelijk doordat \jqm{} geen geldige component is van Bower.


\subsection{Boilerplate code}
Het initialiseren van een nieuwe applicatie van het raamwerk beïnvloed ook sterk de productiviteit.
\st{} biedt hiervoor terug een tool,  Sencha Cmd~\cite{Sencha2012}.
Deze tool kan de initiële applicaties opzetten,  bestanden toevoegen en de applicatie bouwen en uitrollen.
Na de geautomatiseerde initialisatie van een project zijn de gemaakte folders en bestanden echter niet geheel duidelijk.
Een nieuwe \code{Controller},  \code{Store},  \code{Model} of \code{View} genereren kan manueel of automatisch met Sencha Cmd.
Bij de manuele aanpak moet het nieuwe \js-bestand in de juiste folder worden ondergebracht.
\st{} legt een strenge structuur van folders op die ongewijzigd moet blijven opdat de applicatie zou werken.
%TODO weglaten De ervaringen van de auteurs met \st{} zeggen dat deze structuur tot veel verwarring leidt.

\jqm{} en \kendo{} beschrijven boilerplate code in hun documentatie~\cite{JQuery2012b,Telerikd}.
Ook is bij de documentatie van beide raamwerken een expliciete sectie aanwezig die de programmeur helpt om een applicatie op te zetten.
Bij \lungo{} is dit niet het geval.  
Om een \lungo{} applicatie op te zetten, moest naar de broncode van de voorbeelden op de documentatie gekeken worden.

\subsection{Documentatie}
Ook de kwaliteit en kwantiteit van de documentatie kunnen de scores van de productiviteit verklaren.
De \st{} documentatie is het grootst in vergelijking met de andere raamwerken.
De grootte van de documentatie maakt het moeilijk om zelf gericht naar onduidelijkheden te zoeken.
De zoekfunctie met auto-aanvulling is noodzakelijk om de juiste documentatie terug te vinden.
De documentatie van \kendo{} is overzichtelijker en volledig.
De combinatie van de secties API en Getting Started boden de voornaamste hulp.
Ook worden alle kenmerken die het raamwerk aanbiedt met demo's en codevoorbeelden getoond.
De zoekfunctie van de documentatie is niet optimaal en hierdoor werd deze maar weinig gebruikt.
De documentatie van \jqm{}~1.2 bevat geen zoekfunctie en codevoorbeelden.
Hierdoor moet naar de broncode worden gekeken om de code van een kenmerk te begrijpen.
Een voordeel van de \jqm{} documentatie is dat de documentatie zelf met \jqm{} is gebouwd.
Een belangrijke opmerking is dat de \jqm{}~1.3 wel een zoekfunctie en codevoorbeelden in zijn documentatie heeft opgenomen.
De documentatie van \lungo{} is zeer beknopt.
Dezelfde opmerking kan worden gemaakt bij de documentatie van \quo.
Alle kenmerken van \lungo{} worden met codevoorbeelden verduidelijkt.
Sommige voorbeelden zijn echter incorrect.

\subsection{Debugging}
Het zoeken naar bugs in de code verliep bij elk raamwerk gelijkaardig.
De applicatie werd lokaal uitgerold en in de Web Inspector van Chrome gedebugd.
Debuggen op de apparaten kon door deze te connecteren via USB met de computer.
Android toestellen kunnen gedebugd worden met de Web Inspector van de Chrome browser.
Debuggen op iOS toestellen kan op een Mac met dezelfde Web Inspector maar in de Safari browser.

\subsection{Literatuur}
%TODO et. al na auteurs vermelden?
Zowel \jqm{} als \st{} werden vaak in de literatuur aangehaald.
Op Safari Books Online kunnen 13 boeken teruggevonden worden die volledig op \jqm{} zijn toegespitst.
Het boek van Dutson in de Sam Teach Yourself serie werd gebruikt om met \jqm{} vertrouwd te geraken~\cite{PhilDutson2012}.
Ook Pro jQuery Mobile van Broulik~\cite{Broulik2012} werd gelezen maar dit is nagenoeg een kopie van de documentatie en bevat bijgevolg niet veel extra informatie.
Vier boeken over \st{} kunnen op Safari Books Online worden teruggevonden.
Het boek van Clark werd gebruikt~\cite{JohnEClark2012}.
\kendo{} komt slechts in een boek voor~\cite{Bhandari2013},  \lungo{} helemaal niet.
Het boek rond \kendo{} is slechts een proefdruk en wordt pas in augustus 2013 officieel geplubliceerd.
Dit werk werd niet gebruikt.


\subsection{Vragen}
Wanneer de programmeur met problemen van het raamwerk werd geconfronteerd, kan op het web naar oplossingen gezocht worden.
In sectie \ref{sec:evaluatie-populariteit} werden reeds het aantal vragen van het raamwerk op \so{} bekeken.
Hoe groter dit aantal,  hoe groter de kans dat een probleem reeds is aangehaald.
Bij \lungo{} is dit aantal minimaal en was de programmeur vaak aangewezen om zelf oplossingen te zoeken.
Voor \st{} en \kendo{} geldt dat ook het forum voor professionele ondersteuning hulpvaardig was.
Het plaatsen van vragen bij \st{} is onmogelijk zonder betaalde ondersteuning,  \kendo{} laat het stellen van maximaal vijf vragen toe zonder de aankoop van een licentie.
De meest voorkomende problemen zijn op de fora al aangehaald en gedetailleerd besproken.


\subsection{Lijnen code}
Als laatste wordt in tabel \ref{tabel:evaluatie-gebruik-loc} het aantal lijnen geschreven code getoond dat nodig is om de POC en login applicatie te maken.
Het type van het raamwerk (\js- of opmaakgedreven) komt sterk in de tabel naar boven.

\begin{table}[H]
\centering
\pgfplotstabletypeset[
	column type=l,
	every head row/.style={
		before row={%
			\toprule
			\textbf{Code}
			& \multicolumn{2}{c}{\textbf{\sta}}
			& \multicolumn{2}{c}{\textbf{\kendoa}} 
			& \multicolumn{2}{c}{\textbf{\jqma}}
			& \multicolumn{2}{c}{\textbf{\lungoa}} \\
			\cmidrule(r){2-3}
			\cmidrule(r){4-5}
			\cmidrule(r){6-7}
			\cmidrule(r){8-9}
		},
		after row=\midrule,
		},
  	every last row/.style={
  		before row=\toprule,
 		after row=\bottomrule},
	columns={Geschreven,ST(poc),ST(login),Kendo(poc),Kendo(login),jQM(poc),jQM(login),Lungo(poc),Lungo(login)},
	begin table=\begin{tabular*}{\textwidth}{@{\extracolsep{\fill}} lcccccccc},
	end table=\end{tabular*},
	header=true,
	skip coltypes=true,
	columns/Geschreven/.style ={column name=},
	columns/ST(poc)/.style ={column name=POC},
	columns/ST(login)/.style={column name=Login},
	columns/Kendo(poc)/.style ={column name=POC},
	columns/Kendo(login)/.style={column name=Login},
	columns/jQM(poc)/.style ={column name=POC},
	columns/jQM(login)/.style={column name=Login},
	columns/Lungo(poc)/.style ={column name=POC},
	columns/Lungo(login)/.style={column name=Login},
	col sep=comma,
	string type,
]{tabellen/gebruik/loc.csv}
\caption{Overzicht van het aantal lijnen geschreven code.}
\label{tabel:evaluatie-gebruik-loc}
\end{table}